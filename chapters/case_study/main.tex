%!TEX root = ../../main.tex
\chapter{Μελέτη Περίπτωσης}\label{ch:case_study}
Σε αυτό το κεφάλαιο παρουσιάζεται μια λεπτομερής ανάλυση του αυτοματοποιημένου συστήματος παραγωγής που χρησιμοποιείται για την κατασκευή των καρεκλών γραφείου τύπου \en{Gregor}. Επικεντρώνεται στην αρχιτεκτονική του συστήματος, τις βασικές λειτουργίες και τα ρομποτικά στοιχεία που επιτρέπουν την αυτόματη συναρμολόγηση με υψηλή ακρίβεια και επαναληψιμότητα. Θα εξετάσουμε τις διάφορες διαδικασίες που εμπλέκονται, καθώς και το πώς το σύστημα διαχειρίζεται την ενσωμάτωση των διαφορετικών συναρμολογήσεων για να παράγει ένα τελικό προϊόν που συμμορφώνεται με τα υψηλά πρότυπα ποιότητας.

\section{Επισκόπηση Συστήματος}
\noindent Το σύστημα αυτοματοποιημένης παραγωγής που αναπτύχθηκε για την κατασκευή των καρεκλών γραφείου τύπου \en{Gregor} χαρακτηρίζεται από μια σειρά καλά δομημένων σταδίων συναρμολόγησης, τα οποία διαχειρίζονται με ακρίβεια από εξειδικευμένα ρομπότ. Κάθε ρομπότ είναι ανατεθειμένο με συγκεκριμένες εργασίες, σχεδιασμένες για να εκτελούνται με βάση τη διαθεσιμότητα και την κατάσταση των εξαρτημάτων στις εργασιακές θέσεις.

Οι διαδικασίες παραγωγής ξεκινούν με τη συναρμολόγηση των βασικών δομικών στοιχείων, όπως είναι οι βάσεις και τα πόδια των καρεκλών. Το Ρομπότ 1 αναλαμβάνει την εκτέλεση του αρχικού σταδίου συναρμολόγησης, τοποθετώντας και ενώνοντας τα πόδια με τη βάση. Στη συνέχεια, το Ρομπότ 2 εισέρχεται για να προσθέσει επιπλέον δομικά στοιχεία όπως τα ροδάκια και τον μηχανισμό ανύψωσης, ολοκληρώνοντας το σκελετό της καρέκλας.

Η διαδικασία συνεχίζεται με τη συναρμολόγηση του καθίσματος και της πλάτης, που γίνεται σε χωριστές φάσεις και απαιτεί την εναλλαγή των ρομπότ ανάλογα με το στάδιο που βρίσκεται η καρέκλα στην παραγωγική γραμμή. Η προσθήκη των μπράτσων και της τελικής συναρμολόγησης είναι επιφορτισμένη στο Ρομπότ 3, το οποίο διαχειρίζεται τα τελευταία στάδια της διαδικασίας με μεγάλη ακρίβεια.

Κάθε ρομπότ υποβάλλει αιτήματα για να λάβει πρόσβαση στις εργασιακές θέσεις των πάγκων και διασφαλίζει ότι τα εξαρτήματα βρίσκονται στην κατάλληλη κατάσταση πριν ξεκινήσει η κάθε διαδικασία. Αυτό το σύστημα επιτρέπει την αποφυγή λαθών και συγκρούσεων, βελτιώνοντας τη συνολική αποδοτικότητα και ποιότητα της παραγωγής.

\section{Στάδια Παραγωγής}
\noindent Η διαδικασία παραγωγής των καρεκλών γραφείου τύπου \en{Gregor} διαχωρίζεται σε καθορισμένα στάδια που εξασφαλίζουν την ακριβή και αποδοτική συναρμολόγηση κάθε μέρους της καρέκλας. Αρχικά, το στάδιο \textit{\en{S1}} υποδέχεται την εναρκτήρια δράση από το Ρομπότ 1, το οποίο συναρμολογεί τα πόδια με την κεντρική βάση, δημιουργώντας τον πρωταρχικό σκελετό της καρέκλας. Συνεχίζοντας, στο στάδιο \textit{\en{S2}}, το Ρομπότ 2 προσθέτει τα ροδάκια και τον μηχανισμό ανύψωσης, ενισχύοντας τη βάση για καλύτερη κινητικότητα και ρυθμιζόμενη στήριξη.

Η συνέχεια της κατασκευής μεταφέρεται ξανά στο Ρομπότ 1, το οποίο στο στάδιο \textit{\en{S3}} προετοιμάζει το κάθισμα ενσωματώνοντάς το με την πλάκα καθίσματος. Ακολούθως, στο στάδιο \textit{\en{S4}}, το Ρομπότ 2 αναλαμβάνει να ενώσει το κάθισμα με την προετοιμασμένη βάση, ολοκληρώνοντας την ενσωμάτωση των κυριότερων μερών της καρέκλας.

Τα τελευταία στάδια εντάσσουν τη δράση του Ρομπότ 3, όπου το \textit{\en{S5}} αφορά την προσθήκη της πλάτης της καρέκλας. Στο \textit{\en{S6}} και \textit{\en{S7}}, το ίδιο ρομπότ προσθέτει τα μπράτσα, το αριστερό και το δεξί αντίστοιχα, ενισχύοντας τη λειτουργικότητα και την άνεση του σχεδίου. Το τελικό στάδιο, \textit{\en{S8}}, σηματοδοτεί την πλήρη ολοκλήρωση της καρέκλας, με την κατασκευή να έχει ολοκληρωθεί πλήρως και την καρέκλα να είναι έτοιμη για χρήση.

Η οργανωμένη διαδικασία και η σειριακή ολοκλήρωση των σταδίων διασφαλίζουν ότι κάθε μέρος της καρέκλας συναρμολογείται με ακρίβεια και αποδοτικότητα, αποφεύγοντας επαναλήψεις ή παραλείψεις στην παραγωγική διαδικασία.

\section{Ρόλοι και Ευθύνες}

\subsection{Ρομπότ 1}
\noindent Το Ρομπότ 1 κατέχει κεντρικό ρόλο στην αυτοματοποιημένη γραμμή παραγωγής των καρεκλών γραφείου τύπου \en{Gregor}, με τις ευθύνες του να επικεντρώνονται σε κρίσιμες εργασίες συναρμολόγησης στην αρχική φάση της κατασκευής. Αρχικά, το Ρομπότ 1 εκτελεί την Εργασία Συναρμολόγησης 1 (\en{AT1}), όπου είναι υπεύθυνο για τη συναρμολόγηση της βάσης της καρέκλας, ενώνοντας τα πόδια με την κεντρική βάση στον πρώτο πάγκο εργασίας. Το αντίστοιχο διάγραμμα ροής του Ρομπότ 1 φαίνεται στο Σχήμα \ref{fig:robot1_flowchart}.

Κατόπιν, στην Εργασία Συναρμολόγησης 4 (\en{AT4}), το Ρομπότ 1 επιστρέφει για να ενσωματώσει το κάθισμα με την πλάκα καθίσματος, ρυθμίζοντας τα για την τελική τοποθέτηση. Η διαδικασία αυτή απαιτεί υψηλή ακρίβεια και είναι κρίσιμη για την άνεση και την ασφάλεια της τελικής κατασκευής.

Το Ρομπότ 1 έχει επίσης την ευθύνη να επαναλάβει τις εργασίες συναρμολόγησης 1 και 4 (\en{AT1} και \en{AT4}) σε κυκλική διαδικασία, εξασφαλίζοντας τη συνεχή παραγωγή και αποδοτική λειτουργία της γραμμής παραγωγής. Αυτή η κυκλική και επαναληπτική δράση διασφαλίζει ότι η παραγωγή δεν διακόπτεται και ότι κάθε καρέκλα παράγεται με συνέπεια σύμφωνα με τα υψηλά πρότυπα ποιότητας του συστήματος.

\begin{figure}[H]
    \centering
    \begin{tikzpicture}[node distance=1.6cm]
        \node (start) [start] {};
        \node (at1) [process, below of=start] {\en{Assembly Task 1}};
        \node (at4) [process, below of=at1] {\en{Assembly Task 4}};

        \draw [arrow] (start) -- (at1);
        \draw [arrow] (at1) -- (at4) node[midway,right,overlay] {\en{Move to PosB}};
        \draw [arrow] (at4.east) -| ++(1.5,0) -- node[midway,right,overlay] {\en{Move to PosA}} ($(at1.east)+(1.5,0)$) |- (at1.east);
    \end{tikzpicture}
    \caption{Διάγραμμα Ροής Ρομπότ 1}
    \label{fig:robot1_flowchart}
\end{figure}

\subsection{Ρομπότ 2}
\noindent Το Ρομπότ 2 εκτελεί ζωτικές λειτουργίες στην παραγωγή της καρέκλας \en{Gregor}, εστιάζοντας στη συναρμολόγηση βασικών εξαρτημάτων για τη λειτουργικότητα της καρέκλας.  Το αντίστοιχο διάγραμμα ροής του Ρομπότ 2 φαίνεται στο Σχήμα \ref{fig:robot2_flowchart}.

Στην Εργασία Συναρμολόγησης 2 (\en{AT2}), το Ρομπότ 2 τοποθετεί τα ροδάκια και τον μηχανισμό ανύψωσης στην βάση της καρέκλας στον πρώτο πάγκο εργασίας. Στην Εργασία Συναρμολόγησης 3 (\en{AT3}), το Ρομπότ 2 αναλαμβάνει την ενσωμάτωση της πλάκας καθίσματος στην βάση, απαιτώντας ακρίβεια στην τοποθέτηση για να διασφαλιστεί η άνεση και σταθερότητα του καθίσματος. Τέλος, στην Εργασία Συναρμολόγησης 5 (\en{AT5}), το Ρομπότ 2 συναρμολογεί το τελικό κομμάτι της καρέκλας, ενώνοντας το κάθισμα με το υπόλοιπο σκελετό για να ολοκληρώσει τη δομή της καρέκλας.

\begin{figure}[H]
    \centering
    \begin{tikzpicture}[node distance=1.6cm]
        \node (start) [start] {};
        \node (at2) [process, below of=start] {\en{Assembly Task 2}};
        \node (at3) [process, below of=at2] {\en{Assembly Task 3}};
        \node (at5) [process, below of=at3] {\en{Assembly Task 5}};

        \draw [arrow] (start) -- (at2);
        \draw [arrow] (at2) -- (at3) node[midway,right,overlay] {\en{Move to PosB}};
        \draw [arrow] (at3) -- (at5) node[midway,right,overlay] {\en{Move to PosA}};
        \draw [arrow] (at5.east) -| ++(1.5,0) -- ($(at2.east)+(1.5,0)$) |- (at2.east);
    \end{tikzpicture}
    \caption{Διάγραμμα Ροής Ρομπότ 2}
    \label{fig:robot2_flowchart}
\end{figure}

\subsection{Ρομπότ 3}
\noindent Το Ρομπότ 3 είναι αρμόδιο για την τελική φάση της συναρμολόγησης των καρεκλών γραφείου \en{Gregor}, ολοκληρώνοντας τη δομή με την προσθήκη κρίσιμων εξαρτημάτων για τη λειτουργικότητα και άνεση του τελικού προϊόντος. Το αντίστοιχο διάγραμμα ροής του Ρομπότ 3 φαίνεται στο Σχήμα \ref{fig:robot3_flowchart}.

Συγκεκριμένα, στην Εργασία Συναρμολόγησης 6 (\en{AT6}), το Ρομπότ 3 είναι υπεύθυνο για την τοποθέτηση της πλάτης της καρέκλας. Αυτή η εργασία απαιτεί ακρίβεια για να εξασφαλιστεί ότι η πλάτη στερεώνεται σωστά στον σκελετό, παρέχοντας την απαραίτητη στήριξη και άνεση.

Στα επόμενα στάδια, Εργασία Συναρμολόγησης 7 και Εργασία Συναρμολόγησης 8 (\en{AT7} και \en{AT8}), το Ρομπότ 3 προχωρά στην τοποθέτηση των μπράτσων της καρέκλας, το αριστερό και το δεξί αντίστοιχα. Κάθε μπράτσο πρέπει να ενσωματωθεί με ακρίβεια για να διασφαλιστεί η άνεση και η σταθερότητα της καρέκλας, καθώς και η εργονομία για τον τελικό χρήστη.

Οι εργασίες που εκτελεί το Ρομπότ 3 είναι ουσιαστικές για την ολοκλήρωση της κατασκευής της καρέκλας, διασφαλίζοντας ότι το τελικό προϊόν είναι πλήρως λειτουργικό και άνετο. Η ακριβής εκτέλεση αυτών των τελευταίων βημάτων καθορίζει την ποιότητα και την εμφάνιση της καρέκλας, καταστρώνοντας τη βάση για μια επιτυχημένη παραγωγή.

\begin{figure}[H]
    \begin{center}
    \begin{tikzpicture}[node distance=1.6cm]
        \node (start) [start] {};
        \node (at6) [process, below of=start] {\en{Assembly Task 6}};
        \node (at7) [process, below of=at6, left of=at6,xshift=-4] {\en{Assembly Task 7}};
        \node (at8) [process, below of=at6, right of=at6,xshift=4] {\en{Assembly Task 8}};
        \node (comp) [process, below of=at8, left of=at8] {\en{Complete}};

        \draw [arrow] (start) -- (at6);
        \draw [arrow] (at6) -- (at7);
        \draw [arrow] (at6) -- (at8);
        \draw [arrow] (at7) -- (comp);
        \draw [arrow] (at8) -- (comp);
        \draw [arrow] (comp.east) -| ++(2.5,0) -- ($(comp.east)+(2.5,0)$) |- (at6.east);
    \end{tikzpicture}
    \caption{Διάγραμμα Ροής Ρομπότ 3}
    \label{fig:robot3_flowchart}
    \end{center}
\end{figure}

\subsection{Πάγκος Εργασίας 1}
\noindent Ο Πάγκος Εργασίας 1 συνιστά ένα κρίσιμο στοιχείο στην αυτοματοποιημένη γραμμή παραγωγής των καρεκλών \en{Gregor}. Διαθέτει τρεις θέσεις εργασίας, οι οποίες ενεργοποιούνται συγκεκριμένα ανάλογα με τη φάση του προϊόντος που βρίσκεται σε καθεμιά. Το αντίστοιχα διαγράμματα ροής φαίνονται στα Σχήματα \ref{fig:workbench1_flowchart} και \ref{fig:workbench1_position_flowchart}.

Η πρώτη θέση (\en{W1P1}) χρησιμοποιείται για την Εργασία Συναρμολόγησης 1 (\en{AT1}), όπου το Ρομπότ 1 συναρμολογεί τα πόδια με τη βάση της καρέκλας, δημιουργώντας το προϊόν στάδιο \en{S1}. Η δεύτερη θέση (\en{W1P2}) διαχειρίζεται την Εργασία Συναρμολόγησης 2 (\en{AT2}) και την Εργασία Συναρμολόγησης 5 (\en{AT5}), όπου το Ρομπότ 2 προσθέτει τα ροδάκια και τον μηχανισμό ανύψωσης (\en{S2}) και στη συνέχεια το κάθισμα στον σκελετό της καρέκλας, ολοκληρώνοντας το στάδιο \en{S5}. Η τρίτη θέση (\en{W1P3}) είναι προορισμένη για την Εργασία Συναρμολόγησης 6 (\en{AT6}), όπου το Ρομπότ 3 τοποθετεί την πλάτη της καρέκλας, φτάνοντας στο στάδιο \en{S6}.

Οι συνθήκες περιστροφής του Πάγκου Εργασίας 1 είναι καθοριστικές για την εκκίνηση των εργασιών. Μετά την πρώτη περιστροφή, η θέση \en{W1P1} καθίσταται ελεύθερη, η \en{W1P2} φιλοξενεί το προϊόν μετά το \en{AT1} (\en{S1}) και η \en{W1P3} είναι ελεύθερη, εκκινώντας την εργασία \en{AT2}. Για τις επόμενες περιστροφές, απαιτείται η \en{W1P1} να έχει προϊόν μετά το \en{AT1} (\en{S1}), η \en{W1P2} να φιλοξενεί προϊόν μετά το \en{AT5} (\en{S5}) και η \en{W1P3} να είναι ελεύθερη, εκκινώντας τις εργασίες \en{AT2} και \en{AT6}.


\begin{figure}[H]
    \centering
    \begin{subfigure}{0.49\textwidth}
        \centering
        \begin{tikzpicture}
            \node (W1P1) [circle, draw, minimum size=1cm] {\en{S1}};
            \node (W1P2) [circle, draw, minimum size=1cm, right of=W1P1, xshift=2cm] {\en{S5}};
            \node (W1P3) [circle, draw, minimum size=1cm, between=W1P1 and W1P2, yshift=2cm] {\en{Free}};
            \node (W1P1Label) [below right of=W1P1, xshift=0.3cm, yshift=0.4cm] {\small \en{W1P1}};
            \node (W1P2Label) [below right of=W1P2, xshift=0.3cm, yshift=0.4cm] {\small \en{W1P2}};
            \node (W1P3Label) [below right of=W1P3, xshift=0.3cm, yshift=0.4cm] {\small \en{W1P3}};

            \draw [arrow] (W1P1) -- (W1P2);
            \draw [arrow] (W1P2) -- (W1P3);
            \draw [arrow] (W1P3) -- (W1P1);
        \end{tikzpicture}
        \caption{Πριν την Περιστροφή}
        \label{fig:workbench1_before_rotation}
    \end{subfigure}
    \begin{subfigure}{0.49\textwidth}
        \centering
        \begin{tikzpicture}
            \node (W1P1) [circle, draw, minimum size=1cm] {\en{Free}};
            \node (W1P2) [circle, draw, minimum size=1cm, right of=W1P1, xshift=2cm] {\en{S1}};
            \node (W1P3) [circle, draw, minimum size=1cm, between=W1P1 and W1P2, yshift=2cm] {\en{S5}};
            \node (W1P1Label) [below right of=W1P1, xshift=0.3cm, yshift=0.4cm] {\small \en{W1P1}};
            \node (W1P2Label) [below right of=W1P2, xshift=0.3cm, yshift=0.4cm] {\small \en{W1P2}};
            \node (W1P3Label) [below right of=W1P3, xshift=0.3cm, yshift=0.4cm] {\small \en{W1P3}};

            \draw [arrow] (W1P1) -- (W1P2);
            \draw [arrow] (W1P2) -- (W1P3);
            \draw [arrow] (W1P3) -- (W1P1);
        \end{tikzpicture}
        \caption{Μετά την Περιστροφή}
        \label{fig:workbench1_after_rotation}
    \end{subfigure}
    \caption{Διάγραμμα Ροής Πάγκου Εργασίας 1}
    \label{fig:workbench1_flowchart}
\end{figure}

\begin{figure}[H]
    \centering
    \begin{subfigure}{0.32\textwidth}
        \centering
        \begin{tikzpicture}[node distance=1.6cm]
            \node (start) [start] {};
            \node (free) [process, below of=start] {\en{Free}};
            \node (at1) [process, below of=free] {\en{Assembly Task 1}};

            \draw [arrow] (start) -- (free);
            \draw [arrow] (free) -- (at1);
            \draw [arrow] (at1.east) -| ++(1,0) -- ($(at1.east)+(1,0)$) |- (free.east);
        \end{tikzpicture}
        \caption{Διάγραμμα Ροής Θέσης 1}
        \label{fig:workbench1_flowchart_position1}
    \end{subfigure}
    \begin{subfigure}{0.32\textwidth}
        \centering
        \begin{tikzpicture}[node distance=1.6cm]
            \node (start) [start] {};
            \node (s1) [process, below of=start] {\en{S1}};
            \node (at2) [process, below of=s1] {\en{Assembly Task 2}};
            \node (at5) [process, below of=at2] {\en{Assembly Task 5}};

            \draw [arrow] (start) -- (s1);
            \draw [arrow] (s1) -- (at2);
            \draw [arrow] (at2) -- (at5);
            \draw [arrow] (at5.east) -| ++(1,0) -- ($(at5.east)+(1,0)$) |- (s1.east);
        \end{tikzpicture}
        \caption{Διάγραμμα Ροής Θέσης 2}
        \label{fig:workbench1_flowchart_position2}
    \end{subfigure}
    \begin{subfigure}{0.32\textwidth}
        \centering
        \begin{tikzpicture}[node distance=1.6cm]
            \node (start) [start] {};
            \node (s5) [process, below of=start] {\en{S5}};
            \node (at6) [process, below of=s5] {\en{Assembly Task 6}};
            \node (at78) [process, below of=at6] {\en{AT7 and AT8}};
            \node (comp) [process, below of=at78] {\en{Complete}};

            \draw [arrow] (start) -- (s5);
            \draw [arrow] (s5) -- (at6);
            \draw [arrow] (at6) -- (at78);
            \draw [arrow] (at78) -- (comp);
            \draw [arrow] (comp.east) -| ++(1,0) -- ($(comp.east)+(1,0)$) |- (s5.east);
        \end{tikzpicture}
        \caption{Διάγραμμα Ροής Θέσης 3}
        \label{fig:workbench1_flowchart_position3}
    \end{subfigure}
    \caption{Διάγραμμα Ροής Θέσεων Πάγκου Εργασίας 1}
    \label{fig:workbench1_position_flowchart}
\end{figure}

\subsection{Πάγκος Εργασίας 2}
\noindent Ο Πάγκος Εργασίας 2 στην παραγωγική γραμμή των καρεκλών \en{Gregor} είναι ένας εξειδικευμένος πάγκος με μοναδική θέση εργασίας, \en{W2P1}, η οποία διαχειρίζεται δύο κρίσιμες εργασίες συναρμολόγησης. Το Ρομπότ 1 χρησιμοποιεί αυτή τη θέση για την Εργασία Συναρμολόγησης 4 (\en{AT4}), όπου τοποθετείται το κάθισμα στη βάση του καθίσματος. Ακολούθως, το Ρομπότ 2 εκτελεί την Εργασία Συναρμολόγησης 3 (\en{AT3}) στην ίδια θέση, όπου συνδέει την πλάκα καθίσματος με τη βάση. Το αντίστοιχο διάγραμμα ροής του Πάγκο Εργασίας 2 φαίνεται στο Σχήμα \ref{fig:workbench2_flowchart}.

Η ευθύνη του πάγκου εργασίας για τη διαχείριση του αιτήματος πρόσβασης από τα ρομπότ είναι καθοριστική, καθώς απαιτείται ο συντονισμός για την ορθή και έγκαιρη διαθεσιμότητα της θέσης για κάθε εργασία. Αυτός ο πάγκος εξασφαλίζει ότι οι συγκεκριμένες εργασίες εκτελούνται με ακρίβεια και στη σωστή σειρά, παρέχοντας μια αποτελεσματική βάση για την εκτέλεση εξειδικευμένων εργασιών που κρίνονται απαραίτητες για τη λειτουργικότητα και την ασφάλεια της τελικής καρέκλας.


\begin{figure}[H]
    \centering
    \begin{tikzpicture}[node distance=1.6cm]
        \node (start) [start] {};
        \node (free) [process, below of=start] {\en{Free}};
        \node (at4) [process, below of=free] {\en{Assembly Task 4}};
        \node (at3) [process, below of=at4] {\en{Assembly Task 3}};

        \draw [arrow] (start) -- (free);
        \draw [arrow] (free) -- (at4);
        \draw [arrow] (at4) -- (at3);
        \draw [arrow] (at3.east) -| ++(1,0) -- ($(at3.east)+(1,0)$) |- (free.east);
    \end{tikzpicture}
    \caption{Διάγραμμα Ροής Πάγκου Εργασίας 2}
    \label{fig:workbench2_flowchart}
\end{figure}

\section{Εργασίες Συναρμολόγησης}
\noindent Στην ενότητα αυτή, θα εξετάσουμε λεπτομερώς τις διαδικασίες συναρμολόγησης των καρεκλών γραφείου τύπου \en{Gregor}, που αποτελούν κρίσιμο στοιχείο της αυτοματοποιημένης παραγωγικής γραμμής. Κάθε διαδικασία συναρμολόγησης έχει σχεδιαστεί για να εξασφαλίζει ακρίβεια και αποδοτικότητα, διασφαλίζοντας ότι το τελικό προϊόν πληροί τις υψηλές προδιαγραφές λειτουργικότητας που απαιτούνται.

Η διαδικασία συναρμολόγησης διαχωρίζεται σε διάφορα στάδια, με κάθε στάδιο να εμπλέκει την εκτέλεση συγκεκριμένων εργασιών από διαφορετικά ρομπότ. Αυτές οι εργασίες στοχεύουν στη σταδιακή κατασκευή της καρέκλας, από τη βάση μέχρι το τελικό κάθισμα με την πλάτη και τα μπράτσα. Η σωστή οργάνωση και η ακριβής ακολουθία των εργασιών είναι κρίσιμες για την ομαλή λειτουργία της γραμμής παραγωγής.

\subsection{Εργασία Συναρμολόγησης 1}
\noindent Η Εργασία Συναρμολόγησης 1 (\en{AT1}) αποτελεί το θεμελιώδες πρώτο στάδιο στη συναρμολόγηση των καρεκλών \en{Gregor} και διεξάγεται από το Ρομπότ 1 στην πρώτη θέση του Πάγκου Εργασίας 1 (\en{W1P1}). Αυτή η διαδικασία αποτελείται από τέσσερα βήματα και έχει ως στόχο την αρχική συναρμολόγηση της βάσης της καρέκλας.

\begin{enumerate}
    \item \textbf{Βήμα 1:} Το Ρομπότ 1 αναλαμβάνει τα πόδια της καρέκλας από τον Χώρο Αποθήκευσης 1 και τα μεταφέρει προς τον πάγκο εργασίας.
    \item \textbf{Βήμα 2:} Εφόσον λάβει άδεια πρόσβασης στη θέση \en{W1P1}, το ρομπότ τοποθετεί τα πόδια στην καθορισμένη θέση.
    \item \textbf{Βήμα 3:} Στη συνέχεια, το Ρομπότ 1 επιστρέφει στον Χώρο Αποθήκευσης 2 για να προμηθευτεί την κεντρική βάση της καρέκλας και την τοποθετεί πάνω στα πόδια.
    \item \textbf{Βήμα 4:} Κατά την τελική τοποθέτηση της βάσης, το Ρομπότ 1 εξασφαλίζει ότι όλες οι συνδέσεις είναι σταθερές και ακριβείς. Με την ολοκλήρωση αυτού του βήματος, ενεργοποιείται αυτόματα η Εργασία Συναρμολόγησης 4 (\en{AT4}).
\end{enumerate}

Η \en{AT1} εκκινείται αρχικά από το σύστημα κατά την έναρξη της παραγωγής και στη συνέχεια από την Εργασία Συναρμολόγησης 4 (AT4) κάθε φορά που ολοκληρώνεται. Αυτό διασφαλίζει ότι η αρχική βάση της καρέκλας συναρμολογείται σωστά και με την κατάλληλη σειρά, καθιστώντας το πρώτο στάδιο ζωτικής σημασίας για την ορθή εκκίνηση της διαδικασίας παραγωγής.

\subsection{Εργασία Συναρμολόγησης 2}
\noindent Η Εργασία Συναρμολόγησης 2 (\en{AT2}) πραγματοποιείται από το Ρομπότ 2 και εκτελείται στη δεύτερη θέση του Πάγκου Εργασίας 1 (\en{W1P2}). Αυτή η διαδικασία είναι απαραίτητη για την προσθήκη των ροδάκιων και του μηχανισμού ανύψωσης στην ήδη συναρμολογημένη βάση της καρέκλας. Η \en{AT2} αποτελείται από τέσσερα βήματα:

\begin{enumerate}
    \item \textbf{Βήμα 1:} Ενεργοποιείται όταν ο Πάγκος Εργασίας 1 περιστραφεί, φέρνοντας στη θέση \en{W1P2} το αντικείμενο από την \en{AT1}. Το Ρομπότ 2 παίρνει τα ροδάκια από τον Χώρο Αποθήκευσης 4 και ζητά άδεια να εργαστεί στη θέση \en{W1P2}.
    \item \textbf{Βήμα 2:} Ενεργοποιείται μόλις το Ρομπότ 2 λάβει άδεια για τη θέση \en{W1P2}. Συνδυάζει τα ροδάκια με το αντικείμενο σε αυτή τη θέση και ενεργοποιεί το επόμενο βήμα.
    \item \textbf{Βήμα 3:} Συνεχίζεται με το Ρομπότ 2 να παίρνει τον μηχανισμό ανύψωσης από τον Χώρο Αποθήκευσης 5 και να ζητά ξανά άδεια για εργασία στη θέση \en{W1P2}.
    \item \textbf{Βήμα 4:} Αφού λάβει άδεια, το Ρομπότ 2 συνδυάζει τον μηχανισμό ανύψωσης με το αντικείμενο στη θέση \en{W1P2} και με την ολοκλήρωση αυτού του βήματος ενεργοποιεί την Εργασία Συναρμολόγησης 3 (AT3).
\end{enumerate}

Η \en{AT2} εκκινείται από την περιστροφή του Πάγκου Εργασίας 1 που φέρνει στη θέση \en{W1P2} το αντικείμενο μετά την \en{AT1}. Το Ρομπότ 2 διαδραματίζει κεντρικό ρόλο στην επικύρωση και ενίσχυση της βάσης της καρέκλας, προετοιμάζοντας τη για τις επόμενες φάσεις συναρμολόγησης.

\subsection{Εργασία Συναρμολόγησης 3}
\noindent Η Εργασία Συναρμολόγησης 3 (\en{AT3}) εκτελείται από το Ρομπότ 2 και πραγματοποιείται κυρίως στον Πάγκο Εργασίας 2 (\en{W2P1}), ακολουθώντας την ολοκλήρωση της Εργασίας Συναρμολόγησης 4 (\en{AT4}). Αυτή η διαδικασία είναι κρίσιμη για την τελική σύνδεση και στερέωση της πλάκας καθίσματος με την κύρια συναρμολόγηση της καρέκλας. Η \en{AT3} αποτελείται από τέσσερα βήματα:

\begin{enumerate}
    \item \textbf{Βήμα 1:} Ξεκινά μετά την ολοκλήρωση της \en{AT2}, όπου το Ρομπότ 2 κινείται προς τη θέση Β και ζητά άδεια για εργασία στον Πάγκο Εργασίας 2 (\en{W2P1}), όπου αναμένει το αντικείμενο μετά την \en{AT4}.
    \item \textbf{Βήμα 2:} Ακολουθεί η ενεργοποίηση μετά την ολοκλήρωση της \en{AT4} και τη χορήγηση άδειας από τον \en{W2P1}. Σε αυτό το σημείο, το Ρομπότ 2 συναρμολογεί την πλάκα καθίσματος με τη βάση, εξασφαλίζοντας τη στερέωση των δύο μερών.
    \item \textbf{Βήμα 3:} Μετά τη συναρμολόγηση, το Ρομπότ 2 ζητά άδεια για να αναλάβει το αντικείμενο από τον \en{W2P1} και κατευθύνεται προς τον Πάγκο Εργασίας 1 (\en{W1P2}).
    \item \textbf{Βήμα 4:} Εφόσον λάβει άδεια για τη θέση \en{W1P2}, το Ρομπότ 2 τοποθετεί το αντικείμενο σε αυτή τη θέση και με την ολοκλήρωση αυτού του βήματος ενεργοποιεί την Εργασία Συναρμολόγησης 5 (\en{AT5}).
\end{enumerate}

Η \en{AT3} είναι ουσιαστική για την τελική φάση της κατασκευής του καθίσματος και την ένωση των κύριων συστατικών της καρέκλας. Αυτή η εργασία διασφαλίζει ότι το κάθισμα είναι σταθερά συνδεδεμένο με τη βάση, ετοιμάζοντας το προϊόν για τις τελευταίες ρυθμίσεις και προσθήκες πριν την τελική επιθεώρηση.

\subsection{Εργασία Συναρμολόγησης 4}
\noindent Η Εργασία Συναρμολόγησης 4 (\en{AT4}) είναι μια κρίσιμη διαδικασία στην παραγωγή των καρεκλών \en{Gregor}, εκτελείται από το Ρομπότ 1 και διαδραματίζει ρόλο στην τοποθέτηση και στερέωση της πλάκας του καθίσματος στην καρέκλα. Η διαδικασία πραγματοποιείται στην θέση του Πάγκου Εργασίας 2 (\en{W2P1}) και περιλαμβάνει τέσσερα βήματα:

\begin{enumerate}
    \item \textbf{Βήμα 1:} Αρχίζει με την ενεργοποίηση του Ρομπότ 1 μετά την ολοκλήρωση της Εργασίας Συναρμολόγησης 1 (\en{AT1}). Το ρομπότ μετακινείται προς τη θέση Β, αντιμετωπίζει τη μεταφορική ταινία 1 και παίρνει το αντικείμενο (κάθισμα). Στη συνέχεια, ζητά άδεια για να εργαστεί στη θέση \en{W2P1}.
    \item \textbf{Βήμα 2:} Ενεργοποιείται μόλις το Ρομπότ 1 λάβει άδεια για τη θέση \en{W2P1}. Το ρομπότ τοποθετεί το κάθισμα στη θέση, ετοιμάζοντάς το για την προσθήκη της πλάκας του καθίσματος και ενεργοποιεί το επόμενο βήμα.
    \item \textbf{Βήμα 3:} Συνεχίζεται με το Ρομπότ 1 να αντιμετωπίζει τον Χώρο Αποθήκευσης 3, απ' όπου παίρνει την πλάκα του καθίσματος και ζητά άδεια για να επισυνάψει το εξάρτημα στην ίδια θέση (\en{W2P1}).
    \item \textbf{Βήμα 4:} Με τη λήψη άδειας, το Ρομπότ 1 συνδυάζει την πλάκα του καθίσματος με το κάθισμα στη θέση \en{W2P1}, συνδέοντας σταθερά τα δύο μέρη. Με την ολοκλήρωση της συναρμολόγησης, το ρομπότ μετακινείται πίσω στη θέση Α και ενεργοποιεί εκ νέου την Εργασία Συναρμολόγησης 1 (\en{AT1}), συνεχίζοντας την παραγωγική διαδικασία.
\end{enumerate}

Η \en{AT4} είναι ζωτικής σημασίας για τη στερέωση των κρίσιμων συστατικών του καθίσματος και διασφαλίζει τη συνέχιση της ομαλής και αποδοτικής παραγωγής της καρέκλας \en{Gregor}.

\subsection{Εργασία Συναρμολόγησης 5}
\noindent Η Εργασία Συναρμολόγησης 5 (\en{AT5}) αποτελεί ένα κρίσιμο στάδιο στην τελική συναρμολόγηση της καρέκλας \en{Gregor}, όπου το κύριο σώμα της καρέκλας συνδέεται με το κάθισμα. Τη διαδικασία αυτή αναλαμβάνει το Ρομπότ 2 και εκτελείται στη δεύτερη θέση του Πάγκου Εργασίας 1 (\en{W1P2}). Η διαδικασία \en{AT5} αποτελείται από δύο βήματα:

\begin{enumerate}
    \item \textbf{Βήμα 1:} Ενεργοποιείται αμέσως μετά την ολοκλήρωση της Εργασίας Συναρμολόγησης 3 (\en{AT3}). Το Ρομπότ 2 ζητά άδεια για να εργαστεί στη θέση \en{W1P2}, όπου αναμένει το κύριο σώμα της καρέκλας μαζί με το προσαρμοσμένο κάθισμα.
    \item \textbf{Βήμα 2:} Μόλις λάβει άδεια για τη θέση \en{W1P2}, το Ρομπότ 2 συνδυάζει τα δύο μέρη, χρησιμοποιώντας εξειδικευμένα εργαλεία για να σφίξει και να εξασφαλίσει τη στερέωσή τους. Αυτό το βήμα είναι κρίσιμο για τη σωστή λειτουργία της καρέκλας, καθώς η σταθερότητα της συνδέσεως επηρεάζει άμεσα την άνεση και την ασφάλεια του χρήστη.
\end{enumerate}

Η εκκίνηση της \en{AT5} γίνεται αυτόματα μετά την ολοκλήρωση της \en{AT3}, όπου το κάθισμα έχει ήδη προσαρμοστεί στην βάση του καθίσματος και τοποθετηθεί ενώπιον του Ρομπότ 2. Το Ρομπότ 2, με αυτόν τον τρόπο, ολοκληρώνει τη σύνδεση των κύριων μερών της καρέκλας, επιτυγχάνοντας μια σταθερή και λειτουργική δομή που είναι έτοιμη για τις τελικές προσθήκες, όπως η πλάτη και τα μπράτσα.

\subsection{Εργασία Συναρμολόγησης 6}
Η Εργασία Συναρμολόγησης 6 (\en{AT6}) πραγματοποιείται από το Ρομπότ 3 και είναι κρίσιμη για την τοποθέτηση της πλάτης στη συναρμολογημένη καρέκλα. Αυτή η διαδικασία λαμβάνει χώρα στην τρίτη θέση του Πάγκου Εργασίας 1 (\en{W1P3}) και αποτελείται από δύο βήματα:

\begin{enumerate}
    \item \textbf{Βήμα 1:} Ενεργοποιείται μετά από τη δεύτερη περιστροφή του Πάγκου Εργασίας 1, οπότε και η \en{W1P3} πρέπει να περιέχει το αντικείμενο μετά την Εργασία Συναρμολόγησης 5 (\en{AT5}). Το Ρομπότ 3 αντιμετωπίζει τη μεταφορική ταινία 2 και παίρνει την πλάτη της καρέκλας. Στη συνέχεια, ζητά άδεια για να εργαστεί στη θέση \en{W1P3}.
    \item \textbf{Βήμα 2:} Με τη λήψη άδειας, το Ρομπότ 3 τοποθετεί την πλάτη στο κύριο σώμα της καρέκλας και εξασφαλίζει τη στερέωσή της. Αυτό το βήμα απαιτεί ακριβή τοποθέτηση και στερέωση για να διασφαλιστεί η άνεση και η ασφάλεια του τελικού προϊόντος. Με την ολοκλήρωση αυτού του βήματος, το Ρομπότ 3 ενεργοποιεί ταυτόχρονα τις Εργασίες Συναρμολόγησης 7 και 8 (\en{AT7} και \en{AT8}), που αφορούν την τοποθέτηση των μπράτσων της καρέκλας.
\end{enumerate}

Η \en{AT6} είναι ζωτικής σημασίας για την τελική εμφάνιση και λειτουργικότητα της καρέκλας, καθώς η πλάτη παρέχει στήριξη και άνεση για τον χρήστη. Η εκκίνηση αυτής της διαδικασίας από την ενεργοποίηση του Πάγκου Εργασίας 1 διασφαλίζει την ορθή σειρά συναρμολόγησης και την ομαλή ενσωμάτωση της πλάτης στο συνολικό σχεδιασμό της καρέκλας.

\subsection{Εργασία Συναρμολόγησης 7}
\noindent Η Εργασία Συναρμολόγησης 7 (\en{AT7}) είναι μία από τις τελικές φάσεις στην παραγωγή της καρέκλας \en{Gregor}, όπου το Ρομπότ 3 είναι υπεύθυνο για την τοποθέτηση του αριστερού μπράτσου στην καρέκλα. Αυτή η διαδικασία εκτελείται στην τρίτη θέση του Πάγκου Εργασίας 1 (\en{W1P3}) και αποτελείται από δύο βήματα:

\begin{enumerate}
    \item \textbf{Βήμα 1:} Ενεργοποιείται ταυτόχρονα με την Εργασία Συναρμολόγησης 8 (\en{AT8}) αφού ολοκληρωθεί η Εργασία Συναρμολόγησης 6 (\en{AT6}). Το Ρομπότ 3 κινείται προς τον Χώρο Αποθήκευσης 6 και παίρνει το αριστερό μπράτσο της καρέκλας. Στη συνέχεια, ζητά άδεια για να εργαστεί στη θέση \en{W1P3}.
    \item \textbf{Βήμα 2:} Με τη λήψη άδειας, το Ρομπότ 3 τοποθετεί το αριστερό μπράτσο στην καρέκλα, ενώνοντάς το με την υπόλοιπη κατασκευή. Η στερέωση πρέπει να είναι ακριβής και σταθερή, για να εγγυηθεί την άνεση και τη λειτουργικότητα του τελικού προϊόντος.
\end{enumerate}

Η \en{AT7} είναι ζωτική για την προσθήκη των τελευταίων λειτουργικών στοιχείων που καθορίζουν την άνεση και την εργονομική αξία της καρέκλας. Η ενσωμάτωση του μπράτσου γίνεται με προσοχή για να διασφαλιστεί η διαρκής αντοχή και η ασφάλεια κατά τη χρήση.

\subsection{Εργασία Συναρμολόγησης 8}
\noindent Η Εργασία Συναρμολόγησης 8 (\en{AT8}) είναι ένα από τα τελικά στάδια στη διαδικασία συναρμολόγησης της καρέκλας \en{Gregor}, όπου το Ρομπότ 3 αναλαμβάνει την τοποθέτηση του δεξιού μπράτσου. Αυτή η διαδικασία πραγματοποιείται στην τρίτη θέση του Πάγκου Εργασίας 1 (\en{W1P3}) και ενεργοποιείται ταυτόχρονα με την Εργασία Συναρμολόγησης 7 (\en{AT7}) ακριβώς μετά την ολοκλήρωση της Εργασίας Συναρμολόγησης 6 (\en{AT6}). Η διαδικασία \en{AT8} αποτελείται από τα εξής βήματα:

\begin{enumerate}
    \item \textbf{Βήμα 1:} Το Ρομπότ 3 κινείται προς τον Χώρο Αποθήκευσης 6 και αναλαμβάνει το δεξί μπράτσο της καρέκλας. Εν συνεχεία, ζητά άδεια για εργασία στη θέση \en{W1P3}, όπου πρόκειται να προσαρμόσει το μπράτσο στο κύριο σώμα της καρέκλας.
    \item \textbf{Βήμα 2:} Αφού λάβει την απαραίτητη άδεια, το Ρομπότ 3 τοποθετεί το δεξί μπράτσο στην καρέκλα, ενώνοντάς το με το σκελετό. Η σύνδεση πρέπει να γίνει με ακρίβεια για να διασφαλιστεί η στερέωση και η σωστή λειτουργία του τελικού προϊόντος.
\end{enumerate}

Η ολοκλήρωση της \en{AT8} εγγυάται ότι η καρέκλα είναι πλήρως συναρμολογημένη με τα μπράτσα, παρέχοντας την απαραίτητη στήριξη και ολοκληρώνοντας την κατασκευή της. Με την τοποθέτηση του τελευταίου μπράτσου, η καρέκλα είναι έτοιμη για τελική χρήση, προσφέροντας την απαιτούμενη στήριξη και σταθερότητα.
\chapter{\en{Kubernetes}}\label{ch:kubernetes}

Το \en{Kubernetes} αποτελεί μία από τις πιο δημοφιλείς και ολοκληρωμένες πλατφόρμες ενορχήστρωσης (\en{orchestration}) και διαχείρισης εφαρμογών που εκτελούνται σε περιβάλλοντα \en{containers}, αλλά και σε συνδυασμό με \en{virtual machines}. Αναπτύχθηκε αρχικά από την \en{Google} και προσφέρθηκε ως έργο ανοιχτού κώδικα (\en{open-source}), με ισχυρή υποστήριξη από την κοινότητα. Σήμερα, έχει εξελιχθεί σε έναν από τους βασικότερους πυλώνες για την ανάπτυξη κατανεμημένων συστημάτων μεγάλης κλίμακας (\en{distributed systems}) και υποδομών \en{cloud}.

Στο πλαίσιο των προηγούμενων κεφαλαίων, όπου εξετάστηκαν το \en{IoT} (\en{Internet of Things}), τα \en{CPS} (\en{Cyber-Physical Systems}) και οι τεχνολογίες εικονοποίησης (\en{virtual machines} και \en{containers}), το παρόν κεφάλαιο αναδεικνύει τον ρόλο του \en{Kubernetes} ως ενοποιημένη πλατφόρμα υλοποίησης, ενορχήστρωσης και διαχείρισης αυτών των υποσυστημάτων. Αναλυτικότερα, θα παρουσιαστούν οι βασικές αρχές του \en{Kubernetes}, η αρχιτεκτονική του, οι δυνατότητες ασφαλούς και κλιμακούμενης (\en{scalable}) ανάπτυξης εφαρμογών, καθώς και η σύνδεσή του με τις έννοιες του \en{IoT}, των \en{CPS} και των \en{VMs}.

\section{Βασικές αρχές του \en{Kubernetes}}

Το \en{Kubernetes} ακολουθεί μια αρχιτεκτονική \en{client-server}, όπου ένα κεντρικό \en{control plane} αναλαμβάνει τον έλεγχο πολλών \en{worker nodes}. Στο \en{control plane} περιλαμβάνονται διάφορες κρίσιμες υπηρεσίες, όπως ο \en{API server}, ο \en{scheduler} και οι \en{controllers}, ενώ στους \en{worker nodes} εκτελούνται τα πραγματικά φορτία εργασίας (\en{workloads}), δηλαδή οι \en{containers} ή οι \en{virtual machines}.

% \selectlanguage{english}
% \begin{center}
% \begin{tikzpicture}[
%     font=\small,
%     node distance=1.5cm, % Adjust as needed
%     >=stealth',
%     every node/.style={draw, rectangle, rounded corners, align=center}
% ]
% % Control Panel
% \coordinate (clusterCenter) at (0,0);
% \node[fill=blue!10] (api) at (clusterCenter) {kube-apiserver}; 
% \node[fill=blue!10, above=of api,xshift=3cm]     (ccm)       {cloud-controller-manager};
% \node[fill=blue!10, above=of api,xshift=-3cm]     (kcm)       {kube-controller-manager};
% \node[fill=blue!10, left=of api]      (etcd)      {etcd};
% \node[fill=blue!10, right=of api]     (scheduler)        {scheduler};
% \node[
%     draw,
%     dashed,
%     rounded corners,
%     label={[yshift=0.1cm]above:\textbf{Control Plane}},
%     fit=(ccm)(api)(etcd)(kcm)(scheduler)
% ] (cpbox) {};
% \draw[<->] (api) -- (ccm);
% \draw[->]  (api) -- (etcd);
% \draw[->]  (scheduler) -- (api);
% \draw[->]  (kcm) -- (api);

% % Worker Node 1 (left)
% \node[fill=green!10, below=of clusterCenter, xshift=-1.3cm]                  (kubeproxy1) {kube-proxy};
% \node[fill=green!10, left=of kubeproxy1] (kubelet1) {kubelet};
% \node[fill=green!10, below=of kubeproxy1]                (pods1)      {Pods};

% \node[
%     draw,
%     dashed,
%     rounded corners,
%     label={[yshift=0.1cm]above:\textbf{Worker Node 1}},
%     fit=(kubelet1)(kubeproxy1)(pods1)
% ] (wn1box) {};

% % Worker Node 2 (right)
% \node[fill=green!10, below=of clusterCenter, xshift=1.3cm] (kubelet2) {kubelet};
% \node[fill=green!10, right=of kubelet2]                  (kubeproxy2) {kube-proxy};
% \node[fill=green!10, below=of kubeproxy2]                (pods2)      {Pods};

% \node[
%     draw,
%     dashed,
%     rounded corners,
%     label={[yshift=0.1cm]above:\textbf{Worker Node 2}},
%     fit=(kubelet2)(kubeproxy2)(pods2)
% ] (wn2box) {};
% \end{tikzpicture}
% \end{center}
% \begin{center}
% \begin{tikzpicture}[
%     font=\small,
%     node distance=1.5cm, % Adjust as needed
%     >=stealth',
%     every node/.style={draw, rectangle, rounded corners, align=center}
% ]

% % 1) Place "cloud-controller-manager" at the origin (0,0)
% \node[fill=blue!10] (api) at (0,0) {kube-apiserver};

% % 2) Use relative positioning for other Control Plane nodes
% \node[fill=blue!10, above=of api]     (ccm)       {cloud-controller-manager};
% \node[fill=blue!10, left=3.2cm of api]      (etcd)      {etcd};
% \node[fill=blue!10, right=3.2cm of api]     (cm)        {kube-controller-manager};
% \node[fill=blue!10, right=of ccm]       (scheduler) {scheduler};

% % 3) Dashed bounding box around Control Plane
% \node[
%     draw,
%     dashed,
%     rounded corners,
%     label={[yshift=0.2cm]above:\textbf{Control Plane}},
%     fit=(ccm)(api)(etcd)(cm)(scheduler)
% ] (cpbox) {};

% % 4) Reference coordinate at bottom center of Control Plane bounding box
% \coordinate (cpcenter) at (cpbox.south);

% % -------------------------------------------------------
% % Worker Node 1 (left)
% % -------------------------------------------------------
% \node[fill=green!10, below=2cm of cpcenter, xshift=-5cm] (kubelet1) {kubelet};
% \node[fill=green!10, right=of kubelet1]                  (kubeproxy1) {kube-proxy};
% \node[fill=green!10, below=of kubeproxy1]                (pods1)      {Pods};

% \node[
%     draw,
%     dashed,
%     rounded corners,
%     label={[yshift=0.2cm]above:\textbf{Worker Node 1}},
%     fit=(kubelet1)(kubeproxy1)(pods1)
% ] (wn1box) {};

% % -------------------------------------------------------
% % Worker Node 2 (right)
% % -------------------------------------------------------
% \node[fill=yellow!10, below=2cm of cpcenter, xshift=2cm] (kubelet2) {kubelet};
% \node[fill=yellow!10, right=of kubelet2]                 (kubeproxy2) {kube-proxy};
% \node[fill=yellow!10, below=of kubeproxy2]               (pods2)      {Pods};

% \node[
%     draw,
%     dashed,
%     rounded corners,
%     label={[yshift=0.2cm]above:\textbf{Worker Node 2}},
%     fit=(kubelet2)(kubeproxy2)(pods2)
% ] (wn2box) {};

% % -------------------------------------------------------
% % Example arrows
% % -------------------------------------------------------
% \draw[<->] (api) -- (ccm);
% \draw[->]  (api) -- (etcd);
% \draw[->]  (scheduler) -- (api);
% \draw[->]  (cm) -- (api);

% \draw[->]  (kubelet1) -- (api);
% \draw[<->] (kubelet2) -- (api);

% % -------------------------------------------------------
% % "Cluster" bounding box around EVERYTHING
% % -------------------------------------------------------
% \node[
%     draw,
%     thick,
%     rounded corners,
%     label={[yshift=0.4cm]above:\textbf{Cluster}},
%     fit=(cpbox)(wn1box)(wn2box)
% ] (clusterbox) {};

% \end{tikzpicture}
% \end{center}
% \begin{center}
% \begin{tikzpicture}[
%     font=\small,
%     node distance=1cm,
%     >=stealth',
%     every node/.style={draw, rectangle, rounded corners, align=center}
% ]

% % -------------------------------------------------------
% % Control Plane components
% % -------------------------------------------------------
% \node[fill=blue!10] (ccm) {cloud-controller-manager};
% \node[fill=blue!10, below=of ccm] (api) {kube-apiserver};
% \node[fill=blue!10, left=of api] (etcd) {etcd};
% \node[fill=blue!10, below=of api] (cm) {kube-controller-manager};
% \node[fill=blue!10, left=of cm] (scheduler) {scheduler};

% % Dashed bounding box around Control Plane
% \node[
%     draw,
%     dashed,
%     rounded corners,
%     label={[yshift=0.2cm]above:\textbf{Control Plane}},
%     fit=(api)(scheduler)(cm)(etcd)(ccm)
% ] (cpbox) {};

% % -------------------------------------------------------
% % Reference coordinate at bottom center of Control Plane
% % -------------------------------------------------------
% \coordinate (cpcenter) at (cpbox.south);

% % -------------------------------------------------------
% % Worker Node 1 (left)
% % -------------------------------------------------------
% \node[fill=green!10, below=2cm of cpcenter, xshift=-3cm] (kubelet1) {kubelet};
% \node[fill=green!10, right=of kubelet1] (kubeproxy1) {kube-proxy};
% \node[fill=green!10, below=of kubeproxy1] (pods1) {Pods};

% \node[
%     draw,
%     dashed,
%     rounded corners,
%     label={[yshift=0.2cm]above:\textbf{Worker Node 1}},
%     fit=(kubelet1)(kubeproxy1)(pods1)
% ] (wn1box) {};

% % -------------------------------------------------------
% % Worker Node 2 (right)
% % -------------------------------------------------------
% \node[fill=yellow!10, below=2cm of cpcenter, xshift=3cm] (kubelet2) {kubelet};
% \node[fill=yellow!10, right=of kubelet2] (kubeproxy2) {kube-proxy};
% \node[fill=yellow!10, below=of kubeproxy2] (pods2) {Pods};

% \node[
%     draw,
%     dashed,
%     rounded corners,
%     label={[yshift=0.2cm]above:\textbf{Worker Node 2}},
%     fit=(kubelet2)(kubeproxy2)(pods2)
% ] (wn2box) {};

% % % API Server
% \draw[<->] (api) -- (ccm);
% \draw[->] (api) -- (etcd);

% % Scheduler
% \draw[->] (scheduler) -- (api);

% % Controller Manager
% \draw[->] (cm) -- (api);

% % Communication to kube-apiserver
% \draw[->] (kubelet1) -- (api);
% \draw[<->] (kubelet2) -- (api);

% % -------------------------------------------------------
% % "Cluster" bounding box around EVERYTHING
% % -------------------------------------------------------
% \node[
%     draw,
%     thick,
%     rounded corners,
%     label={[yshift=0.4cm]above:\textbf{Cluster}},
%     fit=(cpbox)(wn1box)(wn2box)
% ] (clusterbox) {};


% \end{tikzpicture}
% \end{center}
% \begin{tikzpicture}[
%     font=\small,
%     node distance=1.3cm,
%     auto,
%     >=stealth',
%     every node/.style={draw, rectangle, rounded corners, align=center}
% ]

% % -------------------------------------------------------
% % Control Plane components
% % -------------------------------------------------------
% \node[fill=blue!10] (ccm) {cloud-controller-manager};
% \node[fill=blue!10, below=of ccm] (api) {kube-apiserver};
% \node[fill=blue!10, left=of api] (etcd) {etcd};
% \node[fill=blue!10, below=of api] (cm) {kube-controller-manager};
% \node[fill=blue!10, left=of cm] (scheduler) {scheduler};

% % Dashed bounding box around Control Plane
% \node[
%     draw,
%     dashed,
%     rounded corners,
%     label={[yshift=0.2cm]above:\textbf{Control Plane}},
%     fit=(api)(scheduler)(cm)(etcd)(ccm)
% ] (cpbox) {};

% % -------------------------------------------------------
% % Worker Node 1
% % -------------------------------------------------------
% \node[fill=green!10, below=1.5cm of cm] (kubelet1) {kubelet};
% \node[fill=green!10, right=0.5cm of kubelet1] (kubeproxy1) {kube-proxy};
% \node[fill=green!10, below=of kubeproxy1] (pods1) {Pods};

% \node[
%     draw,
%     dashed,
%     rounded corners,
%     label={[yshift=0.2cm]above:\textbf{Worker Node 1}},
%     fit=(kubelet1)(kubeproxy1)(pods1)
% ] (wn1box) {};

% % -------------------------------------------------------
% % Worker Node 2
% % -------------------------------------------------------
% \node[fill=yellow!10, right=2.5cm of kubelet1] (kubelet2) {kubelet};
% \node[fill=yellow!10, right=of kubelet2] (kubeproxy2) {kube-proxy};
% \node[fill=yellow!10, below=of kubeproxy2] (pods2) {Pods};

% \node[
%     draw,
%     dashed,
%     rounded corners,
%     label={[yshift=0.2cm]above:\textbf{Worker Node 2}},
%     fit=(kubelet2)(kubeproxy2)(pods2)
% ] (wn2box) {};

% \node[
%     draw,
%     dashed,
%     rounded corners,
%     label={[yshift=0.2cm]above:\textbf{Cluster}},
%     fit=(cpbox)(wn1box)(wn2box)
% ] (cluster) {};
% % -------------------------------------------------------
% % Draw arrows to represent communication
% % -------------------------------------------------------
% % API Server
% \draw[<->] (api) -- (ccm);
% \draw[->] (api) -- (etcd);

% % Scheduler
% \draw[->] (scheduler) -- (api);

% % Controller Manager
% \draw[->] (cm) -- (api);

% % Communication to kube-apiserver
% \draw[->] (kubelet1) -- (api);
% \draw[<->] (kubelet2) -- (api);

% \end{tikzpicture}
\selectlanguage{greek}

\begin{itemize}
    \item \textbf{\en{API server}}: Παρέχει το κεντρικό σημείο επικοινωνίας μέσω \en{REST} \en{API}, μέσω του οποίου οι χρήστες, τα \en{controllers} και ο \en{scheduler} αλληλεπιδρούν με το \en{Kubernetes} \en{cluster}.
    \item \textbf{\en{etcd}}: Αποθηκεύει την κατάσταση (\en{state}) του \en{cluster}, π.χ. τις ρυθμίσεις (\en{configurations}) και τις προδιαγραφές των εφαρμογών.
    \item \textbf{\en{scheduler}}: Αποφασίζει σε ποιον \en{worker node} θα εκτελεστούν οι \en{pods}, βάσει διαθεσιμότητας πόρων (\en{CPU}, μνήμη) και διαφόρων πολιτικών.
    \item \textbf{\en{kube-controller-manager}}: Ελέγχουν και συντονίζουν τη «στοχευμένη» κατάσταση (\en{desired state}) με την «τρέχουσα» κατάσταση (\en{current state}). Για παράδειγμα, ένας \en{controller} εξασφαλίζει πως αν δηλωθούν 5 αντίγραφα (\en{replicas}) μιας εφαρμογής, θα εκτελούνται πάντοτε 5 \en{pods}.
\end{itemize}

Από την άλλη πλευρά, κάθε \en{worker node} φιλοξενεί:
\begin{itemize}
    \item Τον \textbf{\en{kubelet}}, ο οποίος επικοινωνεί με τον \en{API server} και αναλαμβάνει την υλοποίηση των εντολών στο τοπικό σύστημα.
    \item Τον \textbf{\en{kube-proxy}}, υπεύθυνο για τους κανόνες δικτύωσης και την προώθηση (\en{forwarding}) της κίνησης στα σωστά \en{pods}.
    \item Τον \textbf{\en{container runtime}} (\en{Docker}, \en{containerd} ή άλλο) για τη δημιουργία και εκτέλεση \en{containers}.
\end{itemize}

\section{\en{Pods}, \en{Services} και λοιποί πυρήνες μηχανισμοί}

Στο \en{Kubernetes}, η μικρότερη μονάδα ανάπτυξης εφαρμογών είναι το \en{pod}. Ένα \en{pod} μπορεί να περιέχει έναν ή περισσότερους \en{containers}, οι οποίοι μοιράζονται το ίδιο \en{namespace} δικτύου και δίσκου (\en{volumes}). Πάνω από τα \en{pods}, ορίζονται έννοιες υψηλότερου επιπέδου, όπως το \en{Deployment} (εξασφαλίζει κλιμάκωση και ενημερώσεις χωρίς διακοπή) και το \en{StatefulSet} (για εφαρμογές που διατηρούν κατάσταση). 

Για τη δικτύωση, το \en{Kubernetes} χρησιμοποιεί τους \en{Services}, οι οποίοι προσφέρουν ένα σταθερό σημείο πρόσβασης (\en{cluster IP} ή \en{load balancer IP}), ανεξάρτητα από το πού εκτελούνται τα \en{pods} φυσικά. Επίσης, υπάρχουν μηχανισμοί όπως τα \en{Ingress controllers} για την παροχή \en{HTTPS} τερματισμού (\en{TLS termination}) και δρομολόγησης (\en{routing}) σε υπηρεσίες εφαρμογών.

\subsection{Αποθήκευση (\en{Storage})}

Για την επίμονη αποθήκευση δεδομένων (\en{persistent data}), το \en{Kubernetes} ορίζει τα \en{Persistent Volumes (PV)} και τα \en{Persistent Volume Claims (PVC)}. Οι εφαρμογές (\en{pods}) ζητούν αποθηκευτικό χώρο δημιουργώντας ένα \en{PVC}, ενώ ο διαχειριστής (ή ένας μηχανισμός \en{dynamic provisioning}) αντιστοιχίζει ένα \en{PV} στο αίτημα αυτό. Έτσι, οι εφαρμογές μπορούν να κρατούν δεδομένα ανεξαρτήτως του \en{node} όπου εκτελούνται.

\subsection{Ασφάλεια (\en{Security})}

Το \en{Kubernetes} διαθέτει πολλαπλά επίπεδα ασφάλειας:
\begin{itemize}
    \item \textbf{\en{Role-Based Access Control (RBAC)}}: Καθορίζει ποιοι χρήστες και ποιες υπηρεσίες έχουν δικαίωμα να εκτελούν συγκεκριμένες ενέργειες στο \en{API}.
    \item \textbf{\en{Network Policies}}: Ορίζουν κανόνες για τη δικτυακή ροή (\en{ingress}, \en{egress}) σε επίπεδο \en{pod}, εμποδίζοντας τη μη εξουσιοδοτημένη επικοινωνία.
    \item \textbf{\en{Pod Security}}: Επιτρέπει ή απαγορεύει τη χρήση ευαίσθητων προνομίων (\en{privileged}, \en{root}) ή την πρόσβαση σε κρίσιμα τμήματα του λειτουργικού συστήματος.
\end{itemize}

\section{\en{Kubernetes} και ενοποίηση με \en{containers}, \en{VMs}, \en{IoT} και \en{CPS}}

\subsection{\en{Containers} και \en{Kubernetes}}

Η πιο διαδεδομένη χρήση του \en{Kubernetes} αφορά την ενορχήστρωση εφαρμογών που εκτελούνται σε \en{containers}. Ανάμεσα στα δημοφιλέστερα \en{container runtimes} συναντάμε το \en{Docker} και το \en{containerd}. Το \en{Kubernetes} διευκολύνει:
\begin{itemize}
    \item \textbf{Αυτόματη κλιμάκωση} (\en{autoscaling}): Αύξηση ή μείωση του αριθμού \en{pods} βάσει μετρικών (\en{metrics}), όπως η χρήση \en{CPU} ή οι αιτήσεις \en{HTTP}.
    \item \textbf{Αναβαθμίσεις χωρίς διακοπή} (\en{rolling updates}): Σταδιακή αντικατάσταση των \en{pods} με νέες εκδόσεις, διασφαλίζοντας τη συνεχή παροχή υπηρεσίας.
    \item \textbf{Ανθεκτικότητα} (\en{resilience}): Σε περίπτωση σφάλματος κάποιου \en{pod} ή ολόκληρου \en{node}, το \en{Kubernetes} επανεκκινεί αυτόματα τα \en{pods} σε υγιείς κόμβους.
\end{itemize}

\subsection{\en{Virtual Machines} και \en{KubeVirt}}

Παρά την έντονη έμφαση στους \en{containers}, κάποιες εφαρμογές ή περιπτώσεις χρήσης απαιτούν \en{virtual machines}. Το \en{KubeVirt} είναι ένα έργο ανοιχτού κώδικα που επιτρέπει την εκτέλεση \en{VMs} μέσα σε \en{Kubernetes}. Αυτό επιτυγχάνεται με την εισαγωγή \en{CRDs} (\en{Custom Resource Definitions}), μέσω των οποίων ορίζεται ένα \en{VirtualMachine} αντικείμενο στο \en{Kubernetes} \en{API}. 

Με αυτή τη μέθοδο:
\begin{itemize}
    \item \textbf{Υπάρχουσες \en{VM}-βασισμένες εφαρμογές} μπορούν να ενταχθούν στο \en{Kubernetes} οικοσύστημα, αξιοποιώντας τα οφέλη του (\en{autoscaling}, \en{monitoring}, \en{logging}).
    \item \textbf{Μεταβατικά σενάρια} μεταξύ \en{VMs} και \en{containers} γίνονται εφικτά, χωρίς να αλλάξει δραστικά η αρχιτεκτονική των συστημάτων.
    \item \textbf{Κοινή διαχείριση} πολιτικών, όπως η ασφάλεια και η δικτύωση, ενοποιείται για \en{pods} και \en{VMs}.
\end{itemize}

\subsection{\en{IoT} και \en{Edge Computing}}

Στα περιβάλλοντα \en{IoT}, η κατανεμημένη επεξεργασία δεδομένων (\en{edge computing}) αποκτά ιδιαίτερη σημασία, καθώς μειώνει τους χρόνους απόκρισης (\en{latency}) και το κόστος δικτυακής μετάδοσης. Εργαλεία όπως το \en{k3s} και το \en{MicroK8s} επιτρέπουν την εγκατάσταση ενός ελαφρού \en{Kubernetes} \en{distribution} σε \en{edge devices} με περιορισμένους πόρους. Κατ’ αυτόν τον τρόπο, οι \en{IoT} συσκευές (\en{sensors}, \en{actuators}) μπορούν να επικοινωνούν με τοπικούς κόμβους, οι οποίοι τρέχουν συλλογισμούς (\en{analytics}, \en{filtering}) σε \en{containers} εντός ενός μικρού \en{Kubernetes} \en{cluster}.

Παράλληλα, τα δεδομένα που προκύπτουν από την τοπική επεξεργασία μπορούν να προωθηθούν στο κεντρικό \en{cloud} \en{Kubernetes} \en{cluster}, το οποίο αναλαμβάνει περαιτέρω ανάλυση (\en{machine learning}, \en{big data analytics}) και αποθήκευση (\en{data warehousing}). Αυτή η λογική \en{edge-cloud} επιτρέπει στα \en{IoT} συστήματα να επεκταθούν πιο εύκολα, να αυτοματοποιηθούν και να κλιμακώσουν τις υπηρεσίες τους χωρίς να επιβαρύνουν τους κεντρικούς πόρους με μη επεξεργασμένα δεδομένα.

\subsection{\en{Cyber-Physical Systems (CPS)}}

Τα \en{CPS} χαρακτηρίζονται από την αλληλεπίδραση μεταξύ φυσικών διεργασιών και ψηφιακών υποδομών. Το \en{Kubernetes} βοηθάει σημαντικά στην αρχιτεκτονική των \en{CPS}, ιδιαίτερα όταν χρησιμοποιούνται επιμέρους υπηρεσίες (\en{microservices}) για τη συλλογή και επεξεργασία σημάτων από αισθητήρες ή τον έλεγχο ενεργοποιητών (\en{actuators}).

Σε ορισμένες περιπτώσεις, τα \en{CPS} έχουν απαιτήσεις χαμηλής καθυστέρησης (\en{low-latency}) ή πραγματικού χρόνου (\en{real-time}). Παρότι το \en{Kubernetes} δεν σχεδιάστηκε για \en{hard real-time} εφαρμογές, υπάρχουν τροποποιημένοι πυρήνες \en{Linux} (\en{RT kernels}) και ειδικές ρυθμίσεις \en{scheduler} που μπορούν να βελτιώσουν την απόδοση. Έτσι, για συστήματα \en{soft real-time} ή ημι-αυτόνομες βιομηχανικές ροές, το \en{Kubernetes} μπορεί να προσφέρει μια ευέλικτη λύση ενορχήστρωσης και διαχείρισης.

\section{Κλιμακούμενη ανάπτυξη και βέλτιστες πρακτικές}

\subsection{\en{Autoscaling} μηχανισμοί}

Το \en{Kubernetes} υποστηρίζει διάφορες στρατηγικές αυτόματης κλιμάκωσης (\en{autoscaling}):
\begin{itemize}
    \item \textbf{\en{Horizontal Pod Autoscaler (HPA)}}: Αυξομειώνει τον αριθμό των \en{pods} ενός \en{Deployment} ή \en{ReplicaSet} βάσει μετρικών (\en{CPU} / \en{memory} χρήση ή \en{custom metrics}).
    \item \textbf{\en{Cluster Autoscaler}}: Αυτόματη προσθήκη ή αφαίρεση \en{nodes} σε ένα \en{cloud} περιβάλλον, αναλόγως με τις ανάγκες των \en{pods}.
\end{itemize}
Σε περιπτώσεις \en{IoT} ή \en{CPS}, όπου η ροή δεδομένων μπορεί να παρουσιάζει μεγάλες διακυμάνσεις, οι μηχανισμοί αυτοί εξασφαλίζουν την ομαλή ανταπόκριση του συστήματος.

\subsection{\en{Rolling Updates} και \en{Canary Deployments}}

Για τη συνεχή παράδοση (\en{continuous delivery}) αναβαθμίσεων χωρίς να διακόπτεται η λειτουργία, το \en{Kubernetes} χρησιμοποιεί \en{rolling updates}, αντικαθιστώντας σταδιακά τα παλιά \en{pods} με νέα. Επιπλέον, τεχνικές \en{canary deployment} επιτρέπουν τη δοκιμή νέων εκδόσεων σε μικρό ποσοστό χρηστών, προτού γενικευθούν σε όλο το σύστημα. Αυτό είναι ιδιαίτερα χρήσιμο όταν ενσωματώνονται αλγόριθμοι \en{machine learning} ή νέες λειτουργίες σε ευαίσθητα συστήματα \en{CPS}.

\subsection{Παρακολούθηση (\en{Monitoring}) και καταγραφή (\en{Logging})}

Η διαχείριση σύνθετων \en{Kubernetes} \en{clusters} απαιτεί κατάλληλα εργαλεία παρακολούθησης (\en{monitoring}) και καταγραφής (\en{logging}). Συνήθως, χρησιμοποιούνται λύσεις όπως το \en{Prometheus} για τη συλλογή μετρικών, σε συνδυασμό με τον \en{Grafana} για την οπτικοποίηση. Για την ενιαία καταγραφή (\en{logging}), δημοφιλή είναι τα \en{EFK} (\en{Elasticsearch}, \en{Fluentd}, \en{Kibana}) ή το \en{Loki} της \en{Grafana Labs}. Με αυτόν τον τρόπο, οι διαχειριστές εντοπίζουν γρήγορα σφάλματα ή δυσλειτουργίες, τόσο σε εφαρμογές \en{containers} όσο και σε \en{VMs} (μέσω \en{KubeVirt}).

% \section{Παραδείγματα χρήσης}

% \subsection{\en{Smart Factory} σε βιομηχανικό περιβάλλον \en{Industry 4.0}}

% Μια εργοστασιακή μονάδα παραγωγής (\en{smart factory}) μπορεί να υλοποιήσει συλλογή δεδομένων και αυτοματισμούς (αισθητήρες \en{IoT}, ρομπότ \en{CPS}) σε \en{edge devices}, τα οποία τρέχουν ελαφριές διανομές \en{Kubernetes}. Τοπικά \en{containers} επεξεργάζονται τα δεδομένα (π.χ. συνάγουν τάσεις, εντοπίζουν σφάλματα) και αποστέλλουν προ-επεξεργασμένα αποτελέσματα προς ένα κεντρικό \en{Kubernetes} \en{cluster} στο \en{cloud}. Το κεντρικό σύστημα χρησιμοποιεί \en{machine learning} για προβλέψεις συντήρησης (\en{predictive maintenance}), ενώ παράλληλα ενδέχεται να τρέχει «παλαιότερες» κρίσιμες εφαρμογές (\en{legacy SCADA}) σε \en{VMs} μέσω \en{KubeVirt}.

% \subsection{\en{Healthcare} και \en{wearable IoT}}

% Σε νοσοκομεία ή στην κατ’ οίκον φροντίδα ασθενών, \en{wearable} συσκευές (\en{IoT}) συλλέγουν ζωτικές ενδείξεις (πίεση, καρδιακοί παλμοί, θερμοκρασία). Τα δεδομένα μεταφέρονται σε \en{edge gateways} (\en{k3s}) που εκτελούν βασικές αναλύσεις για την άμεση ειδοποίηση ιατρικού προσωπικού σε περίπτωση ανωμαλιών. Η περαιτέρω επεξεργασία γίνεται σε ένα κεντρικό \en{Kubernetes} \en{cluster}, όπου τρέχουν υπηρεσίες \en{machine learning} (σε \en{containers}) και πιθανόν ειδικά \en{VMs} για εφαρμογές ηλεκτρονικού φακέλου υγείας (\en{EHR systems}) μέσω \en{KubeVirt}. Έτσι συνδυάζεται η ελαφρότητα και ταχύτητα του \en{edge} με την υπολογιστική ισχύ του \en{cloud}.

% \section{Συμπεράσματα}

% Το \en{Kubernetes} έχει αναδειχθεί σε ένα από τα πιο ευρέως χρησιμοποιούμενα εργαλεία για την ανάπτυξη, ενορχήστρωση και διαχείριση εφαρμογών μεγάλης κλίμακας. Η ομαλή ενσωμάτωση με τις τεχνολογίες \en{containers}, η δυνατότητα χρήσης \en{virtual machines} μέσω του \en{KubeVirt}, καθώς και η συμβατότητα με περιβάλλοντα \en{IoT} και \en{CPS}, καθιστούν το \en{Kubernetes} μια ευέλικτη και ισχυρή πλατφόρμα.

% Σε ό,τι αφορά τις σύγχρονες απαιτήσεις για υλοποίηση \en{smart} συστημάτων, είτε πρόκειται για βιομηχανική παραγωγή (\en{Industry 4.0}), συστήματα υγείας (\en{Healthcare}) ή έξυπνες πόλεις (\en{Smart Cities}), το \en{Kubernetes} προσφέρει τους απαραίτητους μηχανισμούς για αυτοματοποιημένη κλιμάκωση, ανθεκτικότητα σε σφάλματα και ευέλικτη ανάπτυξη. Παράλληλα, τα εργαλεία ασφαλείας (\en{RBAC}, \en{Network Policies}, \en{Pod Security}) μπορούν να θωρακίσουν ευαίσθητες εφαρμογές από επιθέσεις ή κακόβουλη χρήση.

% Συνολικά, η επιλογή του \en{Kubernetes} ως κεντρικής πλατφόρμας ενορχήστρωσης επιτρέπει την ενοποίηση ποικίλων υποσυστημάτων: από \en{containers} και \en{VMs} μέχρι σύνθετα \en{IoT}/\en{CPS} περιβάλλοντα. Με τη συνεχή εξέλιξη του οικοσυστήματος (π.χ. το \en{KubeVirt}, \en{k3s}, \en{operators}), οι δυνατότητες διαρκώς επεκτείνονται, διευκολύνοντας τη μετάβαση σε μια δυναμική, αυτοματοποιημένη υποδομή, ικανή να ανταποκριθεί στις υψηλές απαιτήσεις της σύγχρονης πληροφορικής και βιομηχανίας.
\pagestyle{plain}
\begin{center}
  {\LARGE \textbf{Περίληψη}}\\[1cm]
  \textbf{\doctitle}\\[1cm]
  \begin{minipage}{0.38\textwidth}
    \begin{flushleft}
      \textbf{Ονοματεπώνυμο Φοιτητή}\\
      \textbf{\nommesmall}
    \end{flushleft}
  \end{minipage}
  \begin{minipage}{0.38\textwidth}
    \begin{flushright}
      \textbf{Ονοματεπώνυμο Επιβλέποντος}\\
      \textbf{\supname}
    \end{flushright}
  \end{minipage}
\end{center}

\vspace{10mm}

Η παρούσα διπλωματική εργασία εξετάζει την αξιοποίηση υπηρεσιών ιδιωτικού \en{cloud} στο πλαίσιο του
Διαδικτύου των Αντικειμένων (\en{IoT}). Ειδικότερα, η εργασία επικεντρώνεται στη χρήση του \en{Kubernetes}
ως υποδομή ιδιωτικού \en{cloud} για την ανάπτυξη και διαχείριση μικροϋπηρεσιών (\en{microservices}) καθώς
και πώς η ιδιωτική υποδομή αυτή θα μπορούσε να υλοποιηθεί σε ενα περιβαλλόν εργοστασίου. Στην συνέχεια
σχεδιάστηκαν και υλοποιήθηκαν ένας αριθμός από μικρουπηρεσίες (\en{microservices}) με στόχο τον έλεγχο και την παρακολούθηση
μιας γραμμής παραγωγής, επιδεικνύοντας την αποτελεσματικότητα και την ευελιξία που παρέχουν οι τεχνολογίες \en{cloud} σε
βιομηχανικές εφαρμογές.

Η καινοτομία του συστήματος έγκειται στη χρήση κυβερνοφυσικών \en{microservices}, οι οποίες αλληλεπιδρούν με ρομποτικούς βραχίονες
μέσω \en{HTTP requests}. Κάθε \en{microservice} αντιπροσωπεύει ένα συγκεκριμένο βήμα παραγωγής, επιτυγχάνοντας ουσιαστικό διαχωρισμό \en{(decoupling)}
ανάμεσα στη λογική των βημάτων και στους ρομποτικούς βραχίονες. Οι βραχίονες λειτουργούν ως απομακρυσμένα
\en{API endpoints} που εκτελούν την κίνηση ή τη διεργασία που ζητούν τα \en{microservices}, χωρίς να χρειάζεται
να γνωρίζουν οι ίδιοι τη σειρά ή τον τρόπο υλοποίησης των επιμέρους σταδίων. Επιπλέον, πραγματοποιήθηκε ενορχήστρωση
των υπηρεσιών \en{(service orchestration)} για τον συγχρονισμό και το συντονισμό των διαφορετικών βημάτων παραγωγής,
διασφαλίζοντας την ομαλή και ευέλικτη ροή της διαδικασίας.

\clearpage
\selectlanguage{english}
\begin{center}
  {\LARGE \textbf{Extensive English Summary}}\\[1cm]
  \textbf{\engdoctitle}\\[1cm]
  \begin{minipage}{0.38\textwidth}
    \begin{flushleft}
      \textbf{Student name, surname}\\
      \textbf{\nommesmallenglish}
    \end{flushleft}
  \end{minipage}
  \begin{minipage}{0.38\textwidth}
    \begin{flushright}
      \textbf{Supervisor name, surname}\\
      \textbf{\supnameegnlish}
    \end{flushright}
  \end{minipage}
\end{center}
\vspace{10mm}

This thesis examines the utilization of private cloud services in the context of the Internet of Things (IoT).
Specifically, the study focuses on the use of Kubernetes as a private cloud infrastructure for the development
and management of microservices, as well as how this private infrastructure could be implemented within
a factory environment. Subsequently, a number of microservices were designed and implemented with the aim of controlling
and monitoring a production line, demonstrating the effectiveness and flexibility offered by cloud technologies in
industrial applications.

The innovation of the system lies in the use of cyber-physical microservices, which interact with robotic arms
via HTTP requests. Each microservice represents a specific production step, achieving substantial decoupling
between the logic of the steps and the robotic arms. The robotic arms function as remote API endpoints that execute the
movement or process requested by the microservices, without needing to know the sequence or the implementation details
of the individual stages themselves. In addition, service orchestration was implemented to
synchronize and coordinate the various production steps, ensuring a smooth and flexible workflow.

\selectlanguage{greek}
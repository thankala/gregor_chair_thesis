%!TEX root = ../../main.tex

\chapter{Διαδίκτυο των Αντικειμένων και Κυβερνοφυσικά Συστήματα}\label{ch:iot}

Το Διαδίκτυο των Αντικειμένων (\en{Internet of Things} ή \en{IoT}) αποτελεί έναν από τους σημαντικότερους παράγοντες εξέλιξης της σύγχρονης τεχνολογίας και πληροφορικής. Παράλληλα, η έννοια των Κυβερνοφυσικών Συστημάτων (\en{Cyber-Physical Systems} ή \en{CPS}) έχει έρθει στο προσκήνιο τα τελευταία χρόνια, καθώς συνδυάζει συσκευές του φυσικού κόσμου με την υπολογιστική ισχύ και τις δυνατότητες δικτύωσης που χαρακτηρίζουν την ψηφιακή εποχή. Σε αυτό το κεφάλαιο παρουσιάζονται οι βασικές έννοιες, οι αρχές λειτουργίας και τα πλεονεκτήματα που προκύπτουν από την ενσωμάτωση τόσο του \en{IoT} όσο και των \en{CPS} σε διάφορους τομείς.

\section{Εισαγωγή στο Διαδίκτυο των Αντικειμένων}

Το \en{Internet of Things (IoT)} αναφέρεται σε ένα οικοσύστημα όπου φυσικές συσκευές,
αισθητήρες και άλλα αντικείμενα συνδέονται και ανταλλάσσουν δεδομένα μέσω του Διαδικτύου
\cite{gubbi_internet_2013}. Η ιδέα αυτή θέτει τις βάσεις για την έξυπνη διασύνδεση και
αλληλεπίδραση μεταξύ ποικίλων συστημάτων, από βιομηχανικές εφαρμογές έως έξυπνα σπίτια,
προκειμένου να δημιουργηθούν νέες υπηρεσίες και βελτιστοποιημένες διαδικασίες
\cite{weber_internet_2010}.

Σύμφωνα με εκτιμήσεις, ο αριθμός των συνδεδεμένων συσκευών στο Διαδίκτυο αυξήθηκε
ραγδαία την τελευταία δεκαετία. Ειδικότερα, υπολογίζεται πως το 2010 υπήρχαν περίπου
12.5 δισεκατομμύρια συνδεδεμένες συσκευές παγκοσμίως, ενώ προβλεπόταν να φτάσουν τα
50 δισεκατομμύρια έως το 2020 \cite{Evans2011}. Αυτή η τάση αναμένεται να συνεχιστεί,
καθώς τεχνολογίες όπως το \en{Cloud Computing} και η \en{Artificial Intelligence} γίνονται
όλο και πιο προσιτές, συμβάλλοντας στην ευρύτερη υιοθέτηση του \en{IoT} σε διάφορους τομείς.

\subsection{Η εξέλιξη του \en{Web} (από \en{Web 1.0} έως \en{Web 4.0})}

Ιστορικά, η εξέλιξη του Παγκόσμιου Ιστού (\en{World Wide Web}) χωρίζεται σε διάφορες φάσεις:
\en{Web 1.0}, \en{Web 2.0}, \en{Web 3.0} και \en{Web 4.0}. Στο \en{Web 1.0}, το περιεχόμενο
ήταν κυρίως στατικό, επιτρέποντας μονόδρομη αλληλεπίδραση (παροχή πληροφορίας από
διακομιστές σε χρήστες). Η μετάβαση στο \en{Web 2.0} εστίασε στην αλληλεπίδραση και τη
δημιουργία περιεχομένου από τους ίδιους τους χρήστες, οδηγώντας στην ανάπτυξη κοινωνικών
δικτύων και πλατφορμών ανταλλαγής περιεχομένου.

Η επόμενη γενιά, το \en{Web 3.0}, χαρακτηρίζεται από τη «σημασιολογική» διάσταση, όπου οι
μηχανές και οι εφαρμογές μπορούν να «κατανοήσουν» καλύτερα τα δεδομένα και τις έννοιες,
επιτρέποντας πιο ευφυείς υπηρεσίες. Τέλος, το \en{Web 4.0} περιγράφεται συχνά ως
«υπεύφυες» πλατφόρμες, οι οποίες συνδέουν ανθρώπους και συσκευές σε πραγματικό χρόνο
και ανταλλάσσουν πολυδιάστατα δεδομένα, γεγονός που συνδέεται άμεσα με την ανάπτυξη του
\en{IoT}, της \en{Cloud Computing} τεχνολογίας και προηγμένων αλγορίθμων \en{Artificial Intelligence}.

\subsection{Βασικά επίπεδα (\en{Layers}) του \en{IoT} κατά \en{Atzori}}

Η αρχιτεκτονική του \en{IoT} μπορεί να περιγραφεί μέσα από πολλά μοντέλα (\en{layers}),
προκειμένου να γίνει κατανοητή η ροή και η επεξεργασία των δεδομένων. Σύμφωνα με τους
\en{Atzori et al.} \cite{atzori_internet_2010}, το \en{IoT} μπορεί να αναλυθεί σε πέντε επίπεδα, ως εξής:

\begin{enumerate}
  \item \textbf{Επίπεδο Αντικειμένων (\en{Objects})}: Περιλαμβάνει όλους τους αισθητήρες,
    ενεργοποιητές και γενικότερα τις φυσικές \en{IoT} συσκευές που συλλέγουν δεδομένα από το
    περιβάλλον ή επιδρούν σε αυτό.

  \item \textbf{Επίπεδο Αφαίρεσης Αντικειμένων (\en{Object Abstraction})}: Αφορά τις τεχνικές
    και τους μηχανισμούς που επιτρέπουν την αφαίρεση (\en{abstraction}) των δεδομένων, ώστε
    να μπορούν να διακινηθούν και να υποστούν επεξεργασία ανεξάρτητα από το φυσικό αντικείμενο
    ή τα επιμέρους πρωτόκολλα επικοινωνίας.

  \item \textbf{Επίπεδο Διαχείρισης Υπηρεσιών (\en{Service Management})}: Περιλαμβάνει τις
    λειτουργίες που σχετίζονται με την προσφορά, ανακάλυψη, διαχείριση και εκτέλεση υπηρεσιών
    πάνω στα δεδομένα που προέρχονται από τα αντικείμενα.

  \item \textbf{Επίπεδο Σύνθεσης Υπηρεσιών (\en{Service Composition})}: Σε αυτό το επίπεδο,
    οι επιμέρους υπηρεσίες συνδυάζονται ή «συντίθενται» για να δημιουργήσουν πιο πολύπλοκες
    και ολοκληρωμένες εφαρμογές \en{IoT}, προσφέροντας προηγμένες λειτουργίες στους χρήστες.

  \item \textbf{Επίπεδο Εφαρμογών (\en{Applications})}: Αφορά το «τελικό» στάδιο, όπου οι
    εφαρμογές (\en{smart home}, \en{smart city}, \en{industrial IoT} κ.λπ.) αξιοποιούν τα δεδομένα
    και τις υπηρεσίες, παρέχοντας ουσιαστικές λύσεις και λειτουργικότητα στους τελικούς χρήστες.
\end{enumerate}

Άλλοι ερευνητές προτείνουν ένα πιο συνοπτικό μοντέλο τριών επιπέδων (\en{Perception, Network,
Application}) \cite{gubbi_internet_2013}, ενώ ορισμένες εταιρείες, όπως η \en{Cisco}, περιγράφουν
αναλυτικά μοντέλα πέντε ή έξι επιπέδων με έμφαση στη διαχείριση δικτύων \en{edge}, στον
διαχωρισμό των πυλών (\en{gateways}) και στη σημασιολογική ανάλυση \cite{Cisco2014}. Σε κάθε
περίπτωση, η ασφάλεια και η διαχείριση μεγάλων ποσοτήτων δεδομένων (\en{Big Data}) αποτελούν
κοινές προκλήσεις που διατρέχουν όλα τα επίπεδα του \en{IoT}. Καθώς ο αριθμός των
συνδεδεμένων συσκευών συνεχίζει να αυξάνεται, η ανάγκη για αξιόπιστα, ευέλικτα και ασφαλή
συστήματα θα γίνεται ολοένα και πιο επιτακτική.

\subsection{Εφαρμογές και παραδείγματα}

Οι εφαρμογές του \en{IoT} καλύπτουν ένα ευρύ φάσμα:
\begin{itemize}
  \item \textbf{Έξυπνα σπίτια}: Η αυτόματη ρύθμιση φωτισμού, θέρμανσης/κλιματισμού και η διαχείριση οικιακών συσκευών επιτρέπουν βελτιωμένη άνεση και εξοικονόμηση ενέργειας.
  \item \textbf{Έξυπνες πόλεις}: Αισθητήρες ενσωματωμένοι στο αστικό περιβάλλον συλλέγουν δεδομένα για την κυκλοφορία, τη ρύπανση ή τη διαχείριση απορριμμάτων, βελτιώνοντας τις παρεχόμενες υπηρεσίες και την ποιότητα ζωής.
  \item \textbf{Βιομηχανική παραγωγή}: Σε βιομηχανικά περιβάλλοντα, το \en{IoT} αξιοποιείται για την παρακολούθηση μηχανημάτων, την προληπτική συντήρηση (\en{predictive maintenance}) και τη βελτίωση της αποδοτικότητας των γραμμών παραγωγής.
  \item \textbf{Υγεία}: Φορετές συσκευές (\en{wearables}) και αισθητήρες παρακολουθούν ζωτικές ενδείξεις ασθενών και ενημερώνουν σε πραγματικό χρόνο ιατρικό προσωπικό για πιθανές επιπλοκές.
\end{itemize}

\section{Κυβερνοφυσικά Συστήματα (\en{CPS})}

Τα κυβερνοφυσικά συστήματα (\en{CPS}) αποτελούν μία ολοένα και πιο σημαντική κατηγορία
συστημάτων, τα οποία ενοποιούν αλληλεπιδράσεις μεταξύ του φυσικού και του ψηφιακού
κόσμου \cite{Lee2008, Rajkumar2010, Baheti2011}. Σε γενικές γραμμές, ένα \en{CPS}
περιλαμβάνει υπολογιστικούς πόρους (π.χ. ενσωματωμένα συστήματα, δικτυακές υποδομές)
και φυσικές διεργασίες (π.χ. βιομηχανικός εξοπλισμός, αισθητήρες, ρομπότ), τα οποία
επικοινωνούν και λειτουργούν με στενή σύζευξη. Στόχος αυτής της ενσωμάτωσης είναι η
επίτευξη υψηλών επιπέδων ευφυΐας, αυτονομίας και πραγματικού χρόνου ανταπόκρισης
σε σύνθετα περιβάλλοντα.

Σύμφωνα με την ανάλυση του \en{Lee} \cite{Lee2008}, τα κυβερνοφυσικά συστήματα συναντώνται
σε ένα ευρύ φάσμα εφαρμογών, όπως η βιομηχανική αυτοματοποίηση, τα συστήματα μεταφορών,
τα ενεργειακά δίκτυα (\en{smart grid}), καθώς και η ιατρική παρακολούθηση και υποστήριξη.
Βασικό χαρακτηριστικό είναι η συνεχής ανατροφοδότηση (\en{feedback}) μεταξύ φυσικών
και ψηφιακών οντοτήτων, έτσι ώστε η απόκριση του συστήματος να είναι δυναμική και
εξαρτώμενη από τις εκάστοτε συνθήκες.

Γνωρίζουμε πως τα \en{CPS} αποτελούν την επόμενη
«υπολογιστική επανάσταση»,\cite{Rajkumar2010}\cite{Rajkumar2010} αφού οι απαιτήσεις σε θέματα συγχρονισμού, αξιοπιστίας και
ασφάλειας είναι ιδιαιτέρως αυξημένες συγκριτικά με τα παραδοσιακά \en{IT} ή ακόμα και τα
\en{IoT} συστήματα. Ειδικότερα, η στενή σύζευξη μεταξύ λογισμικού και φυσικών διεργασιών
επιβάλλει την ανάπτυξη νέων μεθοδολογιών σχεδίασης, οι οποίες λαμβάνουν υπόψη περιορισμούς
όπως ο πραγματικός χρόνος (\en{real time}), η ενεργειακή αποδοτικότητα, αλλά και η
προβλεψιμότητα της συμπεριφοράς του συστήματος.

Παράλληλα, η ολοένα αυξανόμενη
δημοτικότητα των \en{CPS} οφείλεται και στις τεχνολογικές εξελίξεις στους τομείς του
\en{cloud computing} \cite{Baheti2011}, της ανάλυσης μεγάλων δεδομένων (\en{big data}), καθώς και των
ασύρματων επικοινωνιών. Με αυτόν τον τρόπο, τα κυβερνοφυσικά συστήματα μπορούν να
υλοποιήσουν πολύπλοκες λειτουργίες, όπως η προληπτική συντήρηση (\en{predictive maintenance})
σε βιομηχανικά περιβάλλοντα.

\subsection{Βασικά Χαρακτηριστικά και Προκλήσεις}

\begin{itemize}
  \item \textbf{Πραγματικός χρόνος (\en{real-time}):} Η αλληλεπίδραση με τον φυσικό κόσμο
    απαιτεί χρονικά περιορισμένες αποκρίσεις, ώστε το σύστημα να αντιδρά με ακρίβεια σε
    μεταβαλλόμενες συνθήκες.
  \item \textbf{Ασφάλεια (\en{security}):} Καθώς τα \en{CPS} συνδέονται ολοένα και περισσότερο
    σε δίκτυα, η κυβερνοασφάλεια αποτελεί κρίσιμο ζήτημα. Οι επιθέσεις μπορούν να έχουν
    σοβαρές επιπτώσεις στον φυσικό κόσμο, όπως διακοπές σε βιομηχανικά συστήματα ή
    συστήματα υγειονομικής περίθαλψης.
  \item \textbf{Αξιοπιστία (\en{reliability}):} Η αδιάλειπτη λειτουργία του συστήματος
    είναι ουσιώδης, ιδίως σε εφαρμογές όπου η δυσλειτουργία ενέχει σημαντικούς κινδύνους
    (π.χ. στα έξυπνα δίκτυα ενέργειας ή στην αυτόνομη οδήγηση).
  \item \textbf{Επεκτασιμότητα (\en{scalability}):} Τα \en{CPS} καλούνται να διαχειριστούν
    μεγάλους όγκους δεδομένων και μεγάλο αριθμό συσκευών, διατηρώντας παράλληλα υψηλά
    επίπεδα απόδοσης.
\end{itemize}

Λαμβάνοντας υπόψη τις ανωτέρω παραμέτρους, τα κυβερνοφυσικά συστήματα αναμένεται να
αποτελέσουν την επόμενη γενιά εφαρμογών σε πληθώρα τομέων, συνδυάζοντας ισχυρούς
υπολογιστικούς πόρους, συνεχείς ροές δεδομένων και έξυπνα φυσικά στοιχεία. Η έρευνα
επικεντρώνεται, μεταξύ άλλων, σε μεθόδους ολοκλήρωσης (\en{integration}) μεταξύ λογισμικού
και υλισμικού, σε τεχνικές ανθεκτικότητας (\en{fault tolerance}) και σε προσεγγίσεις που
εξασφαλίζουν την ασφάλεια και την αξιοπιστία σε όλες τις φάσεις λειτουργίας του συστήματος.

Τα Κυβερνοφυσικά Συστήματα (\en{Cyber-Physical Systems}) αποτελούν έναν ευρύτερο όρο που περιγράφει την ενοποίηση της υπολογιστικής ισχύος (κυβερνοχώρος) με φυσικές διεργασίες. Σε ένα \en{CPS}, τα φυσικά αντικείμενα και οι λειτουργίες τους ελέγχονται στενά από αλγορίθμους που εκτελούνται σε υπολογιστικά συστήματα, με στόχο την υλοποίηση ενός ολοκληρωμένου συστήματος ελέγχου και αυτοματισμού.

\subsection{Εφαρμογές και παραδείγματα \en{CPS}}

\begin{itemize}
  \item \textbf{Έξυπνα εργοστάσια}: Ρομποτικά συστήματα και αισθητήρες ελέγχουν κάθε στάδιο παραγωγής, ενώ οι αποφάσεις για τη γραμμή παραγωγής λαμβάνονται δυναμικά, με βάση την κατάσταση των μηχανημάτων και την παραγωγική ζήτηση.
  \item \textbf{Αυτόνομα οχήματα}: Ένας συνδυασμός αισθητήρων (π.χ. κάμερες, \en{LiDAR}, \en{RADAR}) και αλγορίθμων τεχνητής νοημοσύνης (\en{AI}) επιτρέπει στα οχήματα να αλληλεπιδρούν με το περιβάλλον και να λαμβάνουν αποφάσεις σε πραγματικό χρόνο. Το σύστημα ελέγχου (κυβερνοχώρος) συνδέεται διαρκώς με τις φυσικές κινήσεις του οχήματος.
  \item \textbf{Έξυπνα δίκτυα ενέργειας}: Τα δίκτυα ηλεκτρικής ενέργειας γίνονται ολοένα πιο «έξυπνα», χάρη σε αισθητήρες και συστήματα ελέγχου που παρακολουθούν σε πραγματικό χρόνο την κατανάλωση ενέργειας. Οι ανανεώσιμες πηγές ενέργειας ενσωματώνονται δυναμικά, ενώ κεντρικά συστήματα και αλγόριθμοι βελτιστοποίησης ρυθμίζουν τη λειτουργία του δικτύου.
  \item \textbf{Υγειονομικά συστήματα}: Συνδυάζοντας φορητές συσκευές παρακολούθησης (π.χ. \en{wearable sensors}), βάσεις ιατρικών δεδομένων και ρομποτικές συσκευές, επιτυγχάνεται μια συνεχής αλληλεπίδραση μεταξύ θεραπευτικών πρακτικών και πραγματικού χρόνου παρακολούθησης της κατάστασης του ασθενή.
\end{itemize}

\section{Σημαντικά ζητήματα και προκλήσεις}

\subsection{Ασφάλεια και προστασία ιδιωτικότητας}

Τόσο τα συστήματα \en{IoT} όσο και τα \en{CPS} εκτίθενται σε πλήθος κινδύνων ασφαλείας. Ο μεγάλος αριθμός διασυνδεδεμένων συσκευών αυξάνει την πιθανότητα επίθεσης (\en{attack surface}), ενώ ενδεχόμενη πρόσβαση σε ευαίσθητα δεδομένα μπορεί να έχει σημαντικές επιπτώσεις στην ιδιωτικότητα των χρηστών. Επιπλέον, σε ένα \en{CPS}, επιθέσεις στον κυβερνοχώρο μπορούν να προκαλέσουν φυσικές ζημιές, καθώς η ψηφιακή διάσταση ελέγχει τη φυσική.

Για αυτόν τον λόγο, απαιτούνται πρωτόκολλα κρυπτογράφησης, συστήματα ελέγχου πρόσβασης και μηχανισμοί συνεχούς παρακολούθησης της δικτυακής κίνησης. Η πρόκληση έγκειται στη διατήρηση της ισορροπίας ανάμεσα στην υψηλή ασφάλεια και την ομαλή λειτουργία/επέκταση του δικτύου.

\subsection{Διαλειτουργικότητα και πρότυπα}

Η ραγδαία αύξηση των συσκευών και η ύπαρξη πολλών κατασκευαστών δημιουργούν την ανάγκη για πρότυπα επικοινωνίας, έτσι ώστε ετερογενή συστήματα να μπορούν να συνεργάζονται. Ορισμένα γνωστά πρωτόκολλα \en{IoT} περιλαμβάνουν το \en{MQTT} (\en{Message Queuing Telemetry Transport}), το \en{CoAP} (\en{Constrained Application Protocol}) και το \en{HTTP} (\en{Hypertext Transfer Protocol}).

Ωστόσο, σε περιπτώσεις \en{CPS}, οι απαιτήσεις για χρονική ακρίβεια και αξιοπιστία είναι ακόμη πιο αυστηρές, οδηγώντας σε ανάγκη για εξειδικευμένα πρωτόκολλα πραγματικού χρόνου (\en{real-time protocols}). Παράλληλα, η ευρεία αποδοχή συγκεκριμένων προτύπων θα επιταχύνει την ανάπτυξη λύσεων και θα διευκολύνει την ενσωμάτωση νέων τεχνολογιών.

\subsection{Διαχείριση μεγάλου όγκου δεδομένων}

Καθώς οι αισθητήρες και οι συσκευές \en{IoT}/\en{CPS} παράγουν τεράστιο όγκο δεδομένων, η επεξεργασία και η αποθήκευση τους αποτελεί πρόκληση. Οι παραδοσιακές βάσεις δεδομένων συχνά δεν επαρκούν, οπότε η αξιοποίηση τεχνολογιών \en{Big Data}, \en{cloud computing} και \en{edge/fog computing} θεωρείται απαραίτητη. Ειδικά στα \en{CPS}, η έγκαιρη επεξεργασία των δεδομένων είναι κρίσιμη, επειδή οι αποφάσεις επηρεάζουν το φυσικό περιβάλλον σε πραγματικό χρόνο και τυχόν καθυστέρηση μπορεί να επιφέρει σφάλματα ή κινδύνους για την ασφάλεια.

% \section{Σύνοψη και μελλοντικές τάσεις}

% Το Διαδίκτυο των Αντικειμένων (\en{IoT}) και τα Κυβερνοφυσικά Συστήματα (\en{CPS}) βρίσκονται στο επίκεντρο της επερχόμενης ψηφιακής μετάβασης, επηρεάζοντας πολλούς τομείς: βιομηχανία, υγεία, ενέργεια, μεταφορές κ.ά. Η συνεισφορά τους δεν περιορίζεται μόνο στην παροχή καινοτόμων υπηρεσιών, αλλά επεκτείνεται στη διαμόρφωση νέων επιχειρηματικών μοντέλων και στην επαναπροσδιόριση των υφιστάμενων.

% Οι σημαντικές προκλήσεις που παραμένουν αφορούν την ασφάλεια, την ιδιωτικότητα, την αποδοτική διαχείριση των δεδομένων και τη διαλειτουργικότητα των συσκευών. Στο μέλλον, η ενσωμάτωση της τεχνητής νοημοσύνης (\en{AI}) και η περαιτέρω αξιοποίηση τεχνολογιών \en{edge} υπολογιστικής αναμένεται να διευρύνουν τις δυνατότητες των συστημάτων \en{IoT} και \en{CPS}, παρέχοντας ολοένα και πιο προηγμένες εφαρμογές, ικανές να προσαρμόζονται δυναμικά στις ανάγκες του περιβάλλοντος.

% Τέλος, η αλληλεπίδραση ανθρώπου-μηχανής (\en{Human-Machine Interaction}) θα διαμορφωθεί εκ νέου, δίνοντας έμφαση στον έξυπνο αυτοματισμό, στην προνοητική λήψη αποφάσεων και σε συστήματα που βασίζονται σε ανατροφοδότηση πραγματικού χρόνου. Καθώς η τεχνολογία προχωρά, το \en{IoT} και τα \en{CPS} θα συγχωνεύονται ολοένα και περισσότερο, προσφέροντας μια συνεχή, «ζωντανή» δικτύωση ανάμεσα σε ανθρώπους, μηχανές και φυσικό περιβάλλον.
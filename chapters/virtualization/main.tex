%!TEX root = ../../main.tex

\chapter{Ιδιωτικές Υποδομές}\label{ch:virtualization}

Η εικονοποίηση (\en{virtualization}) αποτελεί μια θεμελιώδη τεχνολογία στον χώρο της πληροφορικής, επιτρέποντας τη δημιουργία πολλαπλών περιβαλλόντων (λειτουργικών συστημάτων ή εφαρμογών) πάνω σε μια κοινή φυσική υποδομή. Με αυτόν τον τρόπο, αξιοποιείται καλύτερα η διαθέσιμη υπολογιστική ισχύς, ενώ ταυτόχρονα παρέχονται δυνατότητες απομόνωσης, φορητότητας και ευελιξίας. Τα τελευταία χρόνια, ιδιαίτερη έμφαση δίνεται στις Εικονικές Μηχανές (\en{Virtual Machines}) και στα \en{containers}, δύο εναλλακτικές προσεγγίσεις εικονοποίησης που επιλύουν διαφορετικά προβλήματα και εξυπηρετούν διαφορετικές ανάγκες. Σε αυτό το κεφάλαιο, θα εξετάσουμε τις βασικές αρχές της εικονοποίησης, θα επικεντρωθούμε στα πλεονεκτήματα και τις προκλήσεις που σχετίζονται με τις \en{virtual machines} και τα \en{containers} και θα παρουσιάσουμε χαρακτηριστικές χρήσεις τους σε σύγχρονα περιβάλλοντα υπολογιστών και υποδομών \en{cloud}.

\section{Εισαγωγή στην εικονοποίηση}

Η εικονοποίηση (\en{virtualization}) είναι η διαδικασία διαχωρισμού των φυσικών πόρων (π.χ. \en{CPU}, μνήμη, αποθηκευτικός χώρος, δικτυακές διεπαφές) από τις λογικές λειτουργικές οντότητες που τους χρησιμοποιούν. Μέσα από ένα ενδιάμεσο επίπεδο λογισμικού, που συχνά ονομάζεται \en{hypervisor}, δημιουργούνται αυτόνομα, εικονικά περιβάλλοντα, τα οποία «νομίζουν» ότι έχουν πρόσβαση σε όλη την υποδομή. Ωστόσο, στην πραγματικότητα, το σύστημα ελέγχει και κατανέμει τους πόρους ανάλογα με τις ανάγκες και τις πολιτικές διαχείρισης.

\subsection{Σύντομη ιστορική αναδρομή}

Η πρώτη μορφή εικονοποίησης εμφανίστηκε στη δεκαετία του 1960 σε υπολογιστές \en{mainframe} της \en{IBM}, όπου οι χρήστες μοιράζονταν υπολογιστική ισχύ μέσω ξεχωριστών εικονικών περιβαλλόντων. Με την πάροδο του χρόνου και τη ραγδαία αύξηση της υπολογιστικής ικανότητας των \en{x86} αρχιτεκτονικών, η εικονοποίηση βγήκε από το πλαίσιο των κεντρικών υπολογιστών (\en{mainframes}) και έγινε προσιτή στα \en{servers} ευρείας χρήσης. Από τότε, έχουν αναπτυχθεί διαφορετικά μοντέλα και τεχνολογίες εικονοποίησης, με πιο δημοφιλείς υλοποιήσεις την εικονοποίηση επιπέδου λειτουργικού συστήματος (\en{OS-level virtualization}), τις \en{virtual machines} και, πιο πρόσφατα, τα \en{containers}.

\subsection{Βασικές τεχνικές}

\begin{itemize}
  \item \textbf{Εικονοποίηση υλικού (\en{hardware-level virtualization})}: Χρησιμοποιεί έναν \en{hypervisor} (π.χ. \en{VMware ESXi}, \en{KVM}, \en{Hyper-V}) ο οποίος αναλαμβάνει τη δρομολόγηση των εντολών μεταξύ του φυσικού επεξεργαστή και των εικονικών μηχανών.
  \item \textbf{Εικονοποίηση επιπέδου λειτουργικού συστήματος (\en{OS-level virtualization})}: Χρησιμοποιεί τις δυνατότητες του πυρήνα του λειτουργικού συστήματος για να δημιουργήσει πολλαπλά λογικά περιβάλλοντα, όπως συμβαίνει με τα \en{containers}.
  \item \textbf{Παραλλαγές \en{paravirtualization}}: Εδώ το λειτουργικό σύστημα μέσα στην εικονική μηχανή (\en{guest OS}) είναι τροποποιημένο, ώστε να υποστηρίζει αποτελεσματικότερα τις λειτουργίες του \en{hypervisor}, όπως συμβαίνει στο \en{Xen}.
\end{itemize}

\section{Εικονικές Μηχανές (\en{Virtual Machines})}

Οι εικονικές μηχανές (\en{virtual machines}) είναι μια παραδοσιακή μέθοδος εικονοποίησης, όπου ένας \en{hypervisor} προσφέρει ένα ολοκληρωμένο εικονικό \en{hardware} περιβάλλον σε κάθε \en{guest} λειτουργικό σύστημα. Αυτό σημαίνει ότι το \en{guest OS} πιστεύει πως «τρέχει» σε έναν πραγματικό υπολογιστή, με τη δική του \en{CPU}, τη δική του μνήμη, τους δίσκους του και άλλα περιφερειακά.

\subsection{Δομή και λειτουργία}

Σε ένα τυπικό σενάριο, ο \en{hypervisor} εγκαθίσταται απευθείας πάνω στο φυσικό υλικό (μοντέλο \en{bare-metal}). Παραδείγματα τέτοιων \en{hypervisors} είναι το \en{VMware ESXi}, το \en{Microsoft Hyper-V} και το \en{Xen}. Ο \en{hypervisor} διαχειρίζεται τους φυσικούς πόρους και τους κατανέμει στα διάφορα \en{guest OS}, τα οποία δεν έχουν άμεση πρόσβαση στο υλικό.

Εναλλακτικά, μπορεί κανείς να χρησιμοποιήσει έναν \en{hypervisor} που εγκαθίσταται πάνω από ένα υπάρχον λειτουργικό σύστημα (μοντέλο \en{hosted}), όπως το \en{VirtualBox} ή το \en{VMware Workstation}. Σε αυτή την περίπτωση, το \en{host} λειτουργικό σύστημα διαμεσολαβεί για την πρόσβαση στο υλικό, ενώ ο \en{hypervisor} λειτουργεί ως μια εφαρμογή που προσφέρει τις υπηρεσίες εικονοποίησης.

\subsection{Πλεονεκτήματα και μειονεκτήματα}

\paragraph{Πλεονεκτήματα}
\begin{itemize}
  \item \textbf{Απομόνωση}: Κάθε \en{virtual machine} λειτουργεί εντελώς ανεξάρτητα από τις υπόλοιπες. Τυχόν σφάλματα ή επιθέσεις σε ένα \en{guest} δεν επηρεάζουν τους άλλους.
  \item \textbf{Ευελιξία}: Μπορούν να εγκατασταθούν διαφορετικά λειτουργικά συστήματα (\en{Windows}, \en{Linux}, \en{BSD} κλπ.) στον ίδιο φυσικό \en{server}.
  \item \textbf{Φορητότητα}: Οι εικονικές μηχανές μπορούν εύκολα να μεταφερθούν ή να αντιγραφούν από έναν \en{host} σε άλλο, επιτρέποντας απρόσκοπτη μετεγκατάσταση και ανάκτηση από βλάβες (\en{disaster recovery}).
\end{itemize}

\paragraph{Μειονεκτήματα}
\begin{itemize}
  \item \textbf{Υψηλότερη κατανάλωση πόρων}: Εφόσον κάθε \en{guest OS} «κουβαλά» το δικό του πυρήνα και ολοκληρωμένες βιβλιοθήκες, η κατανάλωση μνήμης και επεξεργαστικής ισχύος είναι σημαντική.
  \item \textbf{Χρόνοι εκκίνησης}: Η εκκίνηση μιας \en{virtual machine} απαιτεί τη φόρτωση πλήρους λειτουργικού συστήματος, γεγονός που αυξάνει τους χρόνους εκκίνησης σε σχέση με άλλα μοντέλα εικονοποίησης.
\end{itemize}

\subsection{Χρήσεις και παραδείγματα}

\begin{itemize}
  \item \textbf{Παροχή υπηρεσιών \en{cloud}}: Πλατφόρμες όπως το \en{Amazon EC2}, το \en{Microsoft Azure} και το \en{Google Cloud} ξεκίνησαν προσφέροντας \en{virtual machines}, δίνοντας τη δυνατότητα στους πελάτες να «ενοικιάζουν» υπολογιστική ισχύ, εγκαθιστώντας ελεύθερα ό,τι λογισμικό επιθυμούν.
  \item \textbf{Απομόνωση εφαρμογών υψηλής ασφαλείας}: Συχνά, κρίσιμες εφαρμογές ή περιβάλλοντα δοκιμών (\en{testing environments}) φιλοξενούνται σε ξεχωριστές \en{virtual machines} για να εξασφαλίζεται η ασφάλεια των δεδομένων.
  \item \textbf{Περιβάλλοντα ανάπτυξης}: Οι προγραμματιστές αξιοποιούν εργαλεία όπως το \en{VirtualBox} ή το \en{VMware Workstation} για να δοκιμάζουν πολλαπλά λειτουργικά συστήματα στην ίδια φυσική μηχανή.
\end{itemize}

\section{\en{Containers}}

Τα \en{containers} (ή «κοντέινερς») αποτελούν μια πιο ελαφριά μορφή εικονοποίησης. Αντί να δημιουργείται ένα πλήρες εικονικό \en{hardware} περιβάλλον για κάθε \en{guest}, τα \en{containers} μοιράζονται τον ίδιο πυρήνα (\en{kernel}) του λειτουργικού συστήματος, ενώ απομονώνουν τις διεργασίες και τις βιβλιοθήκες που απαιτούνται για την εκτέλεση εφαρμογών.

\subsection{Βασικές αρχές και τεχνολογίες}

Το κλειδί για τη λειτουργία των \en{containers} είναι η αξιοποίηση των δυνατοτήτων απομόνωσης (\en{namespaces}) και ελέγχου πόρων (\en{cgroups}) που προσφέρουν οι μοντέρνοι πυρήνες \en{Linux}. Αυτά επιτρέπουν σε κάθε \en{container} να διατηρεί το δικό του χώρο διεργασιών, δικτύου και συστήματος αρχείων, χωρίς να γνωρίζει την ύπαρξη άλλων \en{containers}.

Μία από τις πιο διαδεδομένες πλατφόρμες για τη δημιουργία και τη διαχείριση \en{containers} είναι το \en{Docker}, το οποίο έχει γίνει συνώνυμο με την ορολογία των \en{containers}. Άλλες σχετικές τεχνολογίες περιλαμβάνουν το \en{LXC} (\en{Linux Containers}), το \en{Podman} και πλατφόρμες ενορχήστρωσης όπως το \en{Kubernetes}.

\subsection{Πλεονεκτήματα και μειονεκτήματα}

\paragraph{Πλεονεκτήματα}
\begin{itemize}
  \item \textbf{Ελαφρότητα}: Επειδή οι \en{containers} δεν περιέχουν ολόκληρο λειτουργικό σύστημα, το αποτύπωμά τους σε \en{CPU}, μνήμη και αποθηκευτικό χώρο είναι σημαντικά μικρότερο.
  \item \textbf{Ταχύτητα}: Η εκκίνηση ενός \en{container} είναι ταχύτατη, καθώς δεν απαιτείται φόρτωση νέου πυρήνα. Επιπλέον, η ανάπτυξη νέων εκδόσεων εφαρμογών γίνεται με μεγαλύτερη ευκολία.
  \item \textbf{Ευκολία μεταφοράς (\en{portability})}: Τα \en{containers} προσφέρουν ένα συνεπές περιβάλλον εκτέλεσης, επιτρέποντας στις εφαρμογές να «τρέχουν» χωρίς αλλαγές σε διαφορετικά συστήματα \en{Linux}, \en{Windows} ή \en{macOS} (με τη βοήθεια ειδικών μηχανισμών).
\end{itemize}

\paragraph{Μειονεκτήματα}
\begin{itemize}
  \item \textbf{Κοινός πυρήνας (\en{kernel})}: Όλοι οι \en{containers} μοιράζονται τον ίδιο πυρήνα. Επομένως, αν εντοπιστεί ένα κενό ασφαλείας σε αυτόν, επηρεάζονται όλα τα \en{containers}.
  \item \textbf{Περιορισμοί σε \en{OS-level} διαφοροποιήσεις}: Δεδομένου ότι δεν υπάρχει ξεχωριστός πυρήνας, οι \en{containers} πρέπει να είναι συμβατοί με την έκδοση του πυρήνα του \en{host}. Αυτό περιορίζει τη δυνατότητα εκτέλεσης διαφορετικών λειτουργικών συστημάτων (π.χ. \en{Windows} σε \en{Linux} μηχανή).
\end{itemize}

\subsection{Χρήσεις και παραδείγματα}

\begin{itemize}
  \item \textbf{\en{Microservices}}: Οι σύγχρονες αρχιτεκτονικές \en{microservices} βασίζονται συχνά σε \en{containers}, αφού κάθε υπηρεσία πακετάρεται με τις βιβλιοθήκες της και τρέχει ανεξάρτητα.
  \item \textbf{\en{Continuous Integration/Continuous Deployment (CI/CD)}}: Εργαλεία όπως το \en{Jenkins}, το \en{GitLab CI} και το \en{GitHub Actions} υποστηρίζουν \en{Docker containers} για να εξασφαλίσουν επαναληψιμότητα και σταθερή συμπεριφορά κατά τη φάση δοκιμών και ανάπτυξης.
  \item \textbf{\en{Serverless} υποδομές}: Αν και τα \en{serverless} περιβάλλοντα κρύβουν την εσωτερική αρχιτεκτονική, πολλά από αυτά βασίζονται σε \en{containers} για την εκτέλεση των λειτουργιών (\en{functions}) των χρηστών.
\end{itemize}

\section{Σύγκριση \en{Virtual Machines} και \en{Containers}}

Παρά το γεγονός ότι αμφότερες οι τεχνολογίες εντάσσονται στο πλαίσιο της εικονοποίησης (\en{virtualization}), υπάρχουν ουσιώδεις διαφορές μεταξύ \en{virtual machines} και \en{containers}. Στον Πίνακα~\ref{tab:vm_containers} παρουσιάζονται συνοπτικά ορισμένες βασικές διαφορές.

\begin{table}[h]
  \centering
  \begin{tabular}{p{0.3\textwidth} p{0.3\textwidth} p{0.3\textwidth}}
    \hline
    \textbf{Χαρακτηριστικό} & \textbf{\en{Virtual Machines}} & \textbf{\en{Containers}} \\
    \hline
    \textbf{Απομόνωση} & Πλήρης απομόνωση μέσω \en{hypervisor} & Απομόνωση σε επίπεδο \en{OS} (\en{namespaces}, \en{cgroups}) \\
    \textbf{Μέγεθος} & Μεγαλύτερο, αφού περιέχει ολόκληρο \en{OS} & Μικρότερο, μόνο οι βιβλιοθήκες και οι εξαρτήσεις της εφαρμογής \\
    \textbf{Κατανάλωση πόρων} & Υψηλότερη & Χαμηλότερη \\
    \textbf{Εκκίνηση} & Αργή, απαιτεί φόρτωση λειτουργικού συστήματος & Γρήγορη, καθώς μοιράζεται τον πυρήνα του \en{host} \\
    \textbf{Διαφορετικά λειτουργικά συστήματα} & Δυνατή εκτέλεση \en{Windows}, \en{Linux}, κλπ. στον ίδιο \en{host} & Όλοι οι \en{containers} πρέπει να είναι συμβατοί με τον πυρήνα του \en{host} \\
    \hline
  \end{tabular}
  \caption{Συνοπτική σύγκριση \en{Virtual Machines} και \en{Containers}}
  \label{tab:vm_containers}
\end{table}

Από τα παραπάνω γίνεται σαφές ότι οι \en{virtual machines} είναι μια πιο «βαριά» αλλά και πιο γενική λύση, αφού υποστηρίζουν πλήρως διαφορετικά λειτουργικά συστήματα και αυξημένα επίπεδα απομόνωσης. Αντίθετα, τα \en{containers} προσφέρουν ταχύτητα, ευελιξία και ελαφρότητα, με βασική προϋπόθεση ότι μοιράζονται κοινό πυρήνα λειτουργικού συστήματος.

\section{Προκλήσεις και τάσεις}

\subsection{Ασφάλεια (\en{Security})}

Οι ανησυχίες γύρω από την ασφάλεια αυξάνονται καθώς όλο και περισσότερες υποδομές βασίζονται σε εικονικά περιβάλλοντα. Στις \en{virtual machines}, ένας επιτιθέμενος που αποκτά πρόσβαση σε μια \en{guest} μπορεί θεωρητικά να προσπαθήσει να «αποδράσει» στο επίπεδο του \en{hypervisor}, ενώ στους \en{containers}, η κοινή χρήση του πυρήνα καθιστά τον έλεγχο ασφάλειας του πυρήνα απολύτως κρίσιμο. Γι’ αυτό, οι διαχειριστές φροντίζουν να παρακολουθούν εκδόσεις πυρήνων, ενημερώσεις λογισμικού και πολιτικές πρόσβασης (\en{AppArmor}, \en{SELinux} κ.λπ.) για να περιορίσουν τους κινδύνους.

\subsection{Διαχείριση κλίμακας (\en{Scalability})}

Η μαζική ανάπτυξη εκατοντάδων ή χιλιάδων \en{containers} ή \en{virtual machines} σε περιβάλλοντα \en{cloud} επιτάσσει εξελιγμένες τεχνικές αυτοματισμού και ενορχήστρωσης (\en{orchestration}). Τα εργαλεία \en{Kubernetes}, \en{Docker Swarm} και \en{Mesos} παρέχουν δυνατότητες αυτόματης διαχείρισης (\en{autoscaling}), παρακολούθησης (\en{monitoring}) και κατανομής φόρτου (\en{load balancing}), καθιστώντας τη λειτουργία σύνθετων συστημάτων βιώσιμη σε μεγάλη κλίμακα.

\subsection{Υβριδικές προσεγγίσεις}

Σε πολλά σύγχρονα περιβάλλοντα, τα \en{containers} συνυπάρχουν με τις \en{virtual machines}. Για παράδειγμα, σε ένα \en{public cloud}, οι υπηρεσίες συνήθως τρέχουν σε \en{virtual machines}, ενώ εντός αυτών εκτελούνται \en{containers}. Αυτή η πολυεπίπεδη αρχιτεκτονική προσφέρει συνδυαστικά τα πλεονεκτήματα της απομόνωσης \en{hypervisor} με την ελαφρότητα και ευελιξία των \en{containers}.

\section{Σύνοψη}

Η εικονοποίηση (\en{virtualization}) αποτελεί κεντρικό πυλώνα στη σύγχρονη αρχιτεκτονική υποδομών \en{IT}, ενισχύοντας την αποτελεσματική διαχείριση πόρων και προσφέροντας δυνατότητες απομόνωσης και φορητότητας. Οι \en{virtual machines} (\en{VMs}) και τα \en{containers} αντιπροσωπεύουν δύο διαφορετικές, αλλά συμπληρωματικές, προσεγγίσεις εικονοποίησης. Οι \en{VMs} παρέχουν πλήρη απομόνωση και δυνατότητα εκτέλεσης διαφορετικών λειτουργικών συστημάτων στον ίδιο \en{host}, αποτελώντας μια καθιερωμένη λύση για πολλά περιβάλλοντα \en{cloud} και \en{on-premises}. Από την άλλη πλευρά, τα \en{containers} εισάγουν μια ταχύτατη, ελαφριά και ευέλικτη εναλλακτική, κατάλληλη κυρίως για εφαρμογές \en{microservices}, \en{CI/CD} ροές και περιπτώσεις όπου επιδιώκεται γρήγορη ανάπτυξη και απομόνωση εφαρμογών εντός κοινού πυρήνα.

Παρά τις επιμέρους προκλήσεις ασφαλείας και διαχείρισης, και οι δύο τεχνολογίες εξελίσσονται διαρκώς, υποβοηθούμενες από ένα πλούσιο οικοσύστημα εργαλείων ενορχήστρωσης, παρακολούθησης και αυτοματισμού. Στο μέλλον, η εικονική και η φυσική υποδομή αναμένεται να συνυπάρχουν σε ακόμη πιο πολύπλοκα, υβριδικά μοντέλα, παρέχοντας υποστήριξη για ποικίλα περιβάλλοντα, εφαρμογές και φόρτους εργασίας. Μέσα σε αυτό το τοπίο, η κατανόηση τόσο των \en{virtual machines} όσο και των \en{containers} παραμένει κομβική για την αποτελεσματική σχεδίαση και λειτουργία υπολογιστικών συστημάτων σε όλα τα επίπεδα της βιομηχανίας και της έρευνας.
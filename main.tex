\documentclass[a4paper,12pt,twoside]{book}
\usepackage{upatras-thesis-kerkis}

\newcommand{\shortdoctitle}{Διπλωματική Εργασία}
\newcommand{\doctitle}{Αξιοποίηση υπηρεσιών ιδιωτικού νέφους στα πλαίσια του Διαδικτύου των Αντικειμένων}
\newcommand{\engdoctitle}{Exploitation of Private Cloud Services in the IoT space}
\newcommand{\docsubtitle}{Υπότιτλος εγγράφου}
\newcommand{\division}{Τομέας Ηλεκτρονικής και Υπολογιστών}
\newcommand{\divisionlarge}{ΤΟΜΕΑΣ ΗΛΕΚΤΡΟΝΙΚΗΣ ΚΑΙ ΥΠΟΛΟΓΙΣΤΩΝ}
\newcommand{\lab}{Εργαστήριο Συστημάτων Υπολογιστών}
\newcommand{\lablarge}{ΕΡΓΑΣΤΗΡΙΟ ΣΥΣΤΗΜΑΤΩΝ ΥΠΟΛΟΓΙΣΤΩΝ}
\newcommand{\me}{Αθάνασιου Καλαθέρη του Μιχαήλ}
\newcommand{\nomme}{Αθανάσιος Καλαθέρης του Μιχαήλ}
\newcommand{\nommesmall}{Αθανάσιος Καλαθέρης}
\newcommand{\nommesmallenglish}{Athanasios Kalatheris}
\newcommand{\studnum}{1019421}
\newcommand{\keywords}{Λέξεις Κλειδιά}
\newcommand{\monthyear}{Ιούλιος 2025}

\newcommand{\supname}{Κλεάνθης Θραμπουλίδης}
\newcommand{\supnameegnlish}{Kleanthis Thramboulidis}
\newcommand{\suptitle}{Καθηγητής}
\newcommand{\supdep}{Τμήμα Μηχανικών Υπολογιστών και Πληροφορικής}
\newcommand{\supuni}{Πανεπιστήμιο Πατρών}

\newcommand{\cosupname}{Χρήστος Φείδας}
\newcommand{\cosuptitle}{Καθηγητής}
\newcommand{\cosupdep}{Τμήμα Ηλεκτρολόγων Μηχανικών και Τεχνολογίας Υπολογιστών}
\newcommand{\cosupuni}{Πανεπιστήμιο Πατρών}

\newcommand{\headofdivision}{Ονοματεπώνυμο Διεθυντή Τομέα}
\newcommand{\headofdivisiontitle}{Τίτλος Διευθυντή Τομέα}

\author{\me}


% GLOSSARY

% BEGIN DOCUMENT
\begin{document}

% SET PAGE NUMBERING TO ROMAN
\pagenumbering{roman}
\setcounter{page}{3}

%*************************%
%         TITLES          %
%*************************%

% \begin{titlepage}
%   % Upper part of the page
%   \begin{tabular}{@{}l@{\qquad}l@{}}
%     \includegraphics[width= 0.2\textwidth]{images/University_of_Patras_(seal).png} &
%     \begin{minipage}[b]{0.8\textwidth}
%       \raggedright
%       {\LARGE \textbf{ΠΑΝΕΠΙΣΤΗΜΙΟ ΠΑΤΡΩΝ}}\\
%       {\large \textbf{ΤΜΗΜΑ ΗΛΕΚΤΡΟΛΟΓΩΝ ΜΗΧΑΝΙΚΩΝ\\ΚΑΙ ΤΕΧΝΟΛΟΓΙΑΣ ΥΠΟΛΟΓΙΣΤΩΝ\\[0.5cm]}}
%       \textbf{\divisionlarge} \\
%       \textbf{\lablarge}\\
%     \end{minipage}
%   \end{tabular}
%   \hfill \break
%   \noindent \rule{\textwidth}{0.4pt}\\
%   \vfill
%   \begin{center}
%     \textbf{\LARGE \doctitle }\\
%     \vfill

%      % Middle part of the page
%     {\LARGE{\shortdoctitle }}\\ [0.5cm]
%     {\LARGE \textbf{\nomme} }\\[0.5cm]

%     % Bottom of the page
%     \vfill
%     {\large \textbf{Επιβλέπων: \supname}}\\[0.5cm]
%     \large{ \textbf{Πάτρα, \monthyear}}
%   \end{center}
% \end{titlepage}

\begin{titlepage}
  % Optional vertical space if you want to push content down a bit:
  % \vspace*{2em}

  \begin{tabular}{@{}l@{\hspace{1cm}}l@{}}
    % LOGO on the left
    \includegraphics[width=0.2\textwidth]{images/University_of_Patras_(seal).png}
    &
    % University / Dept. / Lab info on the right
    \begin{minipage}[b]{0.7\textwidth}
      \raggedright
      {\LARGE \bfseries ΠΑΝΕΠΙΣΤΗΜΙΟ ΠΑΤΡΩΝ}\\[0.5em]
      {\large \bfseries ΤΜΗΜΑ ΗΛΕΚΤΡΟΛΟΓΩΝ ΜΗΧΑΝΙΚΩΝ ΚΑΙ ΤΕΧΝΟΛΟΓΙΑΣ ΥΠΟΛΟΓΙΣΤΩΝ}\\[1em]
      {\bfseries \divisionlarge}\\
      {\bfseries \lablarge}
    \end{minipage}
  \end{tabular}

  % Horizontal rule after the top block
  \vspace{1em}
  \rule{\textwidth}{0.4pt}
  \vspace{2em}

  \vfill

  % Centered middle and bottom parts
  \begin{center}
    % Main title
    {\LARGE \bfseries \doctitle}\\[2em]

    \vfill
    % Subtitle or short title
    {\Large \shortdoctitle}\\[1em]
    % Author name
    {\Large \bfseries \nomme}\\[0.5em]

    \vfill % Push the rest down

    % Supervisor and date at the bottom
    {\Large \bfseries Επιβλέπων: \supname}\\[0.5em]
    {\Large \bfseries Πάτρα, \monthyear}
  \end{center}
\end{titlepage}

\clearpage

\pagestyle{empty}

\vspace*{\fill}
\noindent \rule{\textwidth}{0.4pt}\\
\vspace{1em}

\noindent Πανεπιστήμιο Πατρών, Τμήμα Ηλεκτρολόγων Μηχανικών και Τεχνολογίας Υπολογιστών.

\vspace{0.5em}

\noindent {\nomme}

\vspace{0.5em}

\noindent \copyright \, \monthyear \ -- Με την επιφύλαξη παντός δικαιώματος.

\vspace{0.5em}


\noindent Το σύνολο της εργασίας αποτελεί πρωτότυπο έργο, παραχθέν από τον
{\nomme}, και δεν παραβιάζει δικαιώματα τρίτων καθ’ οιονδήποτε
τρόπο. Αν η εργασία περιέχει υλικό, το οποίο δεν έχει παραχθεί από τον/την ίδιο/α, αυτό
είναι ευδιάκριτο και αναφέρεται ρητώς εντός του κειμένου της εργασίας ως προϊόν
εργασίας τρίτου, σημειώνοντας με παρομοίως σαφή τρόπο τα στοιχεία ταυτοποίησής
του, ενώ παράλληλα βεβαιώνει πως στην περίπτωση χρήσης αυτούσιων γραφικών
αναπαραστάσεων, εικόνων, γραφημάτων κ.λπ., έχει λάβει τη χωρίς περιορισμούς άδεια
του κατόχου των πνευματικών δικαιωμάτων για την συμπερίληψη και επακόλουθη
δημοσίευση του υλικού αυτού.


\vspace{1em}
\noindent \rule{\textwidth}{0.4pt}

\clearpage

\pagestyle{empty}
\begin{center}
  {\LARGE \textbf{Πιστοποίηση}\\[1cm]}
  \large Πιστοποιείται ότι η διπλωματική εργασία με τίτλο\\[1cm]
  \textbf{\large \doctitle }\\[1cm]
  του φοιτητή του Τμήματος Ηλεκτρολόγων Μηχανικών και Τεχνολογίας Υπολογιστών\\[1cm]
  \me \\[0.5cm]
  (Α.Μ.: \studnum )\\[1cm]
  παρουσιάτηκε δημόσια και εξετάστηκε στο Tμήμα  Ηλεκτρολόγων Μηχανικών και Τεχνολογίας Υπολογιστών στις\\[1cm]
  \Large{\_\_/\_\_/\_\_\_}\\[0.5cm]
  και εξετάστηκε από την ακόλουθη εξεταστική επιτροπή:\\[0.5cm]
  \supname, \suptitle, \supdep\\[0.5cm]
  \cosupname, \cosuptitle, \cosupdep\\
\end{center}
\vfill
\begin{minipage}{0.5\textwidth}
  \begin{flushleft} \large
    Ο Επιβλέπων\\[1cm]
    \supname \\
    \emph{\suptitle}
  \end{flushleft}
\end{minipage}
\begin{minipage}{0.5\textwidth}
  \begin{flushright} \large
    Ο Συνεπιβλέπων\\[1cm]
    \cosupname\\
    \emph{\cosuptitle}
  \end{flushright}
\end{minipage}

\clearpage

\begin{center}
  {\LARGE \textbf{Πρόλογος}}\\[1cm]
  \section*{Ευχαριστίες}\label{sec:thankyounotes}
\end{center}

Αρχικά, εκφράζω τις θερμές μου ευχαριστίες στον επιβλέποντα καθηγητή της διπλωματικής μου, κ. Θραμπουλίδη Κλεάνθη, για την ευκαιρία που μου παρείχε να ασχοληθώ με το συγκεκριμένο αντικείμενο, καθώς και για την πολύτιμη καθοδήγηση και στήριξη που μου προσέφερε κατά τη διάρκεια εκπόνησης της εργασίας μου. Οι γνώσεις και οι εμπειρίες που αποκόμισα κατά τη διάρκεια αυτής της διαδικασίας αποτελούν πολύτιμο εφόδιο για τη μελλοντική μου πορεία.

\vspace{5mm}
Επιπλέον, ευχαριστώ θερμά τον συνάδελφο Τριανταφύλλου Αχιλλέα για την καθοδήγησή του και τη βοήθεια που μου παρείχε στην κατανόηση του συστήματος.

\vspace{5mm}
Τέλος, εκφράζω την ειλικρινή μου ευγνωμοσύνη στους γονείς μου, Μιχάλη και Ευαγγελία, καθώς και σε όλη την οικογένειά μου, για τη διαρκή πνευματική υποστήριξη και κατανόηση που έδειξαν καθ' όλη τη διάρκεια των σπουδών μου στο Πανεπιστήμιο Πατρών.

\clearpage

% \include{intro/about}
% \clearpage

\pagestyle{plain}
\begin{center}
  {\LARGE \textbf{Περίληψη}}\\[1cm]
  \textbf{\doctitle}\\[1cm]
  \begin{minipage}{0.38\textwidth}
    \begin{flushleft}
      \textbf{Ονοματεπώνυμο Φοιτητή}\\
      \textbf{\nommesmall}
    \end{flushleft}
  \end{minipage}
  \begin{minipage}{0.38\textwidth}
    \begin{flushright}
      \textbf{Ονοματεπώνυμο Επιβλέποντος}\\
      \textbf{\supname}
    \end{flushright}
  \end{minipage}
\end{center}

\vspace{10mm}

Η παρούσα διπλωματική εργασία εξετάζει την αξιοποίηση υπηρεσιών ιδιωτικού \en{cloud} στο πλαίσιο του 
Διαδικτύου των Αντικειμένων (\en{IoT}). Ειδικότερα, η εργασία επικεντρώνεται στη χρήση του \en{Kubernetes} 
ως υποδομή ιδιωτικού \en{cloud} για την ανάπτυξη και διαχείριση μικροϋπηρεσιών (\en{microservices}) καθώς 
και πώς η ιδιωτική υποδομή αυτή θα μπορούσε να υλοποιηθεί σε ενα περιβαλλόν εργοστασίου. Στην συνέχεια
σχεδιάστηκαν και υλοποιήθηκαν ένας αριθμός από μικρουπηρεσίες (\en{microservices}) με στόχο τον έλεγχο και την παρακολούθηση
μιας γραμμής παραγωγής, επιδεικνύοντας την αποτελεσματικότητα και την ευελιξία που παρέχουν οι τεχνολογίες \en{cloud} σε 
βιομηχανικές εφαρμογές.

Η καινοτομία του συστήματος έγκειται στη χρήση κυβερνοφυσικών \en{microservices}, οι οποίες αλληλεπιδρούν με ρομποτικούς βραχίονες 
μέσω \en{HTTP requests}. Κάθε \en{microservice} αντιπροσωπεύει ένα συγκεκριμένο βήμα παραγωγής, επιτυγχάνοντας ουσιαστικό διαχωρισμό \en{(decoupling)} 
ανάμεσα στη λογική των βημάτων και στους ρομποτικούς βραχίονες. Οι βραχίονες λειτουργούν ως απομακρυσμένα 
\en{API endpoints} που εκτελούν την κίνηση ή τη διεργασία που ζητούν τα \en{microservices}, χωρίς να χρειάζεται 
να γνωρίζουν οι ίδιοι τη σειρά ή τον τρόπο υλοποίησης των επιμέρους σταδίων. Επιπλέον, πραγματοποιήθηκε ενορχήστρωση 
των υπηρεσιών \en{(service orchestration)} για τον συγχρονισμό και το συντονισμό των διαφορετικών βημάτων παραγωγής, 
διασφαλίζοντας την ομαλή και ευέλικτη ροή της διαδικασίας.


\clearpage
\selectlanguage{english}
\begin{center}
  {\LARGE \textbf{Extensive English Summary}}\\[1cm]
  \textbf{\engdoctitle}\\[1cm]
  \begin{minipage}{0.38\textwidth}
    \begin{flushleft}
      \textbf{Student name, surname}\\
      \textbf{\nommesmallenglish}
    \end{flushleft}
  \end{minipage}
  \begin{minipage}{0.38\textwidth}
    \begin{flushright}
      \textbf{Supervisor name, surname}\\
      \textbf{\supnameegnlish}
    \end{flushright}
  \end{minipage}
\end{center}
\vspace{10mm}

This thesis examines the utilization of private cloud services in the context of the Internet of Things (IoT). 
Specifically, the study focuses on the use of Kubernetes as a private cloud infrastructure for the development 
and management of microservices, as well as how this private infrastructure could be implemented within 
a factory environment. Subsequently, a number of microservices were designed and implemented with the aim of controlling 
and monitoring a production line, demonstrating the effectiveness and flexibility offered by cloud technologies in 
industrial applications.

The innovation of the system lies in the use of cyber-physical microservices, which interact with robotic arms
 via HTTP requests. Each microservice represents a specific production step, achieving substantial decoupling 
 between the logic of the steps and the robotic arms. The robotic arms function as remote API endpoints that execute the 
 movement or process requested by the microservices, without needing to know the sequence or the implementation details 
 of the individual stages themselves. In addition, service orchestration was implemented to 
 synchronize and coordinate the various production steps, ensuring a smooth and flexible workflow.

\selectlanguage{greek}
\clearpage

\pagestyle{empty}

\renewcommand{\contentsname}{Περιεχόμενα}
{\hypersetup{linkcolor=black}
  \tableofcontents
}
\clearpage

%*************************%
%    1. Lists of figures  %
%    2. List of Tables    %
%    3. Glossary          %
%*************************%
\renewcommand{\listfigurename}{Ευρετήριο Εικόνων}
\renewcommand{\listtablename}{Ευρετήριο Πινάκων}
\listoffigures{}
\listoftables
% \selectlanguage{english}
% \printglossaries
% \selectlanguage{greek}
\clearpage


\mainmatter % book mode only
\clearpage

\pagestyle{fancy}
\pagenumbering{arabic}
\setcounter{page}{1}

%*************************%
%       Main Chapters     %
%*************************%

%!TEX root = ../../main.tex

\chapter{Εισαγωγή}\label{ch:introduction}


\markboth{Εισαγωγή}{}

\section{Αντικείμενο και στόχοι της διπλωματικής εργασίας}
Αντικείμενο της παρούσας εργασίας είναι η μελέτη και η υλοποίηση μιας παραδοσιακής γραμμής συναρμολόγησης στη βιομηχανία, αξιοποιώντας τεχνολογίες του Διαδικτύου των Αντικειμένων και των Containers. Πιο αναλυτικά, εξετάστηκαν οι δυσκολίες και προκλήσεις που αντιμετωπίζει η βιομηχανία σε μια παραδοσιακή γραμμή συναρμολόγησης και έγινε μελέτη σε διάφορα IoT 
πρωτόκολλ-α και μεθόδους επικοινωνίας. Έτσι, καταλήξαμε στη χρήση του HTTP για τον έλεγχο των φυσικών διεργασίων και του \en{Apache Kafka} για τη ροή μηνυμάτων μεταξύ των microservices, παράλληλα με τους Docker Containers και την πλατφόρμα ορχήστρωσης \en{Kubernetes} ως ιδιωτική υποδομή, με στόχο να διερευνηθούν οι δυνατότητες αξιοποίησής τους στον συγκεκριμένο τομέα.

Η διαδικασία αυτή υλοποιείται μέσα από την ανάπτυξη ενός κατανεμημένου συστήματος συναρμολόγησης, στο οποίο τα επιμέρους υποσυστήματα δεν είναι χωροταξικά συγκεντρωμένα και επικοινωνούν μεταξύ τους με πρωτόκολλα επιπέδου εφαρμογής και ανταλλαγή μηνυμάτων. Η επικοινωνία αυτή επιτυγχάνεται μέσω λογισμικού που αναπτύχθηκε και εκτελείται μέσα σε περιβάλλοντα \en{Containers}, τα οποία ορχηστρώνονται από το \en{Kubernetes}, σε συνδυασμό με τη χρήση \en{Microservices}.

Το σύστημα συναρμολόγησης που χρησιμοποιήθηκε ως σενάριο μελέτης είναι το \en{Gregor Office Chair Assembly System} \cite{thramboulidis_cyber-physical_2018}, το οποίο αποτελείται από τρεις ρομποτικούς βραχίονες και δύο τράπεζες εργασίας για τη συναρμολόγηση καρεκλών. Για την προσομοίωση του συστήματος, κατασκευάστηκε μια πειραματική διάταξη, ώστε να μελετηθούν οι δυνατότητες επικοινωνίας μεταξύ συσκευών με περιορισμένες υπολογιστικές δυνατότητες, χρησιμοποιώντας \en{HTTP}, \en{Apache Kafka}, \en{Docker Containers} και \en{Kubernetes}.

Μέσα από την εν λόγω μελέτη, εκτιμάται ότι η εφαρμογή των συγκεκριμένων τεχνολογιών μπορεί να προσφέρει ποικίλα πλεονεκτήματα στην ανάπτυξη λογισμικού για αντίστοιχα συστήματα στη βιομηχανία. Ενδεικτικά, η χρήση τους μπορεί να μειώσει το κόστος συναρμολόγησης, να επιταχύνει τον χρόνο ανάπτυξης του λογισμικού και να επιτρέψει την επαναχρησιμοποίηση ήδη υπάρχοντος κώδικα μέσω των \en{Containers} για τη δημιουργία νέων συστημάτων συναρμολόγησης. Τέλος, προέκυψαν διάφορα συμπεράσματα σχετικά με την εφαρμογή τους, τα οποία δείχνουν ότι η ενσωμάτωσή τους στη βιομηχανία μπορεί να αποφέρει σημαντικά οφέλη τόσο στη διαδικασία ανάπτυξης των συστημάτων όσο και στον τρόπο διαχείρισης αυτών από τους μηχανικούς που τα υποστηρίζουν.
\section{Μεθοδολογία}

\section{Οργάνωση του κειμένου}

%!TEX root = ../../main.tex

\chapter{Αξιοποίηση του \en{Apache Kafka}}\label{ch:apache_kafka}
%!TEX root = ../../main.tex

\chapter{Αξιοποίηση του \en{Apache Kafka}}\label{ch:apache_kafka}
\chapter{\en{Kubernetes}}\label{ch:kubernetes}

Το \en{Kubernetes} αποτελεί μία από τις πιο δημοφιλείς και ολοκληρωμένες πλατφόρμες ενορχήστρωσης (\en{orchestration}) και διαχείρισης εφαρμογών που εκτελούνται σε περιβάλλοντα \en{containers}, αλλά και σε συνδυασμό με \en{virtual machines}. Αναπτύχθηκε αρχικά από την \en{Google} και προσφέρθηκε ως έργο ανοιχτού κώδικα (\en{open-source}), με ισχυρή υποστήριξη από την κοινότητα. Σήμερα, έχει εξελιχθεί σε έναν από τους βασικότερους πυλώνες για την ανάπτυξη κατανεμημένων συστημάτων μεγάλης κλίμακας (\en{distributed systems}) και υποδομών \en{cloud}.

Στο πλαίσιο των προηγούμενων κεφαλαίων, όπου εξετάστηκαν το \en{IoT} (\en{Internet of Things}), τα \en{CPS} (\en{Cyber-Physical Systems}) και οι τεχνολογίες εικονοποίησης (\en{virtual machines} και \en{containers}), το παρόν κεφάλαιο αναδεικνύει τον ρόλο του \en{Kubernetes} ως ενοποιημένη πλατφόρμα υλοποίησης, ενορχήστρωσης και διαχείρισης αυτών των υποσυστημάτων. Αναλυτικότερα, θα παρουσιαστούν οι βασικές αρχές του \en{Kubernetes}, η αρχιτεκτονική του, οι δυνατότητες ασφαλούς και κλιμακούμενης (\en{scalable}) ανάπτυξης εφαρμογών, καθώς και η σύνδεσή του με τις έννοιες του \en{IoT}, των \en{CPS} και των \en{VMs}.

\section{Βασικές αρχές του \en{Kubernetes}}

Το \en{Kubernetes} ακολουθεί μια αρχιτεκτονική \en{client-server}, όπου ένα κεντρικό \en{control plane} αναλαμβάνει τον έλεγχο πολλών \en{worker nodes}. Στο \en{control plane} περιλαμβάνονται διάφορες κρίσιμες υπηρεσίες, όπως ο \en{API server}, ο \en{scheduler} και οι \en{controllers}, ενώ στους \en{worker nodes} εκτελούνται τα πραγματικά φορτία εργασίας (\en{workloads}), δηλαδή οι \en{containers} ή οι \en{virtual machines}.

% \selectlanguage{english}
% \begin{center}
% \begin{tikzpicture}[
%     font=\small,
%     node distance=1.5cm, % Adjust as needed
%     >=stealth',
%     every node/.style={draw, rectangle, rounded corners, align=center}
% ]
% % Control Panel
% \coordinate (clusterCenter) at (0,0);
% \node[fill=blue!10] (api) at (clusterCenter) {kube-apiserver}; 
% \node[fill=blue!10, above=of api,xshift=3cm]     (ccm)       {cloud-controller-manager};
% \node[fill=blue!10, above=of api,xshift=-3cm]     (kcm)       {kube-controller-manager};
% \node[fill=blue!10, left=of api]      (etcd)      {etcd};
% \node[fill=blue!10, right=of api]     (scheduler)        {scheduler};
% \node[
%     draw,
%     dashed,
%     rounded corners,
%     label={[yshift=0.1cm]above:\textbf{Control Plane}},
%     fit=(ccm)(api)(etcd)(kcm)(scheduler)
% ] (cpbox) {};
% \draw[<->] (api) -- (ccm);
% \draw[->]  (api) -- (etcd);
% \draw[->]  (scheduler) -- (api);
% \draw[->]  (kcm) -- (api);

% % Worker Node 1 (left)
% \node[fill=green!10, below=of clusterCenter, xshift=-1.3cm]                  (kubeproxy1) {kube-proxy};
% \node[fill=green!10, left=of kubeproxy1] (kubelet1) {kubelet};
% \node[fill=green!10, below=of kubeproxy1]                (pods1)      {Pods};

% \node[
%     draw,
%     dashed,
%     rounded corners,
%     label={[yshift=0.1cm]above:\textbf{Worker Node 1}},
%     fit=(kubelet1)(kubeproxy1)(pods1)
% ] (wn1box) {};

% % Worker Node 2 (right)
% \node[fill=green!10, below=of clusterCenter, xshift=1.3cm] (kubelet2) {kubelet};
% \node[fill=green!10, right=of kubelet2]                  (kubeproxy2) {kube-proxy};
% \node[fill=green!10, below=of kubeproxy2]                (pods2)      {Pods};

% \node[
%     draw,
%     dashed,
%     rounded corners,
%     label={[yshift=0.1cm]above:\textbf{Worker Node 2}},
%     fit=(kubelet2)(kubeproxy2)(pods2)
% ] (wn2box) {};
% \end{tikzpicture}
% \end{center}
% \begin{center}
% \begin{tikzpicture}[
%     font=\small,
%     node distance=1.5cm, % Adjust as needed
%     >=stealth',
%     every node/.style={draw, rectangle, rounded corners, align=center}
% ]

% % 1) Place "cloud-controller-manager" at the origin (0,0)
% \node[fill=blue!10] (api) at (0,0) {kube-apiserver};

% % 2) Use relative positioning for other Control Plane nodes
% \node[fill=blue!10, above=of api]     (ccm)       {cloud-controller-manager};
% \node[fill=blue!10, left=3.2cm of api]      (etcd)      {etcd};
% \node[fill=blue!10, right=3.2cm of api]     (cm)        {kube-controller-manager};
% \node[fill=blue!10, right=of ccm]       (scheduler) {scheduler};

% % 3) Dashed bounding box around Control Plane
% \node[
%     draw,
%     dashed,
%     rounded corners,
%     label={[yshift=0.2cm]above:\textbf{Control Plane}},
%     fit=(ccm)(api)(etcd)(cm)(scheduler)
% ] (cpbox) {};

% % 4) Reference coordinate at bottom center of Control Plane bounding box
% \coordinate (cpcenter) at (cpbox.south);

% % -------------------------------------------------------
% % Worker Node 1 (left)
% % -------------------------------------------------------
% \node[fill=green!10, below=2cm of cpcenter, xshift=-5cm] (kubelet1) {kubelet};
% \node[fill=green!10, right=of kubelet1]                  (kubeproxy1) {kube-proxy};
% \node[fill=green!10, below=of kubeproxy1]                (pods1)      {Pods};

% \node[
%     draw,
%     dashed,
%     rounded corners,
%     label={[yshift=0.2cm]above:\textbf{Worker Node 1}},
%     fit=(kubelet1)(kubeproxy1)(pods1)
% ] (wn1box) {};

% % -------------------------------------------------------
% % Worker Node 2 (right)
% % -------------------------------------------------------
% \node[fill=yellow!10, below=2cm of cpcenter, xshift=2cm] (kubelet2) {kubelet};
% \node[fill=yellow!10, right=of kubelet2]                 (kubeproxy2) {kube-proxy};
% \node[fill=yellow!10, below=of kubeproxy2]               (pods2)      {Pods};

% \node[
%     draw,
%     dashed,
%     rounded corners,
%     label={[yshift=0.2cm]above:\textbf{Worker Node 2}},
%     fit=(kubelet2)(kubeproxy2)(pods2)
% ] (wn2box) {};

% % -------------------------------------------------------
% % Example arrows
% % -------------------------------------------------------
% \draw[<->] (api) -- (ccm);
% \draw[->]  (api) -- (etcd);
% \draw[->]  (scheduler) -- (api);
% \draw[->]  (cm) -- (api);

% \draw[->]  (kubelet1) -- (api);
% \draw[<->] (kubelet2) -- (api);

% % -------------------------------------------------------
% % "Cluster" bounding box around EVERYTHING
% % -------------------------------------------------------
% \node[
%     draw,
%     thick,
%     rounded corners,
%     label={[yshift=0.4cm]above:\textbf{Cluster}},
%     fit=(cpbox)(wn1box)(wn2box)
% ] (clusterbox) {};

% \end{tikzpicture}
% \end{center}
% \begin{center}
% \begin{tikzpicture}[
%     font=\small,
%     node distance=1cm,
%     >=stealth',
%     every node/.style={draw, rectangle, rounded corners, align=center}
% ]

% % -------------------------------------------------------
% % Control Plane components
% % -------------------------------------------------------
% \node[fill=blue!10] (ccm) {cloud-controller-manager};
% \node[fill=blue!10, below=of ccm] (api) {kube-apiserver};
% \node[fill=blue!10, left=of api] (etcd) {etcd};
% \node[fill=blue!10, below=of api] (cm) {kube-controller-manager};
% \node[fill=blue!10, left=of cm] (scheduler) {scheduler};

% % Dashed bounding box around Control Plane
% \node[
%     draw,
%     dashed,
%     rounded corners,
%     label={[yshift=0.2cm]above:\textbf{Control Plane}},
%     fit=(api)(scheduler)(cm)(etcd)(ccm)
% ] (cpbox) {};

% % -------------------------------------------------------
% % Reference coordinate at bottom center of Control Plane
% % -------------------------------------------------------
% \coordinate (cpcenter) at (cpbox.south);

% % -------------------------------------------------------
% % Worker Node 1 (left)
% % -------------------------------------------------------
% \node[fill=green!10, below=2cm of cpcenter, xshift=-3cm] (kubelet1) {kubelet};
% \node[fill=green!10, right=of kubelet1] (kubeproxy1) {kube-proxy};
% \node[fill=green!10, below=of kubeproxy1] (pods1) {Pods};

% \node[
%     draw,
%     dashed,
%     rounded corners,
%     label={[yshift=0.2cm]above:\textbf{Worker Node 1}},
%     fit=(kubelet1)(kubeproxy1)(pods1)
% ] (wn1box) {};

% % -------------------------------------------------------
% % Worker Node 2 (right)
% % -------------------------------------------------------
% \node[fill=yellow!10, below=2cm of cpcenter, xshift=3cm] (kubelet2) {kubelet};
% \node[fill=yellow!10, right=of kubelet2] (kubeproxy2) {kube-proxy};
% \node[fill=yellow!10, below=of kubeproxy2] (pods2) {Pods};

% \node[
%     draw,
%     dashed,
%     rounded corners,
%     label={[yshift=0.2cm]above:\textbf{Worker Node 2}},
%     fit=(kubelet2)(kubeproxy2)(pods2)
% ] (wn2box) {};

% % % API Server
% \draw[<->] (api) -- (ccm);
% \draw[->] (api) -- (etcd);

% % Scheduler
% \draw[->] (scheduler) -- (api);

% % Controller Manager
% \draw[->] (cm) -- (api);

% % Communication to kube-apiserver
% \draw[->] (kubelet1) -- (api);
% \draw[<->] (kubelet2) -- (api);

% % -------------------------------------------------------
% % "Cluster" bounding box around EVERYTHING
% % -------------------------------------------------------
% \node[
%     draw,
%     thick,
%     rounded corners,
%     label={[yshift=0.4cm]above:\textbf{Cluster}},
%     fit=(cpbox)(wn1box)(wn2box)
% ] (clusterbox) {};


% \end{tikzpicture}
% \end{center}
% \begin{tikzpicture}[
%     font=\small,
%     node distance=1.3cm,
%     auto,
%     >=stealth',
%     every node/.style={draw, rectangle, rounded corners, align=center}
% ]

% % -------------------------------------------------------
% % Control Plane components
% % -------------------------------------------------------
% \node[fill=blue!10] (ccm) {cloud-controller-manager};
% \node[fill=blue!10, below=of ccm] (api) {kube-apiserver};
% \node[fill=blue!10, left=of api] (etcd) {etcd};
% \node[fill=blue!10, below=of api] (cm) {kube-controller-manager};
% \node[fill=blue!10, left=of cm] (scheduler) {scheduler};

% % Dashed bounding box around Control Plane
% \node[
%     draw,
%     dashed,
%     rounded corners,
%     label={[yshift=0.2cm]above:\textbf{Control Plane}},
%     fit=(api)(scheduler)(cm)(etcd)(ccm)
% ] (cpbox) {};

% % -------------------------------------------------------
% % Worker Node 1
% % -------------------------------------------------------
% \node[fill=green!10, below=1.5cm of cm] (kubelet1) {kubelet};
% \node[fill=green!10, right=0.5cm of kubelet1] (kubeproxy1) {kube-proxy};
% \node[fill=green!10, below=of kubeproxy1] (pods1) {Pods};

% \node[
%     draw,
%     dashed,
%     rounded corners,
%     label={[yshift=0.2cm]above:\textbf{Worker Node 1}},
%     fit=(kubelet1)(kubeproxy1)(pods1)
% ] (wn1box) {};

% % -------------------------------------------------------
% % Worker Node 2
% % -------------------------------------------------------
% \node[fill=yellow!10, right=2.5cm of kubelet1] (kubelet2) {kubelet};
% \node[fill=yellow!10, right=of kubelet2] (kubeproxy2) {kube-proxy};
% \node[fill=yellow!10, below=of kubeproxy2] (pods2) {Pods};

% \node[
%     draw,
%     dashed,
%     rounded corners,
%     label={[yshift=0.2cm]above:\textbf{Worker Node 2}},
%     fit=(kubelet2)(kubeproxy2)(pods2)
% ] (wn2box) {};

% \node[
%     draw,
%     dashed,
%     rounded corners,
%     label={[yshift=0.2cm]above:\textbf{Cluster}},
%     fit=(cpbox)(wn1box)(wn2box)
% ] (cluster) {};
% % -------------------------------------------------------
% % Draw arrows to represent communication
% % -------------------------------------------------------
% % API Server
% \draw[<->] (api) -- (ccm);
% \draw[->] (api) -- (etcd);

% % Scheduler
% \draw[->] (scheduler) -- (api);

% % Controller Manager
% \draw[->] (cm) -- (api);

% % Communication to kube-apiserver
% \draw[->] (kubelet1) -- (api);
% \draw[<->] (kubelet2) -- (api);

% \end{tikzpicture}
\selectlanguage{greek}

\begin{itemize}
    \item \textbf{\en{API server}}: Παρέχει το κεντρικό σημείο επικοινωνίας μέσω \en{REST} \en{API}, μέσω του οποίου οι χρήστες, τα \en{controllers} και ο \en{scheduler} αλληλεπιδρούν με το \en{Kubernetes} \en{cluster}.
    \item \textbf{\en{etcd}}: Αποθηκεύει την κατάσταση (\en{state}) του \en{cluster}, π.χ. τις ρυθμίσεις (\en{configurations}) και τις προδιαγραφές των εφαρμογών.
    \item \textbf{\en{scheduler}}: Αποφασίζει σε ποιον \en{worker node} θα εκτελεστούν οι \en{pods}, βάσει διαθεσιμότητας πόρων (\en{CPU}, μνήμη) και διαφόρων πολιτικών.
    \item \textbf{\en{kube-controller-manager}}: Ελέγχουν και συντονίζουν τη «στοχευμένη» κατάσταση (\en{desired state}) με την «τρέχουσα» κατάσταση (\en{current state}). Για παράδειγμα, ένας \en{controller} εξασφαλίζει πως αν δηλωθούν 5 αντίγραφα (\en{replicas}) μιας εφαρμογής, θα εκτελούνται πάντοτε 5 \en{pods}.
\end{itemize}

Από την άλλη πλευρά, κάθε \en{worker node} φιλοξενεί:
\begin{itemize}
    \item Τον \textbf{\en{kubelet}}, ο οποίος επικοινωνεί με τον \en{API server} και αναλαμβάνει την υλοποίηση των εντολών στο τοπικό σύστημα.
    \item Τον \textbf{\en{kube-proxy}}, υπεύθυνο για τους κανόνες δικτύωσης και την προώθηση (\en{forwarding}) της κίνησης στα σωστά \en{pods}.
    \item Τον \textbf{\en{container runtime}} (\en{Docker}, \en{containerd} ή άλλο) για τη δημιουργία και εκτέλεση \en{containers}.
\end{itemize}

\section{\en{Pods}, \en{Services} και λοιποί πυρήνες μηχανισμοί}

Στο \en{Kubernetes}, η μικρότερη μονάδα ανάπτυξης εφαρμογών είναι το \en{pod}. Ένα \en{pod} μπορεί να περιέχει έναν ή περισσότερους \en{containers}, οι οποίοι μοιράζονται το ίδιο \en{namespace} δικτύου και δίσκου (\en{volumes}). Πάνω από τα \en{pods}, ορίζονται έννοιες υψηλότερου επιπέδου, όπως το \en{Deployment} (εξασφαλίζει κλιμάκωση και ενημερώσεις χωρίς διακοπή) και το \en{StatefulSet} (για εφαρμογές που διατηρούν κατάσταση). 

Για τη δικτύωση, το \en{Kubernetes} χρησιμοποιεί τους \en{Services}, οι οποίοι προσφέρουν ένα σταθερό σημείο πρόσβασης (\en{cluster IP} ή \en{load balancer IP}), ανεξάρτητα από το πού εκτελούνται τα \en{pods} φυσικά. Επίσης, υπάρχουν μηχανισμοί όπως τα \en{Ingress controllers} για την παροχή \en{HTTPS} τερματισμού (\en{TLS termination}) και δρομολόγησης (\en{routing}) σε υπηρεσίες εφαρμογών.

\subsection{Αποθήκευση (\en{Storage})}

Για την επίμονη αποθήκευση δεδομένων (\en{persistent data}), το \en{Kubernetes} ορίζει τα \en{Persistent Volumes (PV)} και τα \en{Persistent Volume Claims (PVC)}. Οι εφαρμογές (\en{pods}) ζητούν αποθηκευτικό χώρο δημιουργώντας ένα \en{PVC}, ενώ ο διαχειριστής (ή ένας μηχανισμός \en{dynamic provisioning}) αντιστοιχίζει ένα \en{PV} στο αίτημα αυτό. Έτσι, οι εφαρμογές μπορούν να κρατούν δεδομένα ανεξαρτήτως του \en{node} όπου εκτελούνται.

\subsection{Ασφάλεια (\en{Security})}

Το \en{Kubernetes} διαθέτει πολλαπλά επίπεδα ασφάλειας:
\begin{itemize}
    \item \textbf{\en{Role-Based Access Control (RBAC)}}: Καθορίζει ποιοι χρήστες και ποιες υπηρεσίες έχουν δικαίωμα να εκτελούν συγκεκριμένες ενέργειες στο \en{API}.
    \item \textbf{\en{Network Policies}}: Ορίζουν κανόνες για τη δικτυακή ροή (\en{ingress}, \en{egress}) σε επίπεδο \en{pod}, εμποδίζοντας τη μη εξουσιοδοτημένη επικοινωνία.
    \item \textbf{\en{Pod Security}}: Επιτρέπει ή απαγορεύει τη χρήση ευαίσθητων προνομίων (\en{privileged}, \en{root}) ή την πρόσβαση σε κρίσιμα τμήματα του λειτουργικού συστήματος.
\end{itemize}

\section{\en{Kubernetes} και ενοποίηση με \en{containers}, \en{VMs}, \en{IoT} και \en{CPS}}

\subsection{\en{Containers} και \en{Kubernetes}}

Η πιο διαδεδομένη χρήση του \en{Kubernetes} αφορά την ενορχήστρωση εφαρμογών που εκτελούνται σε \en{containers}. Ανάμεσα στα δημοφιλέστερα \en{container runtimes} συναντάμε το \en{Docker} και το \en{containerd}. Το \en{Kubernetes} διευκολύνει:
\begin{itemize}
    \item \textbf{Αυτόματη κλιμάκωση} (\en{autoscaling}): Αύξηση ή μείωση του αριθμού \en{pods} βάσει μετρικών (\en{metrics}), όπως η χρήση \en{CPU} ή οι αιτήσεις \en{HTTP}.
    \item \textbf{Αναβαθμίσεις χωρίς διακοπή} (\en{rolling updates}): Σταδιακή αντικατάσταση των \en{pods} με νέες εκδόσεις, διασφαλίζοντας τη συνεχή παροχή υπηρεσίας.
    \item \textbf{Ανθεκτικότητα} (\en{resilience}): Σε περίπτωση σφάλματος κάποιου \en{pod} ή ολόκληρου \en{node}, το \en{Kubernetes} επανεκκινεί αυτόματα τα \en{pods} σε υγιείς κόμβους.
\end{itemize}

\subsection{\en{Virtual Machines} και \en{KubeVirt}}

Παρά την έντονη έμφαση στους \en{containers}, κάποιες εφαρμογές ή περιπτώσεις χρήσης απαιτούν \en{virtual machines}. Το \en{KubeVirt} είναι ένα έργο ανοιχτού κώδικα που επιτρέπει την εκτέλεση \en{VMs} μέσα σε \en{Kubernetes}. Αυτό επιτυγχάνεται με την εισαγωγή \en{CRDs} (\en{Custom Resource Definitions}), μέσω των οποίων ορίζεται ένα \en{VirtualMachine} αντικείμενο στο \en{Kubernetes} \en{API}. 

Με αυτή τη μέθοδο:
\begin{itemize}
    \item \textbf{Υπάρχουσες \en{VM}-βασισμένες εφαρμογές} μπορούν να ενταχθούν στο \en{Kubernetes} οικοσύστημα, αξιοποιώντας τα οφέλη του (\en{autoscaling}, \en{monitoring}, \en{logging}).
    \item \textbf{Μεταβατικά σενάρια} μεταξύ \en{VMs} και \en{containers} γίνονται εφικτά, χωρίς να αλλάξει δραστικά η αρχιτεκτονική των συστημάτων.
    \item \textbf{Κοινή διαχείριση} πολιτικών, όπως η ασφάλεια και η δικτύωση, ενοποιείται για \en{pods} και \en{VMs}.
\end{itemize}

\subsection{\en{IoT} και \en{Edge Computing}}

Στα περιβάλλοντα \en{IoT}, η κατανεμημένη επεξεργασία δεδομένων (\en{edge computing}) αποκτά ιδιαίτερη σημασία, καθώς μειώνει τους χρόνους απόκρισης (\en{latency}) και το κόστος δικτυακής μετάδοσης. Εργαλεία όπως το \en{k3s} και το \en{MicroK8s} επιτρέπουν την εγκατάσταση ενός ελαφρού \en{Kubernetes} \en{distribution} σε \en{edge devices} με περιορισμένους πόρους. Κατ’ αυτόν τον τρόπο, οι \en{IoT} συσκευές (\en{sensors}, \en{actuators}) μπορούν να επικοινωνούν με τοπικούς κόμβους, οι οποίοι τρέχουν συλλογισμούς (\en{analytics}, \en{filtering}) σε \en{containers} εντός ενός μικρού \en{Kubernetes} \en{cluster}.

Παράλληλα, τα δεδομένα που προκύπτουν από την τοπική επεξεργασία μπορούν να προωθηθούν στο κεντρικό \en{cloud} \en{Kubernetes} \en{cluster}, το οποίο αναλαμβάνει περαιτέρω ανάλυση (\en{machine learning}, \en{big data analytics}) και αποθήκευση (\en{data warehousing}). Αυτή η λογική \en{edge-cloud} επιτρέπει στα \en{IoT} συστήματα να επεκταθούν πιο εύκολα, να αυτοματοποιηθούν και να κλιμακώσουν τις υπηρεσίες τους χωρίς να επιβαρύνουν τους κεντρικούς πόρους με μη επεξεργασμένα δεδομένα.

\subsection{\en{Cyber-Physical Systems (CPS)}}

Τα \en{CPS} χαρακτηρίζονται από την αλληλεπίδραση μεταξύ φυσικών διεργασιών και ψηφιακών υποδομών. Το \en{Kubernetes} βοηθάει σημαντικά στην αρχιτεκτονική των \en{CPS}, ιδιαίτερα όταν χρησιμοποιούνται επιμέρους υπηρεσίες (\en{microservices}) για τη συλλογή και επεξεργασία σημάτων από αισθητήρες ή τον έλεγχο ενεργοποιητών (\en{actuators}).

Σε ορισμένες περιπτώσεις, τα \en{CPS} έχουν απαιτήσεις χαμηλής καθυστέρησης (\en{low-latency}) ή πραγματικού χρόνου (\en{real-time}). Παρότι το \en{Kubernetes} δεν σχεδιάστηκε για \en{hard real-time} εφαρμογές, υπάρχουν τροποποιημένοι πυρήνες \en{Linux} (\en{RT kernels}) και ειδικές ρυθμίσεις \en{scheduler} που μπορούν να βελτιώσουν την απόδοση. Έτσι, για συστήματα \en{soft real-time} ή ημι-αυτόνομες βιομηχανικές ροές, το \en{Kubernetes} μπορεί να προσφέρει μια ευέλικτη λύση ενορχήστρωσης και διαχείρισης.

\section{Κλιμακούμενη ανάπτυξη και βέλτιστες πρακτικές}

\subsection{\en{Autoscaling} μηχανισμοί}

Το \en{Kubernetes} υποστηρίζει διάφορες στρατηγικές αυτόματης κλιμάκωσης (\en{autoscaling}):
\begin{itemize}
    \item \textbf{\en{Horizontal Pod Autoscaler (HPA)}}: Αυξομειώνει τον αριθμό των \en{pods} ενός \en{Deployment} ή \en{ReplicaSet} βάσει μετρικών (\en{CPU} / \en{memory} χρήση ή \en{custom metrics}).
    \item \textbf{\en{Cluster Autoscaler}}: Αυτόματη προσθήκη ή αφαίρεση \en{nodes} σε ένα \en{cloud} περιβάλλον, αναλόγως με τις ανάγκες των \en{pods}.
\end{itemize}
Σε περιπτώσεις \en{IoT} ή \en{CPS}, όπου η ροή δεδομένων μπορεί να παρουσιάζει μεγάλες διακυμάνσεις, οι μηχανισμοί αυτοί εξασφαλίζουν την ομαλή ανταπόκριση του συστήματος.

\subsection{\en{Rolling Updates} και \en{Canary Deployments}}

Για τη συνεχή παράδοση (\en{continuous delivery}) αναβαθμίσεων χωρίς να διακόπτεται η λειτουργία, το \en{Kubernetes} χρησιμοποιεί \en{rolling updates}, αντικαθιστώντας σταδιακά τα παλιά \en{pods} με νέα. Επιπλέον, τεχνικές \en{canary deployment} επιτρέπουν τη δοκιμή νέων εκδόσεων σε μικρό ποσοστό χρηστών, προτού γενικευθούν σε όλο το σύστημα. Αυτό είναι ιδιαίτερα χρήσιμο όταν ενσωματώνονται αλγόριθμοι \en{machine learning} ή νέες λειτουργίες σε ευαίσθητα συστήματα \en{CPS}.

\subsection{Παρακολούθηση (\en{Monitoring}) και καταγραφή (\en{Logging})}

Η διαχείριση σύνθετων \en{Kubernetes} \en{clusters} απαιτεί κατάλληλα εργαλεία παρακολούθησης (\en{monitoring}) και καταγραφής (\en{logging}). Συνήθως, χρησιμοποιούνται λύσεις όπως το \en{Prometheus} για τη συλλογή μετρικών, σε συνδυασμό με τον \en{Grafana} για την οπτικοποίηση. Για την ενιαία καταγραφή (\en{logging}), δημοφιλή είναι τα \en{EFK} (\en{Elasticsearch}, \en{Fluentd}, \en{Kibana}) ή το \en{Loki} της \en{Grafana Labs}. Με αυτόν τον τρόπο, οι διαχειριστές εντοπίζουν γρήγορα σφάλματα ή δυσλειτουργίες, τόσο σε εφαρμογές \en{containers} όσο και σε \en{VMs} (μέσω \en{KubeVirt}).

\section{Παραδείγματα χρήσης}

\subsection{\en{Smart Factory} σε βιομηχανικό περιβάλλον \en{Industry 4.0}}

Μια εργοστασιακή μονάδα παραγωγής (\en{smart factory}) μπορεί να υλοποιήσει συλλογή δεδομένων και αυτοματισμούς (αισθητήρες \en{IoT}, ρομπότ \en{CPS}) σε \en{edge devices}, τα οποία τρέχουν ελαφριές διανομές \en{Kubernetes}. Τοπικά \en{containers} επεξεργάζονται τα δεδομένα (π.χ. συνάγουν τάσεις, εντοπίζουν σφάλματα) και αποστέλλουν προ-επεξεργασμένα αποτελέσματα προς ένα κεντρικό \en{Kubernetes} \en{cluster} στο \en{cloud}. Το κεντρικό σύστημα χρησιμοποιεί \en{machine learning} για προβλέψεις συντήρησης (\en{predictive maintenance}), ενώ παράλληλα ενδέχεται να τρέχει «παλαιότερες» κρίσιμες εφαρμογές (\en{legacy SCADA}) σε \en{VMs} μέσω \en{KubeVirt}.

\subsection{\en{Healthcare} και \en{wearable IoT}}

Σε νοσοκομεία ή στην κατ’ οίκον φροντίδα ασθενών, \en{wearable} συσκευές (\en{IoT}) συλλέγουν ζωτικές ενδείξεις (πίεση, καρδιακοί παλμοί, θερμοκρασία). Τα δεδομένα μεταφέρονται σε \en{edge gateways} (\en{k3s}) που εκτελούν βασικές αναλύσεις για την άμεση ειδοποίηση ιατρικού προσωπικού σε περίπτωση ανωμαλιών. Η περαιτέρω επεξεργασία γίνεται σε ένα κεντρικό \en{Kubernetes} \en{cluster}, όπου τρέχουν υπηρεσίες \en{machine learning} (σε \en{containers}) και πιθανόν ειδικά \en{VMs} για εφαρμογές ηλεκτρονικού φακέλου υγείας (\en{EHR systems}) μέσω \en{KubeVirt}. Έτσι συνδυάζεται η ελαφρότητα και ταχύτητα του \en{edge} με την υπολογιστική ισχύ του \en{cloud}.

\section{Συμπεράσματα}

Το \en{Kubernetes} έχει αναδειχθεί σε ένα από τα πιο ευρέως χρησιμοποιούμενα εργαλεία για την ανάπτυξη, ενορχήστρωση και διαχείριση εφαρμογών μεγάλης κλίμακας. Η ομαλή ενσωμάτωση με τις τεχνολογίες \en{containers}, η δυνατότητα χρήσης \en{virtual machines} μέσω του \en{KubeVirt}, καθώς και η συμβατότητα με περιβάλλοντα \en{IoT} και \en{CPS}, καθιστούν το \en{Kubernetes} μια ευέλικτη και ισχυρή πλατφόρμα.

Σε ό,τι αφορά τις σύγχρονες απαιτήσεις για υλοποίηση \en{smart} συστημάτων, είτε πρόκειται για βιομηχανική παραγωγή (\en{Industry 4.0}), συστήματα υγείας (\en{Healthcare}) ή έξυπνες πόλεις (\en{Smart Cities}), το \en{Kubernetes} προσφέρει τους απαραίτητους μηχανισμούς για αυτοματοποιημένη κλιμάκωση, ανθεκτικότητα σε σφάλματα και ευέλικτη ανάπτυξη. Παράλληλα, τα εργαλεία ασφαλείας (\en{RBAC}, \en{Network Policies}, \en{Pod Security}) μπορούν να θωρακίσουν ευαίσθητες εφαρμογές από επιθέσεις ή κακόβουλη χρήση.

Συνολικά, η επιλογή του \en{Kubernetes} ως κεντρικής πλατφόρμας ενορχήστρωσης επιτρέπει την ενοποίηση ποικίλων υποσυστημάτων: από \en{containers} και \en{VMs} μέχρι σύνθετα \en{IoT}/\en{CPS} περιβάλλοντα. Με τη συνεχή εξέλιξη του οικοσυστήματος (π.χ. το \en{KubeVirt}, \en{k3s}, \en{operators}), οι δυνατότητες διαρκώς επεκτείνονται, διευκολύνοντας τη μετάβαση σε μια δυναμική, αυτοματοποιημένη υποδομή, ικανή να ανταποκριθεί στις υψηλές απαιτήσεις της σύγχρονης πληροφορικής και βιομηχανίας.
%!TEX root = ../../main.tex
\chapter{Μελέτη Περίπτωσης}\label{ch:case_study}
Σε αυτό το κεφάλαιο παρουσιάζεται μια λεπτομερής ανάλυση του αυτοματοποιημένου συστήματος παραγωγής που χρησιμοποιείται για την κατασκευή των καρεκλών γραφείου τύπου \en{Gregor}. Επικεντρώνεται στην αρχιτεκτονική του συστήματος, τις βασικές λειτουργίες και τα ρομποτικά στοιχεία που επιτρέπουν την αυτόματη συναρμολόγηση με υψηλή ακρίβεια και επαναληψιμότητα. Θα εξετάσουμε τις διάφορες διαδικασίες που εμπλέκονται, καθώς και το πώς το σύστημα διαχειρίζεται την ενσωμάτωση των διαφορετικών συναρμολογήσεων για να παράγει ένα τελικό προϊόν που συμμορφώνεται με τα υψηλά πρότυπα ποιότητας.

\section{Επισκόπηση Συστήματος}
\noindent Το σύστημα αυτοματοποιημένης παραγωγής που αναπτύχθηκε για την κατασκευή των καρεκλών γραφείου τύπου \en{Gregor} χαρακτηρίζεται από μια σειρά καλά δομημένων σταδίων συναρμολόγησης, τα οποία διαχειρίζονται με ακρίβεια από εξειδικευμένα ρομπότ. Κάθε ρομπότ είναι ανατεθειμένο με συγκεκριμένες εργασίες, σχεδιασμένες για να εκτελούνται με βάση τη διαθεσιμότητα και την κατάσταση των εξαρτημάτων στις εργασιακές θέσεις.

Οι διαδικασίες παραγωγής ξεκινούν με τη συναρμολόγηση των βασικών δομικών στοιχείων, όπως είναι οι βάσεις και τα πόδια των καρεκλών. Το Ρομπότ 1 αναλαμβάνει την εκτέλεση του αρχικού σταδίου συναρμολόγησης, τοποθετώντας και ενώνοντας τα πόδια με τη βάση. Στη συνέχεια, το Ρομπότ 2 εισέρχεται για να προσθέσει επιπλέον δομικά στοιχεία όπως τα ροδάκια και τον μηχανισμό ανύψωσης, ολοκληρώνοντας το σκελετό της καρέκλας.

Η διαδικασία συνεχίζεται με τη συναρμολόγηση του καθίσματος και της πλάτης, που γίνεται σε χωριστές φάσεις και απαιτεί την εναλλαγή των ρομπότ ανάλογα με το στάδιο που βρίσκεται η καρέκλα στην παραγωγική γραμμή. Η προσθήκη των μπράτσων και της τελικής συναρμολόγησης είναι επιφορτισμένη στο Ρομπότ 3, το οποίο διαχειρίζεται τα τελευταία στάδια της διαδικασίας με μεγάλη ακρίβεια.

Κάθε ρομπότ υποβάλλει αιτήματα για να λάβει πρόσβαση στις εργασιακές θέσεις των πάγκων και διασφαλίζει ότι τα εξαρτήματα βρίσκονται στην κατάλληλη κατάσταση πριν ξεκινήσει η κάθε διαδικασία. Αυτό το σύστημα επιτρέπει την αποφυγή λαθών και συγκρούσεων, βελτιώνοντας τη συνολική αποδοτικότητα και ποιότητα της παραγωγής.

\section{Στάδια Παραγωγής}
\noindent Η διαδικασία παραγωγής των καρεκλών γραφείου τύπου \en{Gregor} διαχωρίζεται σε καθορισμένα στάδια που εξασφαλίζουν την ακριβή και αποδοτική συναρμολόγηση κάθε μέρους της καρέκλας. Αρχικά, το στάδιο \textit{\en{S1}} υποδέχεται την εναρκτήρια δράση από το Ρομπότ 1, το οποίο συναρμολογεί τα πόδια με την κεντρική βάση, δημιουργώντας τον πρωταρχικό σκελετό της καρέκλας. Συνεχίζοντας, στο στάδιο \textit{\en{S2}}, το Ρομπότ 2 προσθέτει τα ροδάκια και τον μηχανισμό ανύψωσης, ενισχύοντας τη βάση για καλύτερη κινητικότητα και ρυθμιζόμενη στήριξη.

Η συνέχεια της κατασκευής μεταφέρεται ξανά στο Ρομπότ 1, το οποίο στο στάδιο \textit{\en{S3}} προετοιμάζει το κάθισμα ενσωματώνοντάς το με την πλάκα καθίσματος. Ακολούθως, στο στάδιο \textit{\en{S4}}, το Ρομπότ 2 αναλαμβάνει να ενώσει το κάθισμα με την προετοιμασμένη βάση, ολοκληρώνοντας την ενσωμάτωση των κυριότερων μερών της καρέκλας.

Τα τελευταία στάδια εντάσσουν τη δράση του Ρομπότ 3, όπου το \textit{\en{S5}} αφορά την προσθήκη της πλάτης της καρέκλας. Στο \textit{\en{S6}} και \textit{\en{S7}}, το ίδιο ρομπότ προσθέτει τα μπράτσα, το αριστερό και το δεξί αντίστοιχα, ενισχύοντας τη λειτουργικότητα και την άνεση του σχεδίου. Το τελικό στάδιο, \textit{\en{S8}}, σηματοδοτεί την πλήρη ολοκλήρωση της καρέκλας, με την κατασκευή να έχει ολοκληρωθεί πλήρως και την καρέκλα να είναι έτοιμη για χρήση.

Η οργανωμένη διαδικασία και η σειριακή ολοκλήρωση των σταδίων διασφαλίζουν ότι κάθε μέρος της καρέκλας συναρμολογείται με ακρίβεια και αποδοτικότητα, αποφεύγοντας επαναλήψεις ή παραλείψεις στην παραγωγική διαδικασία.

\section{Ρόλοι και Ευθύνες}

\subsection{Ρομπότ 1}
\noindent Το Ρομπότ 1 κατέχει κεντρικό ρόλο στην αυτοματοποιημένη γραμμή παραγωγής των καρεκλών γραφείου τύπου \en{Gregor}, με τις ευθύνες του να επικεντρώνονται σε κρίσιμες εργασίες συναρμολόγησης στην αρχική φάση της κατασκευής. Αρχικά, το Ρομπότ 1 εκτελεί την Εργασία Συναρμολόγησης 1 (\en{AT1}), όπου είναι υπεύθυνο για τη συναρμολόγηση της βάσης της καρέκλας, ενώνοντας τα πόδια με την κεντρική βάση στον πρώτο πάγκο εργασίας. Το αντίστοιχο διάγραμμα ροής του Ρομπότ 1 φαίνεται στο Σχήμα \ref{fig:robot1_flowchart}.

Κατόπιν, στην Εργασία Συναρμολόγησης 4 (\en{AT4}), το Ρομπότ 1 επιστρέφει για να ενσωματώσει το κάθισμα με την πλάκα καθίσματος, ρυθμίζοντας τα για την τελική τοποθέτηση. Η διαδικασία αυτή απαιτεί υψηλή ακρίβεια και είναι κρίσιμη για την άνεση και την ασφάλεια της τελικής κατασκευής.

Το Ρομπότ 1 έχει επίσης την ευθύνη να επαναλάβει τις εργασίες συναρμολόγησης 1 και 4 (\en{AT1} και \en{AT4}) σε κυκλική διαδικασία, εξασφαλίζοντας τη συνεχή παραγωγή και αποδοτική λειτουργία της γραμμής παραγωγής. Αυτή η κυκλική και επαναληπτική δράση διασφαλίζει ότι η παραγωγή δεν διακόπτεται και ότι κάθε καρέκλα παράγεται με συνέπεια σύμφωνα με τα υψηλά πρότυπα ποιότητας του συστήματος.

\begin{figure}[H]
    \centering
    \begin{tikzpicture}[node distance=1.6cm]
        \node (start) [start] {};
        \node (at1) [process, below of=start] {\en{Assembly Task 1}};
        \node (at4) [process, below of=at1] {\en{Assembly Task 4}};

        \draw [arrow] (start) -- (at1);
        \draw [arrow] (at1) -- (at4) node[midway,right,overlay] {\en{Move to PosB}};
        \draw [arrow] (at4.east) -| ++(1.5,0) -- node[midway,right,overlay] {\en{Move to PosA}} ($(at1.east)+(1.5,0)$) |- (at1.east);
    \end{tikzpicture}
    \caption{Διάγραμμα Ροής Ρομπότ 1}
    \label{fig:robot1_flowchart}
\end{figure}

\subsection{Ρομπότ 2}
\noindent Το Ρομπότ 2 εκτελεί ζωτικές λειτουργίες στην παραγωγή της καρέκλας \en{Gregor}, εστιάζοντας στη συναρμολόγηση βασικών εξαρτημάτων για τη λειτουργικότητα της καρέκλας.  Το αντίστοιχο διάγραμμα ροής του Ρομπότ 2 φαίνεται στο Σχήμα \ref{fig:robot2_flowchart}.

Στην Εργασία Συναρμολόγησης 2 (\en{AT2}), το Ρομπότ 2 τοποθετεί τα ροδάκια και τον μηχανισμό ανύψωσης στην βάση της καρέκλας στον πρώτο πάγκο εργασίας. Στην Εργασία Συναρμολόγησης 3 (\en{AT3}), το Ρομπότ 2 αναλαμβάνει την ενσωμάτωση της πλάκας καθίσματος στην βάση, απαιτώντας ακρίβεια στην τοποθέτηση για να διασφαλιστεί η άνεση και σταθερότητα του καθίσματος. Τέλος, στην Εργασία Συναρμολόγησης 5 (\en{AT5}), το Ρομπότ 2 συναρμολογεί το τελικό κομμάτι της καρέκλας, ενώνοντας το κάθισμα με το υπόλοιπο σκελετό για να ολοκληρώσει τη δομή της καρέκλας.

\begin{figure}[H]
    \centering
    \begin{tikzpicture}[node distance=1.6cm]
        \node (start) [start] {};
        \node (at2) [process, below of=start] {\en{Assembly Task 2}};
        \node (at3) [process, below of=at2] {\en{Assembly Task 3}};
        \node (at5) [process, below of=at3] {\en{Assembly Task 5}};

        \draw [arrow] (start) -- (at2);
        \draw [arrow] (at2) -- (at3) node[midway,right,overlay] {\en{Move to PosB}};
        \draw [arrow] (at3) -- (at5) node[midway,right,overlay] {\en{Move to PosA}};
        \draw [arrow] (at5.east) -| ++(1.5,0) -- ($(at2.east)+(1.5,0)$) |- (at2.east);
    \end{tikzpicture}
    \caption{Διάγραμμα Ροής Ρομπότ 2}
    \label{fig:robot2_flowchart}
\end{figure}

\subsection{Ρομπότ 3}
\noindent Το Ρομπότ 3 είναι αρμόδιο για την τελική φάση της συναρμολόγησης των καρεκλών γραφείου \en{Gregor}, ολοκληρώνοντας τη δομή με την προσθήκη κρίσιμων εξαρτημάτων για τη λειτουργικότητα και άνεση του τελικού προϊόντος. Το αντίστοιχο διάγραμμα ροής του Ρομπότ 3 φαίνεται στο Σχήμα \ref{fig:robot3_flowchart}.

Συγκεκριμένα, στην Εργασία Συναρμολόγησης 6 (\en{AT6}), το Ρομπότ 3 είναι υπεύθυνο για την τοποθέτηση της πλάτης της καρέκλας. Αυτή η εργασία απαιτεί ακρίβεια για να εξασφαλιστεί ότι η πλάτη στερεώνεται σωστά στον σκελετό, παρέχοντας την απαραίτητη στήριξη και άνεση.

Στα επόμενα στάδια, Εργασία Συναρμολόγησης 7 και Εργασία Συναρμολόγησης 8 (\en{AT7} και \en{AT8}), το Ρομπότ 3 προχωρά στην τοποθέτηση των μπράτσων της καρέκλας, το αριστερό και το δεξί αντίστοιχα. Κάθε μπράτσο πρέπει να ενσωματωθεί με ακρίβεια για να διασφαλιστεί η άνεση και η σταθερότητα της καρέκλας, καθώς και η εργονομία για τον τελικό χρήστη.

Οι εργασίες που εκτελεί το Ρομπότ 3 είναι ουσιαστικές για την ολοκλήρωση της κατασκευής της καρέκλας, διασφαλίζοντας ότι το τελικό προϊόν είναι πλήρως λειτουργικό και άνετο. Η ακριβής εκτέλεση αυτών των τελευταίων βημάτων καθορίζει την ποιότητα και την εμφάνιση της καρέκλας, καταστρώνοντας τη βάση για μια επιτυχημένη παραγωγή.

\begin{figure}[H]
    \begin{center}
    \begin{tikzpicture}[node distance=1.6cm]
        \node (start) [start] {};
        \node (at6) [process, below of=start] {\en{Assembly Task 6}};
        \node (at7) [process, below of=at6, left of=at6,xshift=-4] {\en{Assembly Task 7}};
        \node (at8) [process, below of=at6, right of=at6,xshift=4] {\en{Assembly Task 8}};
        \node (comp) [process, below of=at8, left of=at8] {\en{Complete}};

        \draw [arrow] (start) -- (at6);
        \draw [arrow] (at6) -- (at7);
        \draw [arrow] (at6) -- (at8);
        \draw [arrow] (at7) -- (comp);
        \draw [arrow] (at8) -- (comp);
        \draw [arrow] (comp.east) -| ++(2.5,0) -- ($(comp.east)+(2.5,0)$) |- (at6.east);
    \end{tikzpicture}
    \caption{Διάγραμμα Ροής Ρομπότ 3}
    \label{fig:robot3_flowchart}
    \end{center}
\end{figure}

\subsection{Πάγκος Εργασίας 1}
\noindent Ο Πάγκος Εργασίας 1 συνιστά ένα κρίσιμο στοιχείο στην αυτοματοποιημένη γραμμή παραγωγής των καρεκλών \en{Gregor}. Διαθέτει τρεις θέσεις εργασίας, οι οποίες ενεργοποιούνται συγκεκριμένα ανάλογα με τη φάση του προϊόντος που βρίσκεται σε καθεμιά. Το αντίστοιχα διαγράμματα ροής φαίνονται στα Σχήματα \ref{fig:workbench1_flowchart} και \ref{fig:workbench1_position_flowchart}.

Η πρώτη θέση (\en{W1P1}) χρησιμοποιείται για την Εργασία Συναρμολόγησης 1 (\en{AT1}), όπου το Ρομπότ 1 συναρμολογεί τα πόδια με τη βάση της καρέκλας, δημιουργώντας το προϊόν στάδιο \en{S1}. Η δεύτερη θέση (\en{W1P2}) διαχειρίζεται την Εργασία Συναρμολόγησης 2 (\en{AT2}) και την Εργασία Συναρμολόγησης 5 (\en{AT5}), όπου το Ρομπότ 2 προσθέτει τα ροδάκια και τον μηχανισμό ανύψωσης (\en{S2}) και στη συνέχεια το κάθισμα στον σκελετό της καρέκλας, ολοκληρώνοντας το στάδιο \en{S5}. Η τρίτη θέση (\en{W1P3}) είναι προορισμένη για την Εργασία Συναρμολόγησης 6 (\en{AT6}), όπου το Ρομπότ 3 τοποθετεί την πλάτη της καρέκλας, φτάνοντας στο στάδιο \en{S6}.

Οι συνθήκες περιστροφής του Πάγκου Εργασίας 1 είναι καθοριστικές για την εκκίνηση των εργασιών. Μετά την πρώτη περιστροφή, η θέση \en{W1P1} καθίσταται ελεύθερη, η \en{W1P2} φιλοξενεί το προϊόν μετά το \en{AT1} (\en{S1}) και η \en{W1P3} είναι ελεύθερη, εκκινώντας την εργασία \en{AT2}. Για τις επόμενες περιστροφές, απαιτείται η \en{W1P1} να έχει προϊόν μετά το \en{AT1} (\en{S1}), η \en{W1P2} να φιλοξενεί προϊόν μετά το \en{AT5} (\en{S5}) και η \en{W1P3} να είναι ελεύθερη, εκκινώντας τις εργασίες \en{AT2} και \en{AT6}.


\begin{figure}[H]
    \centering
    \begin{subfigure}{0.49\textwidth}
        \centering
        \begin{tikzpicture}
            \node (W1P1) [circle, draw, minimum size=1cm] {\en{S1}};
            \node (W1P2) [circle, draw, minimum size=1cm, right of=W1P1, xshift=2cm] {\en{S5}};
            \node (W1P3) [circle, draw, minimum size=1cm, between=W1P1 and W1P2, yshift=2cm] {\en{Free}};
            \node (W1P1Label) [below right of=W1P1, xshift=0.3cm, yshift=0.4cm] {\small \en{W1P1}};
            \node (W1P2Label) [below right of=W1P2, xshift=0.3cm, yshift=0.4cm] {\small \en{W1P2}};
            \node (W1P3Label) [below right of=W1P3, xshift=0.3cm, yshift=0.4cm] {\small \en{W1P3}};

            \draw [arrow] (W1P1) -- (W1P2);
            \draw [arrow] (W1P2) -- (W1P3);
            \draw [arrow] (W1P3) -- (W1P1);
        \end{tikzpicture}
        \caption{Πριν την Περιστροφή}
        \label{fig:workbench1_before_rotation}
    \end{subfigure}
    \begin{subfigure}{0.49\textwidth}
        \centering
        \begin{tikzpicture}
            \node (W1P1) [circle, draw, minimum size=1cm] {\en{Free}};
            \node (W1P2) [circle, draw, minimum size=1cm, right of=W1P1, xshift=2cm] {\en{S1}};
            \node (W1P3) [circle, draw, minimum size=1cm, between=W1P1 and W1P2, yshift=2cm] {\en{S5}};
            \node (W1P1Label) [below right of=W1P1, xshift=0.3cm, yshift=0.4cm] {\small \en{W1P1}};
            \node (W1P2Label) [below right of=W1P2, xshift=0.3cm, yshift=0.4cm] {\small \en{W1P2}};
            \node (W1P3Label) [below right of=W1P3, xshift=0.3cm, yshift=0.4cm] {\small \en{W1P3}};

            \draw [arrow] (W1P1) -- (W1P2);
            \draw [arrow] (W1P2) -- (W1P3);
            \draw [arrow] (W1P3) -- (W1P1);
        \end{tikzpicture}
        \caption{Μετά την Περιστροφή}
        \label{fig:workbench1_after_rotation}
    \end{subfigure}
    \caption{Διάγραμμα Ροής Πάγκου Εργασίας 1}
    \label{fig:workbench1_flowchart}
\end{figure}

\begin{figure}[H]
    \centering
    \begin{subfigure}{0.32\textwidth}
        \centering
        \begin{tikzpicture}[node distance=1.6cm]
            \node (start) [start] {};
            \node (free) [process, below of=start] {\en{Free}};
            \node (at1) [process, below of=free] {\en{Assembly Task 1}};

            \draw [arrow] (start) -- (free);
            \draw [arrow] (free) -- (at1);
            \draw [arrow] (at1.east) -| ++(1,0) -- ($(at1.east)+(1,0)$) |- (free.east);
        \end{tikzpicture}
        \caption{Διάγραμμα Ροής Θέσης 1}
        \label{fig:workbench1_flowchart_position1}
    \end{subfigure}
    \begin{subfigure}{0.32\textwidth}
        \centering
        \begin{tikzpicture}[node distance=1.6cm]
            \node (start) [start] {};
            \node (s1) [process, below of=start] {\en{S1}};
            \node (at2) [process, below of=s1] {\en{Assembly Task 2}};
            \node (at5) [process, below of=at2] {\en{Assembly Task 5}};

            \draw [arrow] (start) -- (s1);
            \draw [arrow] (s1) -- (at2);
            \draw [arrow] (at2) -- (at5);
            \draw [arrow] (at5.east) -| ++(1,0) -- ($(at5.east)+(1,0)$) |- (s1.east);
        \end{tikzpicture}
        \caption{Διάγραμμα Ροής Θέσης 2}
        \label{fig:workbench1_flowchart_position2}
    \end{subfigure}
    \begin{subfigure}{0.32\textwidth}
        \centering
        \begin{tikzpicture}[node distance=1.6cm]
            \node (start) [start] {};
            \node (s5) [process, below of=start] {\en{S5}};
            \node (at6) [process, below of=s5] {\en{Assembly Task 6}};
            \node (at78) [process, below of=at6] {\en{AT7 and AT8}};
            \node (comp) [process, below of=at78] {\en{Complete}};

            \draw [arrow] (start) -- (s5);
            \draw [arrow] (s5) -- (at6);
            \draw [arrow] (at6) -- (at78);
            \draw [arrow] (at78) -- (comp);
            \draw [arrow] (comp.east) -| ++(1,0) -- ($(comp.east)+(1,0)$) |- (s5.east);
        \end{tikzpicture}
        \caption{Διάγραμμα Ροής Θέσης 3}
        \label{fig:workbench1_flowchart_position3}
    \end{subfigure}
    \caption{Διάγραμμα Ροής Θέσεων Πάγκου Εργασίας 1}
    \label{fig:workbench1_position_flowchart}
\end{figure}

\subsection{Πάγκος Εργασίας 2}
\noindent Ο Πάγκος Εργασίας 2 στην παραγωγική γραμμή των καρεκλών \en{Gregor} είναι ένας εξειδικευμένος πάγκος με μοναδική θέση εργασίας, \en{W2P1}, η οποία διαχειρίζεται δύο κρίσιμες εργασίες συναρμολόγησης. Το Ρομπότ 1 χρησιμοποιεί αυτή τη θέση για την Εργασία Συναρμολόγησης 4 (\en{AT4}), όπου τοποθετείται το κάθισμα στη βάση του καθίσματος. Ακολούθως, το Ρομπότ 2 εκτελεί την Εργασία Συναρμολόγησης 3 (\en{AT3}) στην ίδια θέση, όπου συνδέει την πλάκα καθίσματος με τη βάση. Το αντίστοιχο διάγραμμα ροής του Πάγκο Εργασίας 2 φαίνεται στο Σχήμα \ref{fig:workbench2_flowchart}.

Η ευθύνη του πάγκου εργασίας για τη διαχείριση του αιτήματος πρόσβασης από τα ρομπότ είναι καθοριστική, καθώς απαιτείται ο συντονισμός για την ορθή και έγκαιρη διαθεσιμότητα της θέσης για κάθε εργασία. Αυτός ο πάγκος εξασφαλίζει ότι οι συγκεκριμένες εργασίες εκτελούνται με ακρίβεια και στη σωστή σειρά, παρέχοντας μια αποτελεσματική βάση για την εκτέλεση εξειδικευμένων εργασιών που κρίνονται απαραίτητες για τη λειτουργικότητα και την ασφάλεια της τελικής καρέκλας.


\begin{figure}[H]
    \centering
    \begin{tikzpicture}[node distance=1.6cm]
        \node (start) [start] {};
        \node (free) [process, below of=start] {\en{Free}};
        \node (at4) [process, below of=free] {\en{Assembly Task 4}};
        \node (at3) [process, below of=at4] {\en{Assembly Task 3}};

        \draw [arrow] (start) -- (free);
        \draw [arrow] (free) -- (at4);
        \draw [arrow] (at4) -- (at3);
        \draw [arrow] (at3.east) -| ++(1,0) -- ($(at3.east)+(1,0)$) |- (free.east);
    \end{tikzpicture}
    \caption{Διάγραμμα Ροής Πάγκου Εργασίας 2}
    \label{fig:workbench2_flowchart}
\end{figure}

\section{Εργασίες Συναρμολόγησης}
\noindent Στην ενότητα αυτή, θα εξετάσουμε λεπτομερώς τις διαδικασίες συναρμολόγησης των καρεκλών γραφείου τύπου \en{Gregor}, που αποτελούν κρίσιμο στοιχείο της αυτοματοποιημένης παραγωγικής γραμμής. Κάθε διαδικασία συναρμολόγησης έχει σχεδιαστεί για να εξασφαλίζει ακρίβεια και αποδοτικότητα, διασφαλίζοντας ότι το τελικό προϊόν πληροί τις υψηλές προδιαγραφές λειτουργικότητας που απαιτούνται.

Η διαδικασία συναρμολόγησης διαχωρίζεται σε διάφορα στάδια, με κάθε στάδιο να εμπλέκει την εκτέλεση συγκεκριμένων εργασιών από διαφορετικά ρομπότ. Αυτές οι εργασίες στοχεύουν στη σταδιακή κατασκευή της καρέκλας, από τη βάση μέχρι το τελικό κάθισμα με την πλάτη και τα μπράτσα. Η σωστή οργάνωση και η ακριβής ακολουθία των εργασιών είναι κρίσιμες για την ομαλή λειτουργία της γραμμής παραγωγής.

\subsection{Εργασία Συναρμολόγησης 1}
\noindent Η Εργασία Συναρμολόγησης 1 (\en{AT1}) αποτελεί το θεμελιώδες πρώτο στάδιο στη συναρμολόγηση των καρεκλών \en{Gregor} και διεξάγεται από το Ρομπότ 1 στην πρώτη θέση του Πάγκου Εργασίας 1 (\en{W1P1}). Αυτή η διαδικασία αποτελείται από τέσσερα βήματα και έχει ως στόχο την αρχική συναρμολόγηση της βάσης της καρέκλας.

\begin{enumerate}
    \item \textbf{Βήμα 1:} Το Ρομπότ 1 αναλαμβάνει τα πόδια της καρέκλας από τον Χώρο Αποθήκευσης 1 και τα μεταφέρει προς τον πάγκο εργασίας.
    \item \textbf{Βήμα 2:} Εφόσον λάβει άδεια πρόσβασης στη θέση \en{W1P1}, το ρομπότ τοποθετεί τα πόδια στην καθορισμένη θέση.
    \item \textbf{Βήμα 3:} Στη συνέχεια, το Ρομπότ 1 επιστρέφει στον Χώρο Αποθήκευσης 2 για να προμηθευτεί την κεντρική βάση της καρέκλας και την τοποθετεί πάνω στα πόδια.
    \item \textbf{Βήμα 4:} Κατά την τελική τοποθέτηση της βάσης, το Ρομπότ 1 εξασφαλίζει ότι όλες οι συνδέσεις είναι σταθερές και ακριβείς. Με την ολοκλήρωση αυτού του βήματος, ενεργοποιείται αυτόματα η Εργασία Συναρμολόγησης 4 (\en{AT4}).
\end{enumerate}

Η \en{AT1} εκκινείται αρχικά από το σύστημα κατά την έναρξη της παραγωγής και στη συνέχεια από την Εργασία Συναρμολόγησης 4 (AT4) κάθε φορά που ολοκληρώνεται. Αυτό διασφαλίζει ότι η αρχική βάση της καρέκλας συναρμολογείται σωστά και με την κατάλληλη σειρά, καθιστώντας το πρώτο στάδιο ζωτικής σημασίας για την ορθή εκκίνηση της διαδικασίας παραγωγής.

\subsection{Εργασία Συναρμολόγησης 2}
\noindent Η Εργασία Συναρμολόγησης 2 (\en{AT2}) πραγματοποιείται από το Ρομπότ 2 και εκτελείται στη δεύτερη θέση του Πάγκου Εργασίας 1 (\en{W1P2}). Αυτή η διαδικασία είναι απαραίτητη για την προσθήκη των ροδάκιων και του μηχανισμού ανύψωσης στην ήδη συναρμολογημένη βάση της καρέκλας. Η \en{AT2} αποτελείται από τέσσερα βήματα:

\begin{enumerate}
    \item \textbf{Βήμα 1:} Ενεργοποιείται όταν ο Πάγκος Εργασίας 1 περιστραφεί, φέρνοντας στη θέση \en{W1P2} το αντικείμενο από την \en{AT1}. Το Ρομπότ 2 παίρνει τα ροδάκια από τον Χώρο Αποθήκευσης 4 και ζητά άδεια να εργαστεί στη θέση \en{W1P2}.
    \item \textbf{Βήμα 2:} Ενεργοποιείται μόλις το Ρομπότ 2 λάβει άδεια για τη θέση \en{W1P2}. Συνδυάζει τα ροδάκια με το αντικείμενο σε αυτή τη θέση και ενεργοποιεί το επόμενο βήμα.
    \item \textbf{Βήμα 3:} Συνεχίζεται με το Ρομπότ 2 να παίρνει τον μηχανισμό ανύψωσης από τον Χώρο Αποθήκευσης 5 και να ζητά ξανά άδεια για εργασία στη θέση \en{W1P2}.
    \item \textbf{Βήμα 4:} Αφού λάβει άδεια, το Ρομπότ 2 συνδυάζει τον μηχανισμό ανύψωσης με το αντικείμενο στη θέση \en{W1P2} και με την ολοκλήρωση αυτού του βήματος ενεργοποιεί την Εργασία Συναρμολόγησης 3 (AT3).
\end{enumerate}

Η \en{AT2} εκκινείται από την περιστροφή του Πάγκου Εργασίας 1 που φέρνει στη θέση \en{W1P2} το αντικείμενο μετά την \en{AT1}. Το Ρομπότ 2 διαδραματίζει κεντρικό ρόλο στην επικύρωση και ενίσχυση της βάσης της καρέκλας, προετοιμάζοντας τη για τις επόμενες φάσεις συναρμολόγησης.

\subsection{Εργασία Συναρμολόγησης 3}
\noindent Η Εργασία Συναρμολόγησης 3 (\en{AT3}) εκτελείται από το Ρομπότ 2 και πραγματοποιείται κυρίως στον Πάγκο Εργασίας 2 (\en{W2P1}), ακολουθώντας την ολοκλήρωση της Εργασίας Συναρμολόγησης 4 (\en{AT4}). Αυτή η διαδικασία είναι κρίσιμη για την τελική σύνδεση και στερέωση της πλάκας καθίσματος με την κύρια συναρμολόγηση της καρέκλας. Η \en{AT3} αποτελείται από τέσσερα βήματα:

\begin{enumerate}
    \item \textbf{Βήμα 1:} Ξεκινά μετά την ολοκλήρωση της \en{AT2}, όπου το Ρομπότ 2 κινείται προς τη θέση Β και ζητά άδεια για εργασία στον Πάγκο Εργασίας 2 (\en{W2P1}), όπου αναμένει το αντικείμενο μετά την \en{AT4}.
    \item \textbf{Βήμα 2:} Ακολουθεί η ενεργοποίηση μετά την ολοκλήρωση της \en{AT4} και τη χορήγηση άδειας από τον \en{W2P1}. Σε αυτό το σημείο, το Ρομπότ 2 συναρμολογεί την πλάκα καθίσματος με τη βάση, εξασφαλίζοντας τη στερέωση των δύο μερών.
    \item \textbf{Βήμα 3:} Μετά τη συναρμολόγηση, το Ρομπότ 2 ζητά άδεια για να αναλάβει το αντικείμενο από τον \en{W2P1} και κατευθύνεται προς τον Πάγκο Εργασίας 1 (\en{W1P2}).
    \item \textbf{Βήμα 4:} Εφόσον λάβει άδεια για τη θέση \en{W1P2}, το Ρομπότ 2 τοποθετεί το αντικείμενο σε αυτή τη θέση και με την ολοκλήρωση αυτού του βήματος ενεργοποιεί την Εργασία Συναρμολόγησης 5 (\en{AT5}).
\end{enumerate}

Η \en{AT3} είναι ουσιαστική για την τελική φάση της κατασκευής του καθίσματος και την ένωση των κύριων συστατικών της καρέκλας. Αυτή η εργασία διασφαλίζει ότι το κάθισμα είναι σταθερά συνδεδεμένο με τη βάση, ετοιμάζοντας το προϊόν για τις τελευταίες ρυθμίσεις και προσθήκες πριν την τελική επιθεώρηση.

\subsection{Εργασία Συναρμολόγησης 4}
\noindent Η Εργασία Συναρμολόγησης 4 (\en{AT4}) είναι μια κρίσιμη διαδικασία στην παραγωγή των καρεκλών \en{Gregor}, εκτελείται από το Ρομπότ 1 και διαδραματίζει ρόλο στην τοποθέτηση και στερέωση της πλάκας του καθίσματος στην καρέκλα. Η διαδικασία πραγματοποιείται στην θέση του Πάγκου Εργασίας 2 (\en{W2P1}) και περιλαμβάνει τέσσερα βήματα:

\begin{enumerate}
    \item \textbf{Βήμα 1:} Αρχίζει με την ενεργοποίηση του Ρομπότ 1 μετά την ολοκλήρωση της Εργασίας Συναρμολόγησης 1 (\en{AT1}). Το ρομπότ μετακινείται προς τη θέση Β, αντιμετωπίζει τη μεταφορική ταινία 1 και παίρνει το αντικείμενο (κάθισμα). Στη συνέχεια, ζητά άδεια για να εργαστεί στη θέση \en{W2P1}.
    \item \textbf{Βήμα 2:} Ενεργοποιείται μόλις το Ρομπότ 1 λάβει άδεια για τη θέση \en{W2P1}. Το ρομπότ τοποθετεί το κάθισμα στη θέση, ετοιμάζοντάς το για την προσθήκη της πλάκας του καθίσματος και ενεργοποιεί το επόμενο βήμα.
    \item \textbf{Βήμα 3:} Συνεχίζεται με το Ρομπότ 1 να αντιμετωπίζει τον Χώρο Αποθήκευσης 3, απ' όπου παίρνει την πλάκα του καθίσματος και ζητά άδεια για να επισυνάψει το εξάρτημα στην ίδια θέση (\en{W2P1}).
    \item \textbf{Βήμα 4:} Με τη λήψη άδειας, το Ρομπότ 1 συνδυάζει την πλάκα του καθίσματος με το κάθισμα στη θέση \en{W2P1}, συνδέοντας σταθερά τα δύο μέρη. Με την ολοκλήρωση της συναρμολόγησης, το ρομπότ μετακινείται πίσω στη θέση Α και ενεργοποιεί εκ νέου την Εργασία Συναρμολόγησης 1 (\en{AT1}), συνεχίζοντας την παραγωγική διαδικασία.
\end{enumerate}

Η \en{AT4} είναι ζωτικής σημασίας για τη στερέωση των κρίσιμων συστατικών του καθίσματος και διασφαλίζει τη συνέχιση της ομαλής και αποδοτικής παραγωγής της καρέκλας \en{Gregor}.

\subsection{Εργασία Συναρμολόγησης 5}
\noindent Η Εργασία Συναρμολόγησης 5 (\en{AT5}) αποτελεί ένα κρίσιμο στάδιο στην τελική συναρμολόγηση της καρέκλας \en{Gregor}, όπου το κύριο σώμα της καρέκλας συνδέεται με το κάθισμα. Τη διαδικασία αυτή αναλαμβάνει το Ρομπότ 2 και εκτελείται στη δεύτερη θέση του Πάγκου Εργασίας 1 (\en{W1P2}). Η διαδικασία \en{AT5} αποτελείται από δύο βήματα:

\begin{enumerate}
    \item \textbf{Βήμα 1:} Ενεργοποιείται αμέσως μετά την ολοκλήρωση της Εργασίας Συναρμολόγησης 3 (\en{AT3}). Το Ρομπότ 2 ζητά άδεια για να εργαστεί στη θέση \en{W1P2}, όπου αναμένει το κύριο σώμα της καρέκλας μαζί με το προσαρμοσμένο κάθισμα.
    \item \textbf{Βήμα 2:} Μόλις λάβει άδεια για τη θέση \en{W1P2}, το Ρομπότ 2 συνδυάζει τα δύο μέρη, χρησιμοποιώντας εξειδικευμένα εργαλεία για να σφίξει και να εξασφαλίσει τη στερέωσή τους. Αυτό το βήμα είναι κρίσιμο για τη σωστή λειτουργία της καρέκλας, καθώς η σταθερότητα της συνδέσεως επηρεάζει άμεσα την άνεση και την ασφάλεια του χρήστη.
\end{enumerate}

Η εκκίνηση της \en{AT5} γίνεται αυτόματα μετά την ολοκλήρωση της \en{AT3}, όπου το κάθισμα έχει ήδη προσαρμοστεί στην βάση του καθίσματος και τοποθετηθεί ενώπιον του Ρομπότ 2. Το Ρομπότ 2, με αυτόν τον τρόπο, ολοκληρώνει τη σύνδεση των κύριων μερών της καρέκλας, επιτυγχάνοντας μια σταθερή και λειτουργική δομή που είναι έτοιμη για τις τελικές προσθήκες, όπως η πλάτη και τα μπράτσα.

\subsection{Εργασία Συναρμολόγησης 6}
Η Εργασία Συναρμολόγησης 6 (\en{AT6}) πραγματοποιείται από το Ρομπότ 3 και είναι κρίσιμη για την τοποθέτηση της πλάτης στη συναρμολογημένη καρέκλα. Αυτή η διαδικασία λαμβάνει χώρα στην τρίτη θέση του Πάγκου Εργασίας 1 (\en{W1P3}) και αποτελείται από δύο βήματα:

\begin{enumerate}
    \item \textbf{Βήμα 1:} Ενεργοποιείται μετά από τη δεύτερη περιστροφή του Πάγκου Εργασίας 1, οπότε και η \en{W1P3} πρέπει να περιέχει το αντικείμενο μετά την Εργασία Συναρμολόγησης 5 (\en{AT5}). Το Ρομπότ 3 αντιμετωπίζει τη μεταφορική ταινία 2 και παίρνει την πλάτη της καρέκλας. Στη συνέχεια, ζητά άδεια για να εργαστεί στη θέση \en{W1P3}.
    \item \textbf{Βήμα 2:} Με τη λήψη άδειας, το Ρομπότ 3 τοποθετεί την πλάτη στο κύριο σώμα της καρέκλας και εξασφαλίζει τη στερέωσή της. Αυτό το βήμα απαιτεί ακριβή τοποθέτηση και στερέωση για να διασφαλιστεί η άνεση και η ασφάλεια του τελικού προϊόντος. Με την ολοκλήρωση αυτού του βήματος, το Ρομπότ 3 ενεργοποιεί ταυτόχρονα τις Εργασίες Συναρμολόγησης 7 και 8 (\en{AT7} και \en{AT8}), που αφορούν την τοποθέτηση των μπράτσων της καρέκλας.
\end{enumerate}

Η \en{AT6} είναι ζωτικής σημασίας για την τελική εμφάνιση και λειτουργικότητα της καρέκλας, καθώς η πλάτη παρέχει στήριξη και άνεση για τον χρήστη. Η εκκίνηση αυτής της διαδικασίας από την ενεργοποίηση του Πάγκου Εργασίας 1 διασφαλίζει την ορθή σειρά συναρμολόγησης και την ομαλή ενσωμάτωση της πλάτης στο συνολικό σχεδιασμό της καρέκλας.

\subsection{Εργασία Συναρμολόγησης 7}
\noindent Η Εργασία Συναρμολόγησης 7 (\en{AT7}) είναι μία από τις τελικές φάσεις στην παραγωγή της καρέκλας \en{Gregor}, όπου το Ρομπότ 3 είναι υπεύθυνο για την τοποθέτηση του αριστερού μπράτσου στην καρέκλα. Αυτή η διαδικασία εκτελείται στην τρίτη θέση του Πάγκου Εργασίας 1 (\en{W1P3}) και αποτελείται από δύο βήματα:

\begin{enumerate}
    \item \textbf{Βήμα 1:} Ενεργοποιείται ταυτόχρονα με την Εργασία Συναρμολόγησης 8 (\en{AT8}) αφού ολοκληρωθεί η Εργασία Συναρμολόγησης 6 (\en{AT6}). Το Ρομπότ 3 κινείται προς τον Χώρο Αποθήκευσης 6 και παίρνει το αριστερό μπράτσο της καρέκλας. Στη συνέχεια, ζητά άδεια για να εργαστεί στη θέση \en{W1P3}.
    \item \textbf{Βήμα 2:} Με τη λήψη άδειας, το Ρομπότ 3 τοποθετεί το αριστερό μπράτσο στην καρέκλα, ενώνοντάς το με την υπόλοιπη κατασκευή. Η στερέωση πρέπει να είναι ακριβής και σταθερή, για να εγγυηθεί την άνεση και τη λειτουργικότητα του τελικού προϊόντος.
\end{enumerate}

Η \en{AT7} είναι ζωτική για την προσθήκη των τελευταίων λειτουργικών στοιχείων που καθορίζουν την άνεση και την εργονομική αξία της καρέκλας. Η ενσωμάτωση του μπράτσου γίνεται με προσοχή για να διασφαλιστεί η διαρκής αντοχή και η ασφάλεια κατά τη χρήση.

\subsection{Εργασία Συναρμολόγησης 8}
\noindent Η Εργασία Συναρμολόγησης 8 (\en{AT8}) είναι ένα από τα τελικά στάδια στη διαδικασία συναρμολόγησης της καρέκλας \en{Gregor}, όπου το Ρομπότ 3 αναλαμβάνει την τοποθέτηση του δεξιού μπράτσου. Αυτή η διαδικασία πραγματοποιείται στην τρίτη θέση του Πάγκου Εργασίας 1 (\en{W1P3}) και ενεργοποιείται ταυτόχρονα με την Εργασία Συναρμολόγησης 7 (\en{AT7}) ακριβώς μετά την ολοκλήρωση της Εργασίας Συναρμολόγησης 6 (\en{AT6}). Η διαδικασία \en{AT8} αποτελείται από τα εξής βήματα:

\begin{enumerate}
    \item \textbf{Βήμα 1:} Το Ρομπότ 3 κινείται προς τον Χώρο Αποθήκευσης 6 και αναλαμβάνει το δεξί μπράτσο της καρέκλας. Εν συνεχεία, ζητά άδεια για εργασία στη θέση \en{W1P3}, όπου πρόκειται να προσαρμόσει το μπράτσο στο κύριο σώμα της καρέκλας.
    \item \textbf{Βήμα 2:} Αφού λάβει την απαραίτητη άδεια, το Ρομπότ 3 τοποθετεί το δεξί μπράτσο στην καρέκλα, ενώνοντάς το με το σκελετό. Η σύνδεση πρέπει να γίνει με ακρίβεια για να διασφαλιστεί η στερέωση και η σωστή λειτουργία του τελικού προϊόντος.
\end{enumerate}

Η ολοκλήρωση της \en{AT8} εγγυάται ότι η καρέκλα είναι πλήρως συναρμολογημένη με τα μπράτσα, παρέχοντας την απαραίτητη στήριξη και ολοκληρώνοντας την κατασκευή της. Με την τοποθέτηση του τελευταίου μπράτσου, η καρέκλα είναι έτοιμη για τελική χρήση, προσφέροντας την απαιτούμενη στήριξη και σταθερότητα.
%!TEX root = ../../main.tex

\chapter{Αρχιτεκτονική Σχεδιασμού}\label{ch:architecture}


%!TEX root = ../../main.tex

\chapter{Υλοποίηση των υποσυστημάτων με χρήση \en{Kubernetes}}\label{ch:kubernetes_utilization}


%!TEX root = ../../main.tex

\chapter{Αξιοποίηση του \en{Apache Kafka}}\label{ch:apache_kafka}

%**************************%
%    END OF MAIN THESIS    %
%**************************%

% bibliography

% \renewcommand{\bibname}{\foreignlanguage{greek}{Βιβλιογραφία}}
% \selectlanguage{english}
% Set the bibliography language to English
% \selectbiblanguage{english}


% Just before printing the bibliography, switch to English
% \AtBeginBibliography{\selectlanguage{english}}
% \selectlanguage{english}

% Print the bibliography:
%   - The heading will appear in Greek ("Βιβλιογραφία"),
%   - The entries & automatically generated words (e.g., “and”, “In:”) will be in English.
% \begin{otherlanguage*}{english}
\selectlanguage{english}
\sloppy
\printbibliography[
  title={\foreignlanguage{greek}{Βιβλιογραφία}}
]
% \end{otherlanguage*}

% Redefine the bib heading in Greek
% \renewcommand{\bibname}{\foreignlanguage{greek}{Βιβλιογραφία}}
% \bibliographystyle{babunsrt}
% \bibliography{library/bibliography}

\clearpage

\end{document}

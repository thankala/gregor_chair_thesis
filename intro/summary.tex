% \pagestyle{plain}
\begin{center}
  {\LARGE \textbf{Περίληψη}}\\[1cm]
  \textbf{\doctitle}\\[1cm]
\end{center}

Η παρούσα διπλωματική εργασία εξετάζει την αξιοποίηση υπηρεσιών ιδιωτικού \en{cloud} στο πλαίσιο του Διαδικτύου των Αντικειμένων (\en{IoT}). Ειδικότερα, η εργασία επικεντρώνεται στη χρήση του \en{Kubernetes} ως υποδομή ιδιωτικού \en{cloud} για την ανάπτυξη και διαχείριση μικροϋπηρεσιών (\en{microservices}). Οι μικροϋπηρεσίες αυτές σχεδιάστηκαν και υλοποιήθηκαν με στόχο τον έλεγχο και την παρακολούθηση μιας γραμμής παραγωγής, επιδεικνύοντας την αποτελεσματικότητα και την ευελιξία που παρέχουν οι τεχνολογίες \en{cloud} σε βιομηχανικές εφαρμογές.

Η καινοτομία του συστήματος έγκειται στη χρήση κυβερνοφυσικών \en{microservices}, που λαμβάνουν και επεξεργάζονται πληροφορίες από τον φυσικό χώρο και κατόπιν αλληλεπιδρούν με ρομποτικούς βραχίονες μέσω \en{HTTP requests}. Κάθε \en{microservice} αντιπροσωπεύει ένα συγκεκριμένο βήμα παραγωγής, επιτυγχάνοντας ουσιαστικό διαχωρισμό \en{(decoupling)} ανάμεσα στη λογική των βημάτων και στους ρομποτικούς βραχίονες. Οι βραχίονες λειτουργούν απλώς ως απομακρυσμένα \en{API endpoints} που εκτελούν την κίνηση ή τη διεργασία που ζητούν τα \en{microservices}, χωρίς να χρειάζεται να γνωρίζουν οι ίδιοι τη σειρά ή τον τρόπο υλοποίησης των επιμέρους σταδίων. Η επικοινωνία μεταξύ των \en{microservices} πραγματοποιείται μέσω \en{Apache Kafka} εντός ενός \en{Kubernetes cluster}, εξασφαλίζοντας αξιόπιστη ροή δεδομένων και εντολών.

Η αρχιτεκτονική \en{microservices} επιτρέπει εύκολη κλιμάκωση, συντήρηση και επεκτασιμότητα, ενώ η εκτέλεσή τους σε ιδιωτικό \en{cloud} προσφέρει ασφάλεια και καλύτερο έλεγχο των δεδομένων. Τέλος, η χρήση του \en{Apache Kafka} για την ανταλλαγή μηνυμάτων εξασφαλίζει ακεραιότητα, υψηλή διαθεσιμότητα και ικανότητα χειρισμού μεγάλου φόρτου σε πραγματικές συνθήκες βιομηχανικής παραγωγής.

\clearpage

% \pagestyle{plain}
\begin{center}
  {\LARGE \textbf{\en{Extensive English Summary}}}\\[1cm]
  \textbf{\en{\engdoctitle}}\\[1cm]
\end{center}

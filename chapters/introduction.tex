\chapter{Εισαγωγή}\label{ch:introduction}

\markboth{Εισαγωγή}{}

\section{Αντικείμενο και στόχοι της διπλωματικής εργασίας}
Αντικείμενο της παρούσας εργασίας είναι η μελέτη και η υλοποίηση μιας παραδοσιακής γραμμής συναρμολόγησης στη βιομηχανία, αξιοποιώντας τεχνολογίες του Διαδικτύου των Αντικειμένων και των \en{Containers}. Πιο αναλυτικά, εξετάστηκαν οι δυσκολίες και προκλήσεις που αντιμετωπίζει η βιομηχανία σε μια παραδοσιακή γραμμή συναρμολόγησης και πώς τεχνολογίες ιδιωτικών υποδομών μπορούν να υλοποιηθούν. Ιδιαίτερα, ο έλεγχος των φυσικών διεργασιών μέσω \en{HTTP requests} βασίστηκε σε υλοποίηση προηγούμενης διπλωματικής εργασίας, πάνω στην οποία έγιναν επεκτάσεις και βελτιώσεις για τις ανάγκες της παρούσας έρευνας. Ταυτόχρονα, καταλήξαμε στη χρήση του \en{Apache Kafka} για τη ροή μηνυμάτων μεταξύ των \en{microservices}, παράλληλα με τους \en{Docker Containers} και την πλατφόρμα ορχήστρωσης \en{Kubernetes} ως ιδιωτική υποδομή, με στόχο να διερευνηθούν οι δυνατότητες αξιοποίησής τους στον συγκεκριμένο τομέα.

Η διαδικασία αυτή υλοποιείται μέσα από την ανάπτυξη ενός κατανεμημένου συστήματος συναρμολόγησης,
στο οποίο τα επιμέρους υποσυστήματα δεν είναι χωροταξικά συγκεντρωμένα και επικοινωνούν μεταξύ τους με
πρωτόκολλα επιπέδου εφαρμογής και ανταλλαγή μηνυμάτων. Η επικοινωνία αυτή επιτυγχάνεται μέσω λογισμικού
που αναπτύχθηκε και εκτελείται μέσα σε περιβάλλοντα \en{Containers}, τα οποία ορχηστρώνονται από το \en{Kubernetes}.

Το σύστημα συναρμολόγησης που χρησιμοποιήθηκε ως σενάριο μελέτης είναι το
\en{Gregor Office Chair Assembly System} \cite{thramboulidis_cyber-physical_2018},
το οποίο αποτελείται από τρεις ρομποτικούς βραχίονες και δύο τράπεζες εργασίας για τη συναρμολόγηση καρεκλών.
Για την προσομοίωση του συστήματος, κατασκευάστηκε μια πειραματική διάταξη,
ώστε να μελετηθούν οι δυνατότητες επικοινωνίας μεταξύ συσκευών με περιορισμένες υπολογιστικές δυνατότητες,
χρησιμοποιώντας \en{HTTP}, \en{Apache Kafka}, \en{Docker Containers} και \en{Kubernetes}.

Μέσα από την εν λόγω μελέτη, εκτιμάται ότι η εφαρμογή των συγκεκριμένων τεχνολογιών μπορεί να προσφέρει ποικίλα
πλεονεκτήματα στην ανάπτυξη λογισμικού για αντίστοιχα συστήματα στη βιομηχανία. Ενδεικτικά, η χρήση τους μπορεί να μειώσει το κόστος συναρμολόγησης, να επιταχύνει τον χρόνο ανάπτυξης του λογισμικού και να επιτρέψει την επαναχρησιμοποίηση ήδη υπάρχοντος κώδικα μέσω των \en{Containers} για τη δημιουργία νέων συστημάτων συναρμολόγησης. Τέλος, προέκυψαν διάφορα συμπεράσματα σχετικά με την εφαρμογή τους, τα οποία δείχνουν ότι η ενσωμάτωσή τους στη βιομηχανία μπορεί να αποφέρει σημαντικά οφέλη τόσο στη διαδικασία ανάπτυξης των συστημάτων όσο και στον τρόπο διαχείρισης αυτών από τους μηχανικούς που τα υποστηρίζουν.
\section{Σχετικό Έργο}
Στο πλαίσιο της παρούσας διπλωματικής εργασίας, ιδιαίτερα σημαντική υπήρξε η μελέτη και αξιοποίηση προηγούμενων
σχετικών διπλωματικών εργασιών, οι οποίες εστίασαν σε τεχνολογίες Διαδικτύου των Αντικειμένων (\en{IoT}),
\en{containers} και κυβερνοφυσικά συστήματα.

Η εργασία του φοιτητή Νικολάου Νικήτα με θέμα «Αξιοποίηση τεχνολογιών Διαδικτύου των Αντικειμένων και \en{containers}
σε κυβερνοφυσικά συστήματα» μελετά τις τεχνολογίες \en{IoT} και \en{containers} σε σχέση με την υλοποίηση μιας παραδοσιακής
γραμμής συναρμολόγησης στη βιομηχανία. Εξετάζονται τα πρωτόκολλα \en{CoAP} και \en{LwM2M}, καθώς και η χρήση
των \en{Docker containers}, με στόχο τη διερεύνηση της αξιοποίησής τους στον συγκεκριμένο τομέα. Σενάριο μελέτης
αποτέλεσε το \en{Gregor Office Chair Assembly System}, με πειραματική διάταξη για τη μελέτη των δυνατοτήτων επικοινωνίας
μεταξύ συσκευών με περιορισμένες υπολογιστικές δυνατότητες. Ο συγγραφέας καταλήγει στο ότι ο συγκεκριμένος σχεδιασμός
συμβάλλει στη μείωση του κόστους συναρμολόγησης και του χρόνου ανάπτυξης λογισμικού, ενώ διευκολύνει την
επαναχρησιμοποίηση λογισμικού μέσω των \en{containers} για νέα συστήματα συναρμολόγησης.
Η συγκεκριμένη διπλωματική εργασία αποτέλεσε για εμένα το βασικό υπόβαθρο για την κατανόηση τόσο της περίπτωσης μελέτης
\en{Gregor}, όσο και της αρχιτεκτονικής των \en{microservices}. Επιπλέον, βασίστηκα στις υλοποιήσεις της για να κατανοήσω
τον τρόπο με τον οποίο πραγματοποιείται ο συγχρονισμός μεταξύ των επιμέρους υπηρεσιών.

Η εργασία του φοιτητή Δημητρίου Σπυρίδων με τίτλο «Αξιοποίηση \en{Cyber-Physical microservices} σε συστήματα IoT»
επικεντρώνεται στη μελέτη τεχνολογιών \en{IoT} και στην υπηρεσιοκεντρική προσέγγιση για την αξιοποίησή τους σε
συστήματα συναρμολόγησης. Στο πλαίσιο αυτής της εργασίας υλοποιήθηκαν πρωτογενή \en{Cyber-Physical Microservices} που
ελέγχουν τα μέρη του συστήματος συναρμολόγησης, καθώς και σύνθετα \en{Cyber-Physical Microservices} που αξιοποιούν
τα πρωτογενή για την παροχή πιο σύνθετων λειτουργιών. Οι υπηρεσίες αυτές εξάγονται στο δίκτυο μέσω \en{HTTP} \en{endpoints},
τα οποία αποτελούν το \en{REST API} της υποδομής. Μέσω αυτού του \en{API}, είναι δυνατή τόσο η ενορχήστρωση των
υπηρεσιών (\en{service orchestration}) όσο και η ανάπτυξη χορογραφίας υπηρεσιών (\en{service choreography})
για την κατασκευή ενός συστήματος συναρμολόγησης. Από τη συγκεκριμένη διπλωματική εργασία χρησιμοποίησα τα \en{HTTP endpoints},
πάνω στην οποία βασίστηκα αλλά και βελτίωσα για τις ανάγκες της
ενορχήστρωσης των υπηρεσιών (\en{service orchestration}).
Συνοψίζοντας, τα συμπεράσματα και οι τεχνικές που αντλήθηκαν από τα παραπάνω έργα αποτέλεσαν θεμέλιο λίθο
για τη σχεδίαση και υλοποίηση του παρόντος συστήματος, το οποίο ενσωματώνει τις
πλέον σύγχρονες πρακτικές στον χώρο των κυβερνοφυσικών συστημάτων και της βιομηχανικής παραγωγής.

\section{Η μελέτη περίπτωσης}
Ως μελέτη περίπτωσης επιλέχθηκε το παράδειγμα ελέγχου φυσικών μερών για τη συναρμολόγηση μιας καρέκλας γραφείου, όπως απεικονίζεται στο Σχήμα 1.1. Στο σενάριο αυτό, ρομποτικοί βραχίονες και τράπεζες εργασίας συνεργάζονται ώστε να εκτελέσουν τις απαραίτητες εντολές για τη συναρμολόγηση του τελικού προϊόντος. Τα βασικά εξαρτήματα που απαιτούνται για τη συναρμολόγηση φτάνουν είτε μέσω ιμάντων μεταφοράς — στην περίπτωση των μεγαλύτερων εξαρτημάτων όπως το μαξιλάρι — είτε είναι ήδη τοποθετημένα σε θέσεις δίπλα στους ρομποτικούς βραχίονες για τα μικρότερα μέρη, όπως οι ρόδες.

Η κατανομή των εργασιών έχει σχεδιαστεί έτσι ώστε να αξιοποιούνται με τον βέλτιστο δυνατό τρόπο οι δυνατότητες των ρομποτικών βραχιόνων στη διαδικασία συναρμολόγησης μιας καρέκλας. Στο δοκιμαστικό περιβάλλον, για τρεις παραγγελίες καρεκλών, κάθε ρομποτικός βραχίονας αναλαμβάνει συγκεκριμένο στάδιο της συναρμολόγησης, επιτυγχάνοντας παράλληλη εκτέλεση των εργασιών και αυξάνοντας την αποδοτικότητα της γραμμής παραγωγής.

Η διαδικασία ξεκινά με το κάτω μέρος της καρέκλας, το οποίο αποτελείται από τη βάση με τα πόδια και τις ρόδες. Ο ρομποτικός βραχίονας 1 παραλαμβάνει τη βάση των ποδιών και την τοποθετεί στην πρώτη μέγγενη, όπου προσαρμόζονται τα πέντε πόδια ένα προς ένα, με τη μέγγενη να περιστρέφεται ώστε να επιτρέπει τη σταδιακή τοποθέτηση. Η περιστρεφόμενη τριγωνική τράπεζα εργασίας 1 στη συνέχεια μεταφέρει το ημιτελές προϊόν στον ρομποτικό βραχίονα 2, ο οποίος τοποθετεί τις ρόδες και το αμορτισέρ. Ακολουθεί το στάδιο τοποθέτησης του καθίσματος.

Καθώς ο ρομποτικός βραχίονας 2 ολοκληρώνει το έργο του, ο ρομποτικός βραχίονας 1 παράλληλα επεξεργάζεται το κάθισμα στη δεύτερη τράπεζα εργασίας. Μόλις είναι έτοιμο, το κάθισμα μεταφέρεται και τοποθετείται πάνω στο αμορτισέρ του ημιτελούς προϊόντος από τον ρομποτικό βραχίονα 2. Έπειτα, η τράπεζα εργασίας 1 περιστρέφεται ξανά, οδηγώντας το προϊόν μπροστά στον ρομποτικό βραχίονα 3, ο οποίος ολοκληρώνει τη διαδικασία συναρμολόγησης τοποθετώντας την πλάτη και τα μπράτσα της καρέκλας.

Είναι σημαντικό να σημειωθεί ότι μόλις ένας ρομποτικός βραχίονας ολοκληρώσει το έργο του για μία καρέκλα και εφόσον υπάρχει επόμενη παραγγελία, ξεκινά άμεσα τη διαδικασία για το επόμενο προϊόν. Σε περιπτώσεις που δεν υπάρχουν εξαρτήσεις, οι βραχίονες μπορούν να ξεκινούν άμεσα την εργασία τους, επιτρέποντας ακόμα μεγαλύτερη παράλληλη εκτέλεση. Η ταυτόχρονη συναρμολόγηση έως και τριών καρεκλών και η χρήση δύο τραπεζών εργασίας αποτελούν ενδεικτικά παραδείγματα αύξησης της αποδοτικότητας και της ευελιξίας της γραμμής παραγωγής.

\section{Οργάνωση του κειμένου}
Η δομή της παρούσας διπλωματικής εργασίας έχει ως εξής:

\begin{itemize}
  \item \textbf{Κεφάλαιο 1: Εισαγωγή} \
    Παρουσιάζεται το αντικείμενο, οι στόχοι και το γενικό πλαίσιο της εργασίας, καθώς και η μεθοδολογική προσέγγιση που ακολουθήθηκε.
  \item \textbf{Κεφάλαιο 2: Κυβερνοφυσικά Συστήματα και Διαδίκτυο των Αντικειμένων (\en{IoT})} \\
    Περιλαμβάνει ιστορική αναδρομή και θεωρητικές πληροφορίες σχετικά με τα κυβερνοφυσικά συστήματα και το \en{IoT}, αναδεικνύοντας τις βασικές αρχές, τις τεχνολογικές εξελίξεις και τις προκλήσεις στον τομέα.

  \item \textbf{Κεφάλαιο 3: Ιδιωτικές Υποδομές και Εικονοποίηση (\en{Virtualization})} \\
    Παρουσιάζονται βασικές έννοιες της εικονοποίησης και των ιδιωτικών υποδομών, με έμφαση σε τεχνολογίες όπως το \en{OpenStack} και το \en{Kubernetes}. Αναλύεται η διαδικασία επιλογής του \en{Kubernetes} ως ιδιωτική υποδομή και περιγράφεται η υλοποίηση του σε περιβάλλον εργοστασίου.

  \item \textbf{Κεφάλαιο 4: Αξιοποίηση Κυβερνοφυσικών Υπηρεσιών} \\
    Αναφέρεται στη μελέτη περίπτωσης της γραμμής παραγωγής της καρέκλας Gregor, παρουσιάζοντας τα επιμέρους βήματα και τη λειτουργία του συστήματος με αναλυτικό τρόπο καθώς και ο τρόπος με τον οποίο θα αξιοποιηθούν.

  \item \textbf{Κεφάλαιο 5: Υλοποίηση Ενορχήστρωσης και Αξιοποίηση της Ιδιωτικής Υποδομής} \\
    Περιγράφεται η διαδικασία ενορχήστρωσης, δίνοντας έμφαση στις τεχνικές υλοποίησης καθώς και την αξιοποίηση της ιδιωτικής υποδομής για τις μικρουπηρεσ

  \item \textbf{Κεφάλαιο 6: Συμπεράσματα και Μελλοντικές Επεκτάσεις} \\
    Παρουσιάζονται τα βασικά συμπεράσματα της εργασίας και προτείνονται ενδεχόμενες μελλοντικές κατευθύνσεις και προεκτάσεις για περαιτέρω έρευνα.

\end{itemize}
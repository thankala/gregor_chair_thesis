%!TEX root = ../../main.tex

\chapter{Εισαγωγή}\label{ch:introduction}


\markboth{Εισαγωγή}{}

\section{Αντικείμενο και στόχοι της διπλωματικής εργασίας}
Αντικείμενο της παρούσας εργασίας είναι η μελέτη και η υλοποίηση μιας παραδοσιακής γραμμής συναρμολόγησης στη βιομηχανία, αξιοποιώντας τεχνολογίες του Διαδικτύου των Αντικειμένων και των Containers. Πιο αναλυτικά, εξετάστηκαν οι δυσκολίες και προκλήσεις που αντιμετωπίζει η βιομηχανία σε μια παραδοσιακή γραμμή συναρμολόγησης και έγινε μελέτη σε διάφορα IoT 
πρωτόκολλ-α και μεθόδους επικοινωνίας. Έτσι, καταλήξαμε στη χρήση του HTTP για τον έλεγχο των φυσικών διεργασίων και του \en{Apache Kafka} για τη ροή μηνυμάτων μεταξύ των microservices, παράλληλα με τους Docker Containers και την πλατφόρμα ορχήστρωσης \en{Kubernetes} ως ιδιωτική υποδομή, με στόχο να διερευνηθούν οι δυνατότητες αξιοποίησής τους στον συγκεκριμένο τομέα.

Η διαδικασία αυτή υλοποιείται μέσα από την ανάπτυξη ενός κατανεμημένου συστήματος συναρμολόγησης, στο οποίο τα επιμέρους υποσυστήματα δεν είναι χωροταξικά συγκεντρωμένα και επικοινωνούν μεταξύ τους με πρωτόκολλα επιπέδου εφαρμογής και ανταλλαγή μηνυμάτων. Η επικοινωνία αυτή επιτυγχάνεται μέσω λογισμικού που αναπτύχθηκε και εκτελείται μέσα σε περιβάλλοντα \en{Containers}, τα οποία ορχηστρώνονται από το \en{Kubernetes}, σε συνδυασμό με τη χρήση \en{Microservices}.

Το σύστημα συναρμολόγησης που χρησιμοποιήθηκε ως σενάριο μελέτης είναι το \en{Gregor Office Chair Assembly System} \cite{thramboulidis_cyber-physical_2018}, το οποίο αποτελείται από τρεις ρομποτικούς βραχίονες και δύο τράπεζες εργασίας για τη συναρμολόγηση καρεκλών. Για την προσομοίωση του συστήματος, κατασκευάστηκε μια πειραματική διάταξη, ώστε να μελετηθούν οι δυνατότητες επικοινωνίας μεταξύ συσκευών με περιορισμένες υπολογιστικές δυνατότητες, χρησιμοποιώντας \en{HTTP}, \en{Apache Kafka}, \en{Docker Containers} και \en{Kubernetes}.

Μέσα από την εν λόγω μελέτη, εκτιμάται ότι η εφαρμογή των συγκεκριμένων τεχνολογιών μπορεί να προσφέρει ποικίλα πλεονεκτήματα στην ανάπτυξη λογισμικού για αντίστοιχα συστήματα στη βιομηχανία. Ενδεικτικά, η χρήση τους μπορεί να μειώσει το κόστος συναρμολόγησης, να επιταχύνει τον χρόνο ανάπτυξης του λογισμικού και να επιτρέψει την επαναχρησιμοποίηση ήδη υπάρχοντος κώδικα μέσω των \en{Containers} για τη δημιουργία νέων συστημάτων συναρμολόγησης. Τέλος, προέκυψαν διάφορα συμπεράσματα σχετικά με την εφαρμογή τους, τα οποία δείχνουν ότι η ενσωμάτωσή τους στη βιομηχανία μπορεί να αποφέρει σημαντικά οφέλη τόσο στη διαδικασία ανάπτυξης των συστημάτων όσο και στον τρόπο διαχείρισης αυτών από τους μηχανικούς που τα υποστηρίζουν.
\section{Μεθοδολογία}

\section{Οργάνωση του κειμένου}
